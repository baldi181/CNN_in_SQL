\chapter{Grundlagen neuronaler Netze}
\label{kapitel_neuralnetworks}

In diesem Abschnitt werden Künstliche Neuronale Netze \cite{dayhoff1990neural}, kurz KNN, als Forschungsgegenstand der Informatik eingeführt und deren mathematischen Grundlagen präzisiert. 
Sie stellen informationsverarbeitende Systeme nach dem Vorbild von tierischen beziehungsweise menschlichen Gehirnen dar und bestehen aus Neuronen in gewissen Zuständen und Schichten, die über gewichtete Verbindungen miteinander gekoppelt sind. Jene Gewichte sind als freie Parameter des neuronalen Netzes zu verstehen und können während des Trainingsprozesses so angepasst werden, um die entsprechende Aufgabe zu lösen.  
Gelingt dies, so können neuronale Netze genutzt werden, um bestimmte Muster in Daten, typischerweise in Bildern, Audio oder Stromdaten, effizient zu erkennen \cite{pandya1995pattern, pao1989adaptive, urbaniak2021quality}.
Sie eignen sich daher für viele typischen Aufgaben des maschinellen Lernens, beispielsweise für die Klassifikation digitalisierter Objekte.
Im ersten Abschnitt werden die Bestandteile eines neuronalen Netztes in ein mathematisches Modell überführt und im Detail erklärt. Diese Repräsentierung des künstlichen Netzwerks wird im weiteren Verlauf dieser Arbeit genutzt. 

\section{Das abstrakte Neuron}
\label{neuronabs}
Zunächst wird das Neuron als fundamentale und elementare Einheit eines neuronalen Netzes eingeführt.