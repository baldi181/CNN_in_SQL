\chapter{Grundlagen}
\label{kap:fund}

\section{Mathe/ ML Learning}

\section{Relationale Datenbanksysteme}
\label{abs:relation_intro}
Relationale Datenbanksysteme gehören zu den erfolgreichsten und verbreitetsten Datenbanken, welche zur elektronischen Datenverwaltung in Computersystemen eingesetzt werden. In diesem Abschnitt werden wichtige Grundbegriffe relationaler Datenbanksysteme erläutert und erklärt, wie Daten in Relationen repräsentiert und verarbeitet werden können. Die Notation und Bezeichnungen basieren auf Heuer et al.\cite{DBLP:books/daglib/0044627}. Zum Abfragen von Datenbeständen wird die Datenbanksparche SQL im Abschnitt \ref{abs:SQL_intro} eingeführt und deren theoretische Grundlage im Abschnitt \ref{abs:rela_algebra} beleuchtet. Schließlich werden im Abschnitt \ref{abs:SQL_linalg} Methoden vorgestellt, um Objekte der linearen Algebra als Relationen darzustellen und damit verbundene Operationen, beispielsweise die Matrixvektormultiplikation, datenbankgestützt umzusetzen.

\subsection{Das Relationenmodell}
Der Grundbaustein relationaler Datenbanksysteme bildet die Relation. Sie stellt eine mathematische Beschreibung einer Tabelle, welche aus Attributen und zugehörigen Domänen besteht, dar.

\begin{defi}[Universum, Attribut, Domäne]
    \label{def:universum}
    Bezeichne die endliche Menge $\mathcal{U} \neq \emptyset$ das Universum. Ein Element $A \in \mathcal{U}$ heißt Attribut. Für $m \in \mathbb{N}$ sei $\mathcal{D}=\{D_1, \ldots, D_m\}$ eine Menge nichtleere Mengen. Ein Element $D_i \in \mathcal{D}$ wird Domäne genannt. Für eine Funktion $\mathrm{dom}: \mathcal{U} \rightarrow \mathcal{D}$ bezeichne $\mathrm{dom}(A)$ den Wertebereich von $A$ und $w \in \mathrm{dom}(A)$ ein Attributwert.
\end{defi}

Nun können das Relationenschema und zugehörige Begriffe wie Relation und Tupel definiert werden.

\begin{defi}[Relationenschema, Relation, Tupel]
    \label{def:relation}
    Eine Menge $R \subseteq \mathcal{U}$ heißt Relationenschema über dem Universum $\mathcal{U}$. Für $R=\{A_1, \ldots, A_n \}$ ist eine Relation $r$ über $R$, kurz $r(R)$, als eine endliche Menge von Abbildungen
    \begin{equation*}
        t:R \rightarrow \bigcup_{i=1}^m D_i
    \end{equation*}
    definiert. Dabei gilt $t(A) \in \mathrm{dom}(A)$. Die Abbildungen $t$ werden Tupel genannt und mit $t(A)$ ist die Restriktion der Abbildung $t$ auf $A \in R$ gemeint.
\end{defi}

Vereinfacht gesagt, setzt sich eine Datenbank als Menge von Relationen und ein Datenbankschema als Menge der zugehörigen Relationenschemata zusammen.

\begin{defi}[Datenbank, Datenbankschema, vgl.\cite{DBLP:books/daglib/0044627}]
    Für $p \in \mathbb{N}$ ist eine Menge von Relationenschemata $S=\{R_1, \ldots, R_p\}$ als Datenbankschema definiert. Eine Datenbank $d$ über dem Schema $S$, kurz $d(S)$, ist eine Menge von Relationen
    \begin{equation*}
        d=\{r_1, \ldots, r_p \}
    \end{equation*}
    mit $r_i(R_i)$ für $1 \leq i \leq p$. Eine Relation $r \in d$ wird Basisrelation genannt.
\end{defi}

Weiter können Beziehungen zwischen Attributen und Relationen definiert werden. Für diese Arbeit ist der Begriff des Schlüsselattributs wesentlich.

\begin{defi}[Schlüssel]
    \label{def:key}
    Sei $R$ ein Relationenschema und $K=\{B_1, \ldots, B_k\} \subseteq R$. Gilt für jede Relation $r(R)$ die Beziehung
    \begin{equation*}
         \forall t_1, t_2 \in r: \; [ t_1 \neq t_2 \Rightarrow \exists B \in K: \, t_1(B) \neq t_2(B)],
    \end{equation*}
    so wird $K$ identifizierende Attributmenge genannt. Ein Schlüssel ist eine bezüglich der Mengeninklusion $\subset$ minimal identifizierende Attributmenge. Die Attribute eines Schlüssels werden Primattribute genannt.
\end{defi}

Mit diesen Begriffen lässt sich das relationale Datenbanksystem definieren.

\begin{defi}
    Ein relationales Datenbanksystem ist eine Kombination aus Datenbank und Datenbankmanagementsystem, wobei letzteres zur Verwaltung der Daten verwendet wird.
\end{defi}

Das Managementsystem ist als abgekapseltes Softwaremodul zu interpretieren, welches bestimme Funktionen zur Verwaltung der Datenbank unter gewissen Anforderungen liefert. Typischerwiese sind die von Edgar F. Codd etablierten Anforderungspunkte, siehe \cite{DBLP:books/daglib/0044627}, von einem Datenbankmanagementsystem umzusetzen. Eine der geforderten Funktion bildet die Anfrage an eine Datenbank, um Daten auslesen zu können. Dabei sei bemerkt, dass je nach Datenbanksystem die Anfragebearbeitung unterschiedlich abläuft. Für eine vertiefende Analyse von Architekturen, Funktionalitäten und Implementierungsmöglichkeiten von relationalen Datenbanksystemen sei auf Heuer et. al.\cite{DBLP:books/mitp/HSS19, DBLP:books/daglib/0044627} verwiesen. 

Im Folgenden wird die Relationenalgebra als Anfragesprache vorgestellt. Sie bildet die theoretische Grundlage der weitverbreiteten Anfragesprache SQL, welche im Folgeabschnitt \ref{abs:SQL_intro} im Fokus steht. Zusammen mit der erweiterten Relationalgebra gelingt im Abschnitt \ref{abs:SQL_linalg} die Umsetzung von Basisoperationen der linearen Algebra in SQL.

\subsection{Die Relationenalgebra}
\label{abs:rela_algebra}
In der Relationenalgebra werden Relationen als abstrakte Datentypen mit darauf definierten Operationen definiert. Eine Anfrage ist eine Komposition von Operatoren aus einem gewissen Operatorensystem. Ein geeignetes System ist $\omega= \{ \pi, \sigma, \bowtie, \cup, \setminus, \beta \}$, welches im Folgenden definiert wird.

\begin{itemize}
    \item Für eine relation $r(R)$ mit Tupeln $t$ wird die Projektion $\pi_X(r)$ auf das Attribut $X \subseteq R$ durch
    \begin{equation*}
        \pi_X(r):=\{t(X) \; | \; t \in r \}
    \end{equation*}
    definiert. 
    \item Die Konstantenselektion $\sigma_{X \theta c}$ ist als
    \begin{equation*}
        \sigma_{X \theta c}(r):=\{t \; | \; t \in r \wedge t(X) \; \theta \; c\}
    \end{equation*}
    definiert.
    Hierbei ist $\theta \in \{=, \neq\}$ möglich und bei Wertebereichen, welche mit einer Halbordnung ausgestattet sind, ist $\theta \in \{\leq, <, \geq, >, = ,\neq\}$ möglich. 
    Bei Attributen mit demselben Wertebereich ist die Attributselektion $\sigma_{X \theta Y}$ für $X, Y \subseteq R$ als
    \begin{equation*}
        \sigma_{X \theta Y}(r):=\{t \; | \; t \in r \wedge t(X) \; \theta \; t(Y)\}
    \end{equation*}    
    definiert. Zudem können mehrere Selektionsbedingungen beliebig logisch mit $\wedge, \vee $ und $\neg$ in $F$ verknüpft werden. Die Selektion $\sigma$ ist eine Konstanten- oder Attributselektion.
    \item Sind $r_1(R_1)$ und $r_2(R_2)$ Relationen, so verbindet der natürliche Verbund $\bowtie$ Tupel der beiden Relationen mit gleichen Attributwerten von gleichnamigen Attributen, also 
    \begin{equation*}
        r_1 \bowtie r_2 := \{t \; | \; t(R_1 \cup R_2) \wedge \exists t_1 \in r_1: t_1=t(R_1) \wedge \exists t_2 \in r_2: t_2=t(R_2)\}.
    \end{equation*} Ist $R_1=R_2$ so wird $\bowtie$ zum mengentheoretischen Durchschnitt und für $R_1 \cap R_2=\emptyset$ ergibt sich das kartesische Produkt von $r_1$ und $r_2$.
    \item Die Vereinigung zweier Relationen $r_1(R)$ und $r_2(R)$ über dem Relationenschema $R$ ist durch
    \begin{equation*}
        r_1 \cup r_2:=\{ t \; | \; t \in r_1 \wedge t \in r_2 \}
    \end{equation*}
    definiert.
    \item Die Differenz zweier Relationen $r_1(R)$ und $r_2(R)$ über dem Relationenschema $R$ ist als
    \begin{equation*}
        r_1 \setminus r_2:=\{ t \; | \; t \in r_1 \wedge t \notin r_2 \}
    \end{equation*}
    definiert.
    \item Die Umbenennung $\beta$ wird für die obigen Operationen benötigt, da diese von der Attributbenennung abhängen. Für $A \in R, B \notin (R \setminus \{A\})$ sei $R':=(R \setminus \{A\}) \cup B$. Die Umbenennung $\beta$ von $A$ zu $B$ in $r(R)$ ist für $\mathrm{dom}(A)=\mathrm{dom}(B)$ als
    \begin{equation*}
        \beta_{B \leftarrow A}:=\{t' \; | \; \exists t \in r: t'(R \setminus \{A\})=t(R \setminus \{A\}) \wedge t'(B)=t(A)\}
    \end{equation*}
    erklärt.
\end{itemize}
Es sei darauf hingewiesen, dass weitere Operatorensysteme existieren, welche zu $\omega$ äquivalent sind\cite{DBLP:books/daglib/0044627}.

\subsection*{Erweiterung der Relationenalgebra}

Mit dem Operatorensystem $\omega$ von oben sind noch keine arithmetischen Berechnungen über Attribute möglich, welche jedoch zwingend für die Umsetzung linearer Algebra benötigt werden. Daher wird im Folgenden die Relationenalgebra aus Abschnitt \ref{abs:rela_algebra} erweitert. In dieser Arbeit wird oft die Gruppierung von Tupeln bezüglich gleicher Attributwertkombinationen benötigt und daher ein Gruppierungsoperator $\gamma$ eingeführt. Darüber hinaus wird die Projektion $\pi$ erweitert, um tupelweise Funktion, z.B. die Summierung \textbf{SUM}, aus Attributen verwenden zu können.
Dazu werden die bisher eingeführten Operatoren leicht modifiziert, indem sie auf Multimengen definiert werden. Dafür sei an dieser Stelle auf \cite{DBLP:books/daglib/0020812} verwiesen, um die angepasste Definition des Operatorensystems $ \omega$ und die zugrunde liegende Theorie nachzuvollziehen.
\subsubsection*{Der Gruppierungsopeartor $\gamma$}
Der Operator $\gamma_L(r)$ dient zur Aggregation von Attributwerten in Gruppen für eine gegebene Relation $r$. Die Liste $L$ kann aus Attributen der Relation $r$ oder aus bestimmten Aggregatfunktionen bestehen, welche auf den jeweiligen Attributen berechnet werden. Die Ergebnisrelation besteht aus Tupeln von $r$, welche in Gruppen partitioniert ist. Dabei ergeben sich die jeweiligen Gruppen durch gleiche Attributwertkombinationen der Gruppierungsattribute in $L$. Es wird für jede Gruppe ein Tupel berechnet, welches dann aus den Attributwerten der Attribute aus $L$ oder den Ergebnissen der Aggregatfunktionen aus $L$ besteht. Folgendes Beispiel illustriert die Funktionsweise des Gruppierungsoperator $\gamma$. Dazu sei die Relation A in Tabellenform \ref{tab:bsp_group} gegeben.
\begin{table}[h]
    \centering
    \begin{tabular}{|c|c|c|} \hline
        \multicolumn{3}{|c|}{\textbf{A}} \\ \hline
        \hline
        i &j &v\\
        \hline
        1 &1 &3\\
        \hline
        1 &2 &1\\
        \hline
        2 &1 &4\\
        \hline
        2 &2 &1\\
        \hline
    \end{tabular}
    \caption[]{Zu sehen ist die Beispielrelation bestehend aus den Attributen $i, j$ und $v$.}
    \label{tab:bsp_group}
\end{table}
Die Relation \textbf{A} kann als Matrix 
\begin{equation*}
    A=\begin{pmatrix}
        3 &1\\
        4 &1
    \end{pmatrix}
\end{equation*}
interpretiert werden. Weitere Ausführungen dazu werden im späteren Abschnitt \ref{abs:SQL_linalg} vorgestellt. Die Spaltensumme von $A$ kann mit der Gruppierungsoperationen
\begin{equation*}
    \gamma_{j, \; \mathbf{SUM}(v) \rightarrow v}(\mathbf{A}) \Rightarrow 
    \begin{tabular}{|c|c|} \hline
        j &v\\
        \hline
        1 &7\\
        \hline
        2 &2\\
        \hline
    \end{tabular}
\end{equation*} 
berechnet werden. Mit dem Pfeil ist die Umbenennung $\beta$ gemeint.

\subsubsection*{Der erweiterte Projektionsoperator}
Desweiteren soll der Projektionsoperator $\pi_F(r)$ auf tupelweise arithmetische Berechnungen bezüglich einer oder mehrer Attribute einer Relation $\mathbf{r}$ erweitert werden. Dazu wird wieder eine Liste $F$ genutzt, welche 
\begin{itemize}
    \item Attribute der Relation $r$,
    \item Umbenennungen der Form $x \rightarrow y$, wobei $x$ eine Attribut der Relation $r$ ist, oder
    \item Ausdrücke der Form $E \rightarrow z$, wobei $E$ aus Attributen der Relation $r$, Konstanten oder arithmetischen Operationen besteht,
\end{itemize}
beinhalten kann. Die Funktionsweise des erweiterten Projektionsoperators wird an der Beispielrelation \ref{tab:bsp_group} beleuchtet. Die Matrixaddition
\begin{equation*}
    2 \cdot A + \begin{pmatrix}
        1 & 1\\
        1 &1
    \end{pmatrix}
\end{equation*}
lässt sich durch
\begin{equation*}
    \pi_{i,\; j, \; 2*v+1 \rightarrow v}(\mathbf{A}) \Rightarrow
    \begin{tabular}{|c|c|c|} \hline
        i &j &v\\
        \hline
        1 &1 &7\\
        \hline
        1 &2 &3\\
        \hline
        2 &1 &9\\
        \hline
        2 &2 &3\\
        \hline
    \end{tabular}
\end{equation*}
berechnen. Für diese Arbeit genügt die erweiterte Relationenalgebra mit der Gruppierung $\gamma_L$ und der erweiterten Projektion $\pi_F$, um fundamentale Operationen der linearen Algebra in der Datenbanksprache SQL darzustellen. Diese wird im Folgenden Abschnitt vorgestellt.
\subsection{Die Anfragesprache SQL}
\label{abs:SQL_intro}
SQL als Abkürzung für \textit{Structured Query Language} ist eine Datenbanksparche zur Definition von Datenstrukturen in relationalen Datenbanksystemen. Sie wird genutzt, um Datenbestände zu bearbeiten und abzufragen und basiert dabei auf der im vorherigen Abschnitt eingeführten Relationenalgebra. SQL wird als standardisierte Sprache in allen kommerziellen und frei zugänglichen Datenbankmanagementsystemen unterstützt. Da die Sprache unter Mitwirkung von Normungsgremien wie des \textit{American National Standard Institute} (ANSI) und der \textit{International Organization for Standardization} (ISO) seit 1986 standardisiert ist, ist es möglich, über die Grenzen spezieller Systeme hinaus SQL als einheitliche Datenbanksprache zu benutzen. Für eine detailliertere Darstellung der einzelen Standards sei auf die entsprechende Literatur verwiesen. Ein guter Überblick ist in Heuer et. al.\cite{DBLP:books/daglib/0044627} zu finden.

SQL ist aus mehreren Teilsprachen aufgebaut, welche jeweils unterschiedliche Aufgaben des Datenbankmanagementsystems hinsichtlich der Datendefinition, Dateiorganisation und Anfragebearbeitung übernehmen. Um Methoden der linearen Algebra in Sequenzen von SQL-Anfragen darzustellen, wird der Anfrageteil \textit{Interactive Query Language}, kurz IQL, als deklarative Programmiersprache im Folgenden beleuchtet. Diese bildet das Fundament der Datenselektion und wird zur Analyse von Daten in relationalen Systemen genutzt. Eine Anfrage hat die Form 

\begin{align}
    \label{anf:stand}
    \begin{split}
    & \mathbf{SELECT} \; \text{Projektionsliste }\\
    & \mathbf{FROM} \; \text{Relationenliste } \\
    & [\mathbf{WHERE} \; \text{Bedingung}].
\end{split}
\end{align}
Das Resultat einer Anfrage ist wieder eine Relation. Daher ist es möglich, verschachtelte Anfragen zu formulieren. Die \textbf{SELECT}-Anweisung gibt durch die Attribute in der Projektionsliste das Schema der Ergebnistabelle vor. Die \textbf{FROM}-Klausel beinhaltet die Relationenliste, in welcher die Relationen aufgeführt sind, aus denen Attribute selektiert werden sollen. Die Liste kann aus einer oder aus mehreren Unteranfragen bestehen, welche optional durch Operationen wie dem natürlichen Verbund oder dem kartesischem Produkt kombiniert werden können. Ebenfalls optional ist die \textbf{WHERE}-Klausel, welche Selektionsbedingungen bezüglich Konstanten oder Attributen enthält. 

Um dden Anfragemechanismus nachzuvollziehen, seien im Folgenden zwei Relationen \textit{Angestellte} und \textit{Projekt} wie in den Tabellen \ref{abb:angestellte} und \ref{abb:projekt} gegeben.
\begin{table}
    \centering
\begin{tabular}{|c|c|c|c|c|} \hline
    \multicolumn{5}{|c|}{\textbf{Angestellte}} \\ \hline
    \hline
    ID &Name &Spezialisierung &Projektnummer &Gehalt\\ 
    \hline
    1 &Martin &Elektrotechnik &3 &2300\\ 
    \hline
    2 &Lennardt &Informatik &1 &1500\\
    \hline
    3 &Johann &Informatik &3 &1800\\
    \hline
    4 &Anna &Buchhaltung &3 &2000\\
    \hline
    5 &Antonia &Buchhaltung &2 &2000\\ 
    \hline
\end{tabular}
\caption{Abgebildet ist die Beispielrelation Angestellte mit den Attributen ID, Name, Spezialisierung, Projektnummer und Gehalt.}
\label{abb:angestellte}
\end{table}
\begin{table}
    \centering
\begin{tabular}{|c|c|c|c|c|} \hline
    \multicolumn{5}{|c|}{\textbf{Projekt}} \\ \hline
    \hline
    Projektnummer &Projektname &Budget &Ort &Status\\ 
    \hline
    1 &Datenbank 2.0 &50000 &Rostock &abgeschlossen\\ 
    \hline
    2 &Verwaltung &25000 &Rostock &offen\\
    \hline
    3 &Forschungabteilung  &40000 &Schwerin &offen\\ 
    \hline
\end{tabular}
\caption{Abgebildet ist die Beispielrelation Projekt mit den Attributen Projektnummer, Projektname, Budget, Ort und Status.}
\label{abb:projekt}
\end{table}
Die Anfrage 
\begin{align}
    \label{anf:bsp}
    \begin{split}
    & \mathbf{SELECT} \; \text{A.Name, P.Projektname}\\
    & \mathbf{FROM} \; \text{Angestellte A} \; \mathbf{JOIN} \; \text{Projekt P} \; \mathbf{ON} \; \text{A.Projektnummer} = \text{P.Projektnummer}\\
    &\mathbf{WHERE} \; \text{P.Status}=\text{'offen'}
\end{split}
\end{align}
liefert die Namen der Angestellten von offenden Projekten. Dementsprechend ist die Ergebnisrelation in Tabelle \ref{abb:result_relation} dargestellt.
\begin{table}[h]
    \centering
\begin{tabular}{|c|c|} \hline
    \multicolumn{2}{|c|}{\textbf{Ergebnis}} \\ \hline
    \hline
    A.Name & P.Projektname \\ 
    \hline
    Martin &Forschungabteilung \\ 
    \hline
    Johann &Forschungabteilung\\
    \hline
    Anna &Forschungabteilung\\ 
    \hline
    Antonia &Verwaltung \\
    \hline
\end{tabular}
\caption[]{Zu sehen ist die Ergebnisrelation der zuvor beschriebenen Anfrage (\ref{anf:bsp}).}
\label{abb:result_relation}
\end{table}
Die Aliasnamen \glqq A\grqq{} und \glqq P\grqq{} dienen hier zur Übersicht und sind bei geschalteten SQL-Anfragen zwingend notwendig. Die klassiche Anfrage (\ref{anf:stand}) kann mit arithmetischen Operationen und Aggregatfunktionen sowie Gruppierungsfunktionen erweitert werden. Die Funktion \textbf{SUM} kann zur Summierung numerischer Attributwerte eines Attributs über mehrere Tupel verwendet werden. Die \textbf{GROUP BY}-Klausel dient zur Gruppierung von Tupeln bezüglich gleicher Attributwertkombinationen. Eine Kombination aus der Aggregatfunktion \textbf{SUM} und der Gruppierung wird beispielsweise in der Anfrage
\begin{align}
    \label{anf:erw}
    \begin{split}
        & \mathbf{SELECT} \; \text{A.Spezialisierung}, \; \mathbf{SUM}(\text{A.Gehalt}) \; \mathbf{AS} \; \text{Angestelltenkosten}\\
        & \mathbf{FROM} \; \text{Angestellte A} \\
        &\mathbf{GROUP} \, \mathbf{BY} \; \text{A.Spezialisierung}
    \end{split}
\end{align}
verwendet. Diese berechnet die Angestelltenkosten und gruppiert sie nach den jeweiligen Spezialisierungen. Die Ergebnistabelle ist in Tabelle \ref{abb:erg_erw} dargestellt.
\begin{table}[h]
    \centering
\begin{tabular}{|c|c|} \hline
    \multicolumn{2}{|c|}{\textbf{Ergebnis}} \\ \hline
    \hline
    A.Spezialisierung &Angestelltenkosten\\ 
    \hline
    Elektrotechnik &2300 \\ 
    \hline
    Informatik &3300\\
    \hline
    Buchhaltung &4000\\ 
    \hline
\end{tabular}
\caption[]{Es ist die Ergebnisrelation der zuvor beschriebenen Anfrage (\ref{anf:erw})
\label{abb:erg_erw} abgebildet.}
\end{table}
Ein Beispiel einer geschalteten Anfrage zur Bestimmung des Angestellten mit dem höchstem Gehalt ist durch
\begin{align}
    \label{anf:schachtel}
    \begin{split}
        & \mathbf{SELECT} \; \text{A.Name}\;  \mathbf{AS} \; \text{Topverdiener},\; \text{A.Gehalt}\\
        & \mathbf{FROM} \; \text{Angestellte A} \\
        &\mathbf{WHERE} \; \text{A.Gehalt}=(\mathbf{SELECT}\; \mathbf{MAX}(\text{Gehalt}) \; \mathbf{FROM}\; \text{Angestellte})
    \end{split}
\end{align}
gegeben. Dabei wird die Aggregatfunktion \textbf{MAX} verwendet. Die Ergebnisrelation beinhaltet in diesem Beispiel ein Tupel mit Martin als Topverdiener mit einem Gehalt von 2300 Euro.

Weitere Operationen der Relationenalgebra wie die Vereinigung (\textbf{UNION}) und die Mengendifferenz (\textbf{EXCEPT}) können ebenfalls in SQL-Anfragen eingebunden werden. Es sei bemerkt, dass Ergebnisse von SQL-Anfrage immer als Multimengen fungieren. Um eine Duplikateleminierung zur Verfügung zu haben, kann die \textbf{SELECT}-Klausel durch den \textbf{DISTINCT}-Operator erweitert werden. Die Repräsentierung der Operatoren der Relationenalgebra durch SQL-Anfragen wird in Tabelle \ref{table:vgl} dargestellt. In vielen Datenbankmanagementsystemen gibt es weitere Funktionalitäten und Operatoren, welche in dieser Arbeit nicht weiter beleuchtet werden. Für die datenbankgestützte Umsetzung linearer Algebra genügt der SQL-Anfragekern wie in Tabelle \ref{table:vgl}.

\begin{table}[h]
    \centering
    \begin{tabular}{|l|l|} \hline
        Relationenalgebra &SQL \\ 
        \hline
        Projektion $\pi$ &\textbf{SELECT} \textbf{DISTINCT} \\ 
        &[\textbf{GROUP BY}] \\
        \hline
        Selektion $\sigma$ &\textbf{WHERE} ohne Schachtelung\\
        \hline
        Verbund $\bowtie$ &\textbf{FROM}, \textbf{WHERE}\\
        &\textbf{FROM} mit \textbf{JOIN} \\
        \hline
        Umbenennung $\beta$ &\textbf{FROM} mit Tupelvariable \\ &\textbf{AS} \\
        \hline
        Differenz $\setminus$ & \textbf{WHERE} mit Schachtelung \\
                  & \textbf{EXCEPT} \\
        \hline
        Vereinigung $\cup$ &\textbf{UNION}  \\
        \hline        
    \end{tabular}
    \caption[tt]{Vergleich des Operatorensystems aus Abschnitt \ref{abs:rela_algebra} mit dem SQL-Anfragekern, vgl.\cite{DBLP:books/daglib/0044627}.}
    \label{table:vgl}
\end{table}
\subsection{Lineare Algebra in SQL}
\label{abs:SQL_linalg}
In diesem Abschnitt werden Darstellungsformen für Vektoren und Matrizen als Relationen vorgestellt. Es werden zwei Strategien erläutert, welche jeweils zur Repräsentierung von dicht und dünn besetzten Matrizen genutzt werden. Weiter werden Ideen zur Umsetzung wichtiger Basisoperationen mit Vektoren und Matrizen in SQL beleuchtet, da diese mathematischen Objekte bei zahlreichen statistischen Analysen eingesetzt werden.
\subsection*{Dicht besetzte Matrizen}
Im Folgenden wird das \textit{Coordinate-Schema}\cite{martendiss} als Schema für die Darstellung dichtbesetzter Matrizen genutzt. Dieses Schema gestaltet sich als einfach und ist daher weit verbreitet \cite{saad1990sparskit}.
Für eine weiterführende Diskussion anderer Darstellungsmöglichkeiten und deren Vor- und Nachteile, sei auf Marten\cite{martendiss} verwiesen. 
Das Coordinate-Schema beinhaltet drei Arrays, welche den Zeilenindex, Spaltenindex und den Matrixeintrag angeben. Für $A \in \RR^{m \times n}$ kann das Tupel $(i,j,A_{i,j})$ als Schlüssel-Wert-Paar interpretiert werden. So ergibt sich auf natürliche Weise die Überführung von Matrizen und Vektoren in relationale Schemata.
\begin{defi}[Coordinate-Schema]
    \label{def:coordinate_sheme}
    Für $x \in \RR^n$ und $A \in \RR^{m \times n}$ ergeben sich die Relationen
    \begin{align*}
        \mathbf{x}( &\underline{i} \; \; \mathrm{int}, \\
        &v \; \; \mathrm{double})
    \end{align*}
    für den Vektor $x$ und
    \begin{align*}
        \mathbf{A}( &\underline{i} \; \; \mathrm{int}, \\
        &\underline{j} \; \;\mathrm{int},\\
        &v \; \; \mathrm{double})
    \end{align*} für die Matrix $A$. Die Indizies $i$ und $j$ sind jeweils als ganzzahlige Datentypen, engl. \textit{integer}, dargestellt. Die Matrixeinträge werden in \textit{double precision}\cite{30711} abgespeichert. Die unterstrichenen Attribute stellen die Schlüsselattribute, vgl. Definition \ref{def:key} dar. Diese Darstellungsform wird Coordinate-Schema genannt und für dichtbesetzte Matrizen im weiteren Verlauf dieser Arbeit genutzt.
\end{defi}

\begin{bsp}
    \label{besipiel:_coordinate_sheme}
    Ist
    \begin{equation*}
        A=\begin{pmatrix}
            1 & 2 \\
            -5 & 2 \\
            0 & 7 \\
        \end{pmatrix}
        \in \RR^{3 \times 2}
    \end{equation*}
    gegeben, so ergibt sich Coordinate Schema wie in Tabelle \ref{coordinate_scheme_table}.

\begin{table}
    \centering
    \begin{tabular}{|c|c|c|} 
        \hline
    \multicolumn{3}{|c|}{\textbf{A}} \\ \hline
     \hline
     $i$ &$j$ &$v$ \\ 
     \hline
     1 &1 &1\\ 
     \hline
     1 &2 &2\\
     \hline
     2 &1 &$-5$\\
     \hline
     2 &2 &2\\
     \hline
     3 &1 &0\\
     \hline
     3 &2 &7\\
     \hline
    \end{tabular}
    \caption[coordinate]{Das Coordinate Schema zur Matrix $A$ aus Beispiel \ref{besipiel:_coordinate_sheme}.}
    \label{coordinate_scheme_table}
\end{table}
\end{bsp}

\subsection*{Basisoperationen}
\label{abs_basisoperationen}
In diesem Abschnitt werden typische Operationen der linearen Algebra beschrieben. Einfache Funktionen wie die Summation und Multiplikation für reelle Zahlen sind bereits im SQL-Standard\cite{DBLP:books/daglib/0067064} enthalten. Seien nun Vektoren $x,y \in \RR^n$ sowie Matrizen $A,B \in \RR^{m \times n}$ und Skalare $r,s \in \RR$ gegeben. Die jeweiligen Relationen werden gemäß dem Coordinate-Schema erstellt. Die SQL-Anweisung für die Vektoraddition $rx+sy$ lautet
\begin{align*}
    & \mathbf{SELECT} \; x.i \; \mathbf{AS} \; i, \; r*x.v+(s*y.v ) \; \mathbf{AS} \; v \\
    & \mathbf{FROM} \; x \; \mathbf{JOIN} \; y \; \mathbf{ON} \; x.i=y.i
\end{align*}
Ähnlich ergibt sich die Matrixaddition $rA+sB$ zu
\begin{align*}
    & \mathbf{SELECT} \; A.i \; \mathbf{AS} \; i, \; A.j \; \mathbf{AS} \; j, (r*A.v)+(s*B.v ) \; \mathbf{AS} \; v \\
    & \mathbf{FROM} \; A \; \mathbf{JOIN} \; B \; \mathbf{ON} \; A.i=B.i \; \mathbf{AND} \; A.j=B.j
\end{align*} 
Mit der Aggregation \textbf{SUM} können zudem Skalarprodukte und damit Längenbegriffe wie Normen formuliert werden. Die entsprechenden SQL-Anfragen sind im Anhang \ref{app:app_1} zu finden. 

Weitere wichtige Oprationenen sind die Matrixvektor- und Matrixmatrixmultiplikation. 
Durch Kombination vorheriger Basisoperationen ergeben sich die entsprechende Transformation für die Matrixvektormultiplikation $Ax \in \RR^m$ einer Matrix $A \in \RR^{m \times n}$ und Vektors $x \in \RR^n$ zu

\begin{align*}
    & \mathbf{SELECT} \; A.i \; \mathbf{AS} \; i, \; \mathbf{SUM} (A.v*x.v) \; \mathbf{AS} \; v\\
    & \mathbf{FROM} \; A \; \mathbf{JOIN} \; x \; \mathbf{ON} \; A.j=x.i \; \\
    & \mathbf{GROUP} \, \mathbf{BY} \; A.i.
\end{align*}
Die Matrix $C=AB \in \RR^{m \times n}$ als Produkt zweier Matrizen $A \in \RR^{m \times k}$ und $B \in \RR^{k \times n}$ lässt sich durch

\begin{align*}
    & \mathbf{SELECT} \; A.i \; \mathbf{AS} \; i, \; B.j \; \mathbf{AS} \; j, \; \mathbf{SUM} (A.v*B.v) \; \mathbf{AS} \; v\\
    & \mathbf{FROM} \; A \; \mathbf{JOIN} \; B \; \mathbf{ON} \; A.j=B.i \; \\
    & \mathbf{GROUP} \, \mathbf{BY} \; A.i, \, B.j
\end{align*}
berechnen.
Schließlich kann auch die Transponierte $A^T$ einer Matrix $A$ einfach berechnet werden, siehe dazu Anhang \ref{app:app_1}.

\subsection*{Dünn besetzte Matrizen}
Bei bestimmten Anwendungen werden dünnbesetzte Matrizen benötigt, deren Zeilen- und Spaltenanzahl erheblich größer als bei dicht besetzten Problemen sind. Oft enthalten diese Matrizen dann eine Menge von Nicht-Null-Elementen, welche im Vergleich zu den Null-Elementen verschwindend gering ist. Daher lohnt es sich, nur die Nicht-Null-Elemente sinnvoll zu speichern. In dieser Arbeit wird das \textit{Compressed-Sparse-Column-Schema}\cite{duff1989sparse} verwendet, um dünn besetzte Matrizen zu repräsentieren. Es besteht ähnlich des Coordinate-Schemas wieder aus drei Arrays. Das erste Array gibt das Spaltenmuster \textbf{colPtr}, das zweite den Zeilenindex \textbf{rowInd} und das dritte den Matrixeintrag \textbf{val} an. Der formale Zusammenhang zwischen einer Matrix $A \in \RR^{m \times n}$ und der Darstellung im Compressed-Sparse-Column-Schema ist durch 
\begin{equation*}
    A_{i,j}=\mathbf{val}[k] \Leftrightarrow (\mathbf{rowInd}[k]=i) \wedge (\mathbf{colPtr}[j] \leq k < \mathbf{colPtr}[j+1])
\end{equation*} 
gegeben. Folgendes Beispiel illustriert diese Beziehung.
\begin{bsp}
    Für die Matrix
    \begin{equation*}
        A=\begin{pmatrix}
            a_{11} &0 &0 &a_{14} \\
            a_{21} &a_{22} &0 &0 \\
            0 &0 &a_{33} &0 \\
            0 &a_{42} &0 &a_{44} \\
        \end{pmatrix}
    \end{equation*}
    ergeben sich die Arrays des Compressed-Sparse-Column-Schemas zu
    \begin{itemize}
        \item \textbf{val}=$[a_{11}, a_{21}, a_{22}, a_{42}, a_{33}, a_{14}, a_{44}]$,
        \item \textbf{rowInd}=$[1, 2, 2, 4, 3, 1, 4]$,
        \item \textbf{colPtr}=$[1, 3, 5, 6, 8]$.
    \end{itemize}
\end{bsp}
Die indirekte Indizierung der Matrixelemente sorgt zwar für eine Verminderung des Speicherbedarfs, erschwert jedoch die rein relationale Umsetzung von Operationen wie die Matrixvektormultiplikation. Hier kann ein objekt-relationaler Ansatz Abhilfe schaffen, bei dem Array-Datentypen genutzt werden, welche jedoch nur in bestimmten relationalen Datenbankmanagementsystemen unterstützt werden. 

\begin{defi}[Spaltenkompression, vgl.\cite{martendiss}]
    \label{def:spaltenkomp}
Eine dünn besetzte Matrix $A \in \RR^{m \times n}$ wird als Relation
\begin{align*}
    \mathbf{A}(&i \; \mathrm{int} \; \mathrm{ARRAY}, \\
               &\underline{j} \; \mathrm{int}, \\
               &v \; \mathrm{double} \; \mathrm{ARRAY}
               )
\end{align*}
gespeichert. Diese Repräsentierung wird Spaltenkompression genannt und ist insbesondere für die Matrixvektormultiplikation nützlich.
\end{defi}
In dem Open-Source-Datenbanksystem PostgreSQL \cite{momjian2001postgresql}, welches in dieser Arbeit genutzt wird, kann hinsichtlich der Matrixvektormultiplikation die \textbf{UNNEST}-Funktion genutzt werden. Diese Funktion wandelt ein Array in eine Menge von Tupeln um. Die Matrixvektormultiplikation $Ax$ mit dünnbesetzter Matrix A kann dann mithilfe der SQL-Anfage

\begin{align*}
    &\mathbf{SELECT} \; i, \mathbf{SUM}(v)\\
    &\mathbf{FROM} (\\
    &\mathbf{SELECT} \; \mathbf{UNNEST}(A.i)\; \mathbf{AS} \; i, \mathbf{UNNEST}(A.v)*x.v \; \mathbf{AS} \; v \\
    &\mathbf{FROM} \; A \; \mathbf{JOIN} \; x \; \mathbf{ON} \; A.j=x.i ) \; \text{temp} \\
    &\mathbf{GROUP} \, \mathbf{BY} \; i
\end{align*}
berechnet werden.

Zusammenfassend stellt sich heraus, dass wesentliche Objekte der linearen Algebra und fundamentale Operationen im SQL-Kern umgesetzt werden können. Für dicht besetzte Matrizen wird das Coordinate-Schema \ref{def:coordinate_sheme} benutzt, während für dünn besetzte Probleme das Spaltenkompression-Schema \ref{def:spaltenkomp} genutzt wird. Beide Darstellungsformen erlauben kompakte SQL-Anfragen für die Matrixvektormultiplikation. Für eine tiefere Analyse dieser und anderer Darstellungsmöglichkeiten und deren Perfomance hinsichtlich verschiedener Basisoperationen sei auf Marten\cite{martendiss} verwiesen.


%Im folgenden Abschnitt werden diese Resultate genutzt und im Zusammenhang mit einem Machine- Learning-Verfahren eingesetzt.



Training, Aufgabe Leistung

supervisies, unsupervised erklären

Klassifikationsproblem

Merkmalsextraktion( 1FFT 2FFT, IFFT NFFT)

(Faltung)

(FFT Regeln insb convolution/coprrelation theorem mit FFT)

Trennbarkeit linear/nichtlinear Entscheidungsgrenzen Hyperebene

Perzeptron Theorem

numerische Minimierung, kurz Abstiegsverfahren in einfacher Version

falls nötig adaptive Verfahren

warum NN?

warum später CNN?


