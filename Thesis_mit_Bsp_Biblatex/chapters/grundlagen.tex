\chapter{Grundlagen}

\section{Mathe/ ML Learning}

\section{Relationale Datenbanksysteme}
Relationale Datenbanksysteme gehören zu den erfolgreichsten und verbreitetsten Datenbanken, welche zur elektronischen Datenverwaltung in Computersystemen eingesetzt werden\cite{DBLP:books/daglib/0044627}. In diesem Abschnitt werden wichtige Grundbegriffe relationaler Datenbanksysteme erläutert und erklärt, wie Daten repräsentiert und verarbeitet werden können. Die Notation und Bezeichnungen basieren auf Heuer et al. \cite{DBLP:books/daglib/0044627}. Zum Abfragen und Manipulieren der Daten wird die Datenbanksparche SQL eingeführt und deren theoretische Grundlage im Abschnitt \ref{abs:SQL_intro} beleuchtet. Schließlich werden im Abschnitt \ref{abs:SQL_linalg} Methoden vorgestellt, um Objekte der linearen Algebra als Relationen darzustellen und damit verbundene Operationen, beispielsweise die Matrixvektormultiplikation, umzusetzen.

\subsection{Das Relationenmodell}
Der Grundbaustein relationaler Datenbanksysteme bildet die Relation. Sie stellt eine mathematische Beschreibung einer Tabelle, welche aus Attributen und zugehörigen Domänen besteht, dar. Ein Datenbanksystem ist dann eine Menge dieser Tabellen. 

\begin{defi}[Universum, Attribut, Domäne]
    \label{def:universum}
    Bezeichne die endliche Menge $\mathcal{U} \neq \emptyset$ das Universum. Ein Element $A \in \mathcal{U}$ heißt Attribut. Für $m \in \mathbb{N}$ sei $\mathcal{D}=\{D_1, \ldots, D_m\}$ eine Menge nichtleere Mengen. Ein Element $D_i \in \mathcal{D}$ wird Domäne genannt. Für eine Funktion $\mathrm{dom}: \mathcal{U} \rightarrow \mathcal{D}$ bezeichne $\mathrm{dom}(A)$ den Wertebereich von $A$ und $w \in \mathrm{dom}(A)$ ein Attributwert.
\end{defi}

Nun kann das Relationenschema und zugehörige Begriffe wie Relation und Tupel definiert werden.

\begin{defi}[Relationenschema, Relation, Tupel]
    \label{def:relation}
    Eine Menge $R \subseteq \mathcal{U}$ heißt Relationenschema über dem Universum $\mathcal{U}$. Für $R=\{A_1, \ldots, A_n \}$ ist eine Relation $r$ über $R$, kurz $r(R)$, als eine endliche Menge von Abbildungen
    \begin{equation*}
        t:R \rightarrow \bigcup_{i=1}^m D_i
    \end{equation*}
    definiert. Dabei gilt $t(A) \in \mathrm{dom}(A)$. Die Abbildungen $t$ werden Tupel genannt.
\end{defi}

Vereinfacht gesagt, setzt sich eine Datenbank als Menge von Relationen und ein Datenbankschema als Menge der zugehörigen Relationenschemata zusammen.

\begin{defi}[Datenbank, Datenbankschema, vgl.\cite{DBLP:books/daglib/0044627}]
    Für $p \in \mathbb{N}$ ist eine Menge von Relationenschemata $S=\{R_1, \ldots, R_p\}$ als Datenbankschema definiert. Eine Datenbank $d$ über dem Schema $S$, kurz $d(S)$, ist eine Menge von Relationen
    \begin{equation*}
        d=\{r_1, \ldots, r_p \}
    \end{equation*}
    mit $r_i(R_i)$ für $1 \leq i \leq p$. Eine Relation $r \in d$ wird Basisrelation genannt.
\end{defi}
\subsection{Die Anfragesprache SQL}

\subsection{Lineare Algebra in SQL}
\label{linalgsql}
In diesem Abschnitt wird eine Darstellungsform von Vektoren und Matrizen als Relationen vorgestellt. Weiter werden Ideen zur Umsetzung wichtiger Basisoperationen mit Vektoren und Matrizen in SQL beleuchtet, da diese mathematischen Objekte bei zahlreichen statistischen Analysen eingestzt werden.
\subsection{Matrixdarstellunng}
Im Folgenden wird das \textit{Coordinate scheme} \cite{martendiss} als Schema für die Darstellung  von Matrizen und Vektoren genutzt. Dieses Schema gestaltet sich als einfach und ist daher weit verbreitet \cite{saad1990sparskit}.
Für eine weiterführende Diskussion anderer Darstellungsmöglichkeiten, sei an dieser Stelle auf Marten \cite{martendiss} verwiesen. 
Das Coordinate Scheme beinhaltet 3 Arrays, welche den Zeilenindex, Spaltenindex und den Matrixeintrag als jeweilge Nicht-Null Werte strukturieren. 
\begin{bsp}
    \label{besipiel:_coordinate_sheme}
    Für $x \in \RR^n$ und $A \in \RR^{n \times m}$ ergeben sich die Relationen
    \begin{align*}
        X( &\underline{i} \; \; \mathrm{int}, \\
        &v \; \; \mathrm{double})
    \end{align*}
    für den Vektor $x$ und
    \begin{align*}
        A( &\underline{i} \; \; \mathrm{int}, \\
        &\underline{j} \; \;\mathrm{int},\\
        &v \; \; \mathrm{double})
    \end{align*} für die Matrix $A$.
    Ist
    \begin{equation*}
        A=\begin{pmatrix}
            1 & 2 \\
            -5 & 2 \\
            0 & 7 \\
        \end{pmatrix}
        \in \RR^{3 \times 2}
    \end{equation*}
    gegeben, so ergibt sich Coordinate Schema wie in Abbildung 2.1.
\end{bsp}

\begin{figure}[h]
    \label{coordinate_scheme_table}
    \centering
    \begin{tabular}{ |c|c|c|c|c|c|c| } 
     \hline
     Zeile $i$ &1 &1 &2 &2 &3 &3 \\ 
     \hline
     Spalte $j$ &1 &2 &1 &2 &1 &2 \\ 
     \hline
     Eintrag $a_{i,j}$ &1 &2 &-5 &2 &0 &7 \\ 
     \hline
    \end{tabular}
    \caption{Das Coordinate Schema zur Matrix $A$ aus Beispiel \ref{besipiel:_coordinate_sheme}.}
\end{figure}



\subsection{Basisoperationen}
\label{abs_basisoperationen}
In diesem Abschnitt werden typische Operationen mit Objekten der linearen Algebra beschrieben. Einfache Operationen wie Summation und Multiplikation für reelle Zahlen sind bereits im SQL-Standard enthalten. Seien nun Vektoren $x,y \in \RR^n$ sowie Matrizen $A,B \in \RR^{m \times n}$ und Skalare $r,s \in \RR$ gegeben. Die SQL-Anweisung für die Vektoraddition $rx+sy$ lautet
\begin{align*}
    & \mathbf{select} \; x.i \; \mathbf{as} \; i, \; r*x.v+(s*y.v ) \; \mathbf{as} \; v \\
    & \mathbf{from} \; x \; \mathbf{join} \; y \; \mathbf{on} \; x.i=y.i
\end{align*}
Ähnlich ergibt sich die Matrixaddition $rA+sB$ zu
\begin{align*}
    & \mathbf{select} \; A.i \; \mathbf{as} \; i, \; A.j \; \mathbf{as} \; j, r*A.v+(s*B.v ) \; \mathbf{as} \; v \\
    & \mathbf{from} \; A \; \mathbf{join} \; B \; \mathbf{on} \; A.i=B.i \; \mathbf{and} \; A.j=B.j
\end{align*}
Das Auftreten von \textbf{NULL}-Werten in den Relationen sei hierbei ausgeschlossen. 
Mit der Aggregation \textbf{SUM} können zudem Skalarprodukte und damit Längenbegriffe wie Normen und dadurch induzierte Abstandsbergiffe formuliert werden. Die entsprechenden SQL-Anfragen  sind im Anhang \ref{app:app_1} zu finden. \\
Weitere wichtige Oprationenen stellen die Matrixvektor- und Matrixmatrixmultiplikation dar. 
Durch Kombination vorheriger Basisoperationen ergeben sich entsprechende Transformationenen für die Matrixvektormultiplikation $Ax \in \RR^m$ einer Matrix $A \in \RR^{m \times n}$ und Vektors $x \in \RR^n$ zu

\begin{align*}
    & \mathbf{select} \; A.i \; \mathbf{as} \; i, \; \mathbf{sum} (A.v*x.v) \; \mathbf{as} \; v\\
    & \mathbf{from} \; A \; \mathbf{join} \; x \; \mathbf{on} \; A.j=x.i \; \\
    & \mathbf{group} \; \mathbf{by} \; A.i
\end{align*}

Für die Matrix $C=AB \in \RR^{m \times n}$ als Produkt zweier Matrizen $A \in \RR^{m \times k}$ und $B \in \RR^{k \times n}$ lautet die Anfrage

\begin{align*}
    & \mathbf{select} \; A.i \; \mathbf{as} \; i, \; B.j \; \mathbf{as} \; j, \; \mathbf{sum} (A.v*B.v) \; \mathbf{as} \; v\\
    & \mathbf{from} \; A \; \mathbf{join} \; B \; \mathbf{on} \; A.j=B.i \; \\
    & \mathbf{group} \; \mathbf{by} \; A.i, \, B.j
\end{align*}
Schließlich kann auch die Transponierte $A^T$ einer Matrix $A$ einfach berechnet werden, siehe dazu Anhang \ref{app:app_1}.

Zusammenfassend stellt sich heraus, dass wesentliche Objekte der linearen Algebra und damit verbundene fundamentale Operationen im SQL-Kern umgesetzt werden können.  
Im folgenden Abschnitt werden diese Resultate genutzt und im Zusammenhang mit einem Machine- Learning-Verfahren eingesetzt.
Problemstellung(Einleitung)

\begin{defi}
    \label{def:image}
    Eine Matrix $X \in [0,1]^{h \times b}$ heißt (Grauwert)-Bild mit der Höhe $h$ und Breite $b$. Mit $X_{i,j}$ wird der Grauwert des Pixels $p=(i,j)$ bezeichnet.
\end{defi}


Training, Aufgabe Leistung

supervisies, unsupervised erklären

Klassifikationsproblem

Merkmalsextraktion( 1FFT 2FFT, IFFT NFFT)

(Faltung)

(FFT Regeln insb convolution/coprrelation theorem mit FFT)

Trennbarkeit linear/nichtlinear Entscheidungsgrenzen Hyperebene

Perzeptron Theorem

numerische Minimierung, kurz Abstiegsverfahren in einfacher Version

falls nötig adaptive Verfahren

warum NN?

warum später CNN?


\section*{SQL}
Relationen, Tensoren

Marizen/Vektoren als Relationen

Basisoperationen

