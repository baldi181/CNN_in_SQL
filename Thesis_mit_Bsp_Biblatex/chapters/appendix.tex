\chapter{Anhang}

% \tocless
\section{Weitere Basisoperation in SQL}
\label{app:app_1}
Die euklidische Norm 
\begin{equation*}
    ||x||_{2}:=\left(\sum_{i=1}^n x_i^2\right)^{\frac{1}{2}}
\end{equation*}
eines Vektors $x \in \RR^n$ und die Frobeniusnorm 
\begin{equation*}
    ||A||_F:=\left(\sum_{i=1}^m \sum_{j=1}^n a_{ij}^2\right)^{\frac{1}{2}} 
\end{equation*}
einer Matrix $A \in \RR^{m \times n}$ sind als SQL-Anfrage durch
\begin{align*}
    & \mathbf{SELECT} \; \mathbf{sqrt}(\mathbf{SUM}(v*v)) \; \mathbf{AS} \; 2Norm \\
    & \mathbf{FROM} \; x
\end{align*}
beziehungsweise
\begin{align*}
    & \mathbf{SELECT} \; \mathbf{sqrt}(\mathbf{SUM}(v*v)) \; \mathbf{AS} \; FNorm \\
    & \mathbf{FROM} \; A
\end{align*}
gegeben.
%skalaprodukt
Im euklidischen Vektorraum $\RR^n$ ist das Skalarprodukt $\langle x,y \rangle$ zweier Vektoren $x,y \in \RR^n$ definiert als 
\begin{equation*}
    \langle x,y \rangle :=\sum_{i=1}^n x_i y_i.
\end{equation*}
Eine entsprechende Transformation in SQL mit dem Aggregationsoperator \textbf{SUM} lautet
\begin{align*}
    & \mathbf{SELECT} \; \mathbf{SUM}(x.v*y.v) \; \mathbf{AS} \; v \\
    & \mathbf{FROM} \; x \; \mathbf{JOIN} \; y \; \mathbf{ON} \; x.i=y.i
\end{align*}
Die Transponierte Matrix $A^T \in \RR^{n \times m}$ einer Matrix $A \in \RR^{m \times n}$ kann durch die Anfrage 
\begin{align*}
    & \mathbf{SELECT} \; A.j \; \mathbf{AS} \; i, \; A.i \; \mathbf{AS} \; j, \; A.v\\
    & \mathbf{FROM} \; A
\end{align*}
berechnet werden.
% \tocless
%\section{Überschrift A2}
%\label{sec:app_2}
%blablabla