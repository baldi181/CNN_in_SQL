\chapter{Anhang}

% \tocless
\section{Weitere Basisoperation in SQL}
\label{app:app_1}
Die euklidische Norm 
\begin{equation*}
    ||x||_{2}:=\left(\sum_{i=1}^n x_i^2\right)^{\frac{1}{2}}
\end{equation*}
eines Vektors $x \in \RR^n$ und die Frobeniusnorm 
\begin{equation*}
    ||A||_F:=\left(\sum_{i=1}^m \sum_{j=1}^n a_{ij}^2\right)^{\frac{1}{2}} 
\end{equation*}
einer Matrix $A \in \RR^{m \times n}$ sind als SQL-Anfrage durch
\begin{align*}
    & \mathbf{SELECT} \; \mathbf{sqrt}(\mathbf{SUM}(v*v)) \; \mathbf{AS} \; 2Norm \\
    & \mathbf{FROM} \; x
\end{align*}
beziehungsweise
\begin{align*}
    & \mathbf{SELECT} \; \mathbf{sqrt}(\mathbf{SUM}(v*v)) \; \mathbf{AS} \; FNorm \\
    & \mathbf{FROM} \; A
\end{align*}
gegeben.
%skalaprodukt
Im euklidischen Vektorraum $\RR^n$ ist das Skalarprodukt $\langle x,y \rangle$ zweier Vektoren $x,y \in \RR^n$ definiert als 
\begin{equation*}
    \langle x,y \rangle :=\sum_{i=1}^n x_i y_i.
\end{equation*}
Eine entsprechende Transformation in SQL mit dem Aggregationsoperator \textbf{SUM} lautet
\begin{align*}
    & \mathbf{SELECT} \; \mathbf{SUM}(x.v*y.v) \; \mathbf{AS} \; v \\
    & \mathbf{FROM} \; x \; \mathbf{JOIN} \; y \; \mathbf{ON} \; x.i=y.i
\end{align*}
Die Transponierte Matrix $A^T \in \RR^{n \times m}$ einer Matrix $A \in \RR^{m \times n}$ kann durch die Anfrage 
\begin{align*}
    & \mathbf{SELECT} \; A.j \; \mathbf{AS} \; i, \; A.i \; \mathbf{AS} \; j, \; A.v\\
    & \mathbf{FROM} \; A
\end{align*}
berechnet werden.

Sei $A \in \mathbb{C}^{m \times n}$ eine Matrix im erweitertem Coordinate-Schema. Die adjungierte Matrix $A^* \in \mathbb{C}^{n \times m}$ kann durch die SQL-Anfrage
\begin{align*}
    & \mathbf{SELECT} \; A.j \; \mathbf{AS} \; i, \; A.i \; \mathbf{AS} \; j, \; A.re \;\mathbf{AS} \; re, \; -(A.im) \; \mathbf{AS} \; im  \\
    & \mathbf{FROM} \; A
\end{align*}
berechnet werden.
% \tocless
\section{MATLAB-Implementierungen}
\label{app:app_2}

In diesem Abschnitt werden zwei MATLAB-Implementierungen gesammelt. Die Implementierungen \ref{mat:code:zyk} und \ref{mat:code:zyk2} können zur Konstruktion der zyklischen Blockmatrizen aus Abschnitt \ref{abs:conv_in_sql} genutzt werden. Der Code wurde von Royi Avital im Zuge der digitalen Bildverarbeitung entwickelt. Mehr Informationenen sind im ausführlichen GitHub Repository https://github.com/RoyiAvital/StackExchangeCodes (zuletzt zugegriffen am 17.08.2022) zu finden.
\lstinputlisting[label=mat:code:zyk, caption=Code 1 zur Konstruktion zyklischer Blockmatrizen, language=Matlab]{code/matlab files/Create2DKernelConvMtxSparse.m}
\lstinputlisting[label=mat:code:zyk2, caption=Code 2 zur Konstruktion zyklischer Blockmatrizen, language=Matlab]{code/matlab files/CreateConvMtxSparse.m}

%\label{matlab:boundary}
Die Implementierungen \ref{matlab:boundary} und \ref{dft_ff} dienen hinsichtlich der $\mathrm{2}DFT$ zur Einbettung der Kerne in entsprechende $n \times n$- Matrizen sowie das Extrahieren dieser im Coordinate-Schema.
\lstinputlisting[label=matlab:boundary, caption=Code zur Einbettung der Kerne in $n \times n$-Matrizen, language=Matlab]{code/matlab files/CircularExtension2D.m}

\lstinputlisting[label=dft_ff, caption=Code zum Konstruieren und zum Extrahieren des Coordinate-Schemas, language=Matlab]{code/matlab files/write_twiddle_data.m}
%blablabla