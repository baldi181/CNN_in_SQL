%\begin{acronym}[SEPSEP]
 %\acro{cm}[$\mathcal{C}$]{Confidence Matrix}
%\end{acronym}
\begin{tabular}{cp{1\textwidth}}
    $f \ast g$ & Faltung der Funktionen $f$ und $g$\\
    $f \circledast g$ & zyklische Faltung der Funktionen $f$ und $g$ \\
    $L^1(\RR^n)$ & Raum der Lebesgue-integrierbaren Funktionen auf $\RR^n$ \\
    $A_{i,:}$ & $i$-te Zeile einer Matrix $A$ \\
    $A_{:,j}$ & $j$-te Spalte einer Matrix $A$ \\
    $A_{i,j}$ & Eintrag der Matrix an der $i$-ten Zeile und $j$-ten Spalte \\
    ${}_1 A$ & Anzahl der Zeilen einer Matrix $A$ \\
    ${}_2 A$ & Anzahl der Spalten einer Matrix $A$ \\
    $ \theta \in \RR$ & reeller Schwellwert für ein Perzeptron\\
    $ w \in \RR^n$ & Stellungsvektor beim Perzeptron \\
    $ b \in \RR^n$ &Biasvektor eines neuronalen Netzes\\
    $W \in \RR^{m \times n}$ &Gewichtsmatrix eines neuronalen Netzes \\
    $\mathcal{W}$ & Modellparameter \\
    $\mathcal{H}$ & Hyperparameter \\
    $\mathcal{M}$ & Datenmenge \\
    $\mathcal{C}$ & Menge der Klassenlabel \\
    $\mathcal{T}$ & Trainingsmenge \\
    $\mathcal{T}'$ & Testmenge \\
    $K \in \RR^{k \times k}$ & Kern eines gefalteten neuronalen Netzes \\
    $K^{rot180} \in \RR^{k \times k}$ & ein um 180 Grad gedrehter Kern $K \in \RR^{k \times k}$ \\
    $X$ & Eingabebild bzw. Karte bei FNN/CNN \\
    $X \ast K$ & Matrixfaltung \\
    $I$ & Einheitsmatrix mit jeweils passender Dimension \\
    $\mathbf{1}$ &Einsvektor bzw. Einsmatrix mit jeweils passender Dimension \\
    $A^*$ & Adjungierte Matrix der Matrix $A \in \mathbb{C}^{m \times n}$\\ 
    $u \otimes v \in \RR^{m \times n}$ & dyadisches Produkt zweier Vektoren $u \in \RR^m$ und $v \in \RR^n$ \\
    $A \odot B \in \RR^{m \times n}$ & Matrix, welche sich aus der elementweisen Multiplikation \\ & der Einträge in den Matrizen $A \in \RR^{m \times n}$ und $B \in \RR^{m \times n}$ ergibt \\
    $\mathbf{val}[k]$ & Arrayzugriff auf das Array $\mathbf{val}$ an $k$-ter Stelle \\
    $[n]$ & Menge $\{1, \ldots, n\}$ \\
    $S_n$ & symmetrische Gruppe, die aus allen Permutationen \\ &einer $n$-elementigen Menge besteht 
\end{tabular}