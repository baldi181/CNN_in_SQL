\documentclass[12pt,DIV=15,BCOR=15mm,twoside,headsepline,abstract=true,listof=totoc,bibliography=totoc]{scrreprt}

%%%%%%%%%%%%%%%%%%%%%%%%%%%%%%%%%%%%%%%%%%%%%%%%%%%%%%%%%%%
%%%%%%%%%%%%% Uni Rostock Thesis Style %%%%%%%%%%%%%%%%%%%%
%%% bei Problemen, Mail an susann.dittmer@uni-rostock.de
%%%%%%%%%%%%%%%%%%%%%%%%%%%%%%%%%%%%%%%%%%%%%%%%%%%%%%%%%%%
% enthält schon viele wichtige Pakete
\usepackage[mnf]{thesis_uro} %Fakultät wählen: uni (Standard),inf,msf,ief,mnf,mef,juf,wsf,auf,thf,phf

\addbibresource{literatur.bib} %Bibliographiedateien laden

%Definition von Umgebungen
\newtheorem{kor}{Korollar}
\newtheorem{hsatz}{Hilfssatz}
\newtheorem{satz}{Satz}
\newtheorem{prop}{Proposition}
\newtheorem{defi}{Definition}
\newtheorem{lem}{Lemma}
\newtheorem{annahme}{Annahme}
\newtheorem{problem}{Problem}
\theoremstyle{remark}	%Styleänderung (Text aufrecht, ...)
\newtheorem{bem}{Bemerkung}
\newtheorem{bsp}{Beispiel}

%Beispiel für eigene Kommandos
\newcommand{\ol}{\overline} %Kurzform für \overline definiert 
%\newcommand{\RR}{\mathbb{R}}
%newcommand{\RRn}{\mathbb{R}^n}

\newcommand{\RR}{\ensuremath{\mathbb{R}}}
\newcommand{\Rnv}{\ensuremath{\mathbb{R}^{n}}}
\newcommand{\Rnn}{\ensuremath{\mathbb{R}^{n\times n}}}

\newcommand{\NN}{\ensuremath{\mathbb{N}}}

\newcommand{\mK}{\ensuremath{\mathcal{K}}}
\newcommand{\mKp}{\ensuremath{\mathcal{K}}^+}
\newcommand{\Code}{\ensuremath{C^{\textnormal{dgl}}}}

% Daten für die Titelseite
\institut{Institut für Mathematik} %auskommentieren, wenn nicht benötigt
\arbeit{Masterarbeit} %Bachelorarbeit, Masterarbeit oder Abschlussarbeit (wenn Staatsexamensarbeit geschrieben wird, kann man, z. B. bei Untertitel "Wissenschaftliche Abschlussarbeit\\ im Rahmen  des Ersten Staatsexamens" eintragen)
\autor{Vorname Nachname}
\betreuerGutachter{Name des Betreuers und ersten Gutachters\newline Universität Rostock\newline Fakultät} %hier \newline als Zeilenwechsel, da Tabelle mit \parbox im Hintergrund
\gutachter{Name des zweiten Gutachters\newline Universität Musterstadt\newline Fakultät}				%dto.
\date{12.08.2021}
\matrNr{123\,45678}
\titel{Titel der Arbeit,\\ bei Bedarf auch zweizeilig}
\untertitel{Untertitel der Arbeit, auch mehrzeilig oder ganz weglassen.} %auskommentieren, wenn nicht benötigt

\begin{document}
\hypersetup{pageanchor=false}
\begin{titlepage}
\mytitle   %hier werden Daten für die Titelseite gesetzt
\end{titlepage}

\zusammenfassung{
  Platz für eine kurze Zusammenfassung.\\
}
\pagenumbering{Roman}
\tableofcontents % Inhaltsverzeichnis
\listoffigures
\listoftables
\addcontentsline{toc}{chapter}{Algorithmenverzeichnis}
\listofalgorithms
\lstlistoflistings

% Abkürzungsverzeichnis --------ggfs auskommentieren
\chapter*{Abkürzungsverzeichnis}
\addcontentsline{toc}{chapter}{Abkürzungsverzeichnis}
\begin{acronym}[SEPSEP]
 \acro{rnn}[RNN]{Rekurrentes Neuronales Netz}
\end{acronym} 


% Symbolverzeichnis ------------ggfs auskommentieren
\chapter*{Symbolverzeichnis}
\addcontentsline{toc}{chapter}{Symbolverzeichnis}
\begin{acronym}[SEPSEP]
 \acro{cm}[$\mathcal{C}$]{Confidence Matrix}
\end{acronym}
\cleardoublepage
\hypersetup{pageanchor=true}
\pagenumbering{arabic}

%mainmatter
\chapter{Einleitung}
Es mag euch wundern, dass die Einleitung in einem separaten File abgelegt ist. Dies muss natürlich nicht so sein. Es könnte aber bei einer langen Abschlussarbeit durchaus die Übersichtlichkeit erhöhen, wenn ihr für verschiedene Kapitel einzelne Dateien anlegt und diese mittels 
\begin{verbatim}\input{<DateiName>}\end{verbatim}

\bigskip
oder

\bigskip
\begin{verbatim}\include{<DateiName>}\end{verbatim}
einfügt.

\bigskip
\noindent\textcolor{red}{Verwendet keine Umlaute oder Leerzeichen in Dateinamen.}

\noindent\texttt{input} fügt den Text direkt an die Stelle des \texttt{input}-Befehls ein.

\noindent\texttt{include} fügt den Text auf einer neuen Seite ein. 

\chapter{Grundlagen}

\section{Mathe/ ML Learning}

\section{Relationale Datenbanksysteme}
Relationale Datenbanksysteme gehören zu den erfolgreichsten und verbreitetsten Datenbanken, welche zur elektronischen Datenverwaltung in Computersystemen eingesetzt werden\cite{DBLP:books/daglib/0044627}. In diesem Abschnitt werden wichtige Grundbegriffe relationaler Datenbanksysteme erläutert und erklärt, wie Daten repräsentiert und verarbeitet werden können. Die Notation und Bezeichnungen basieren auf Heuer et al. \cite{DBLP:books/daglib/0044627}. Zum Abfragen und Manipulieren der Daten wird die Datenbanksparche SQL eingeführt und deren theoretische Grundlage im Abschnitt \ref{abs:SQL_intro} beleuchtet. Schließlich werden im Abschnitt \ref{abs:SQL_linalg} Methoden vorgestellt, um Objekte der linearen Algebra als Relationen darzustellen und damit verbundene Operationen, beispielsweise die Matrixvektormultiplikation, umzusetzen.

\subsection{Das Relationenmodell}
Der Grundbaustein relationaler Datenbanksysteme bildet die Relation. Sie stellt eine mathematische Beschreibung einer Tabelle, welche aus Attributen und zugehörigen Domänen besteht, dar. Ein Datenbanksystem ist dann eine Menge dieser Tabellen. 

\begin{defi}[Universum, Attribut, Domäne]
    \label{def:universum}
    Bezeichne die endliche Menge $\mathcal{U} \neq \emptyset$ das Universum. Ein Element $A \in \mathcal{U}$ heißt Attribut. Für $m \in \mathbb{N}$ sei $\mathcal{D}=\{D_1, \ldots, D_m\}$ eine Menge nichtleere Mengen. Ein Element $D_i \in \mathcal{D}$ wird Domäne genannt. Für eine Funktion $\mathrm{dom}: \mathcal{U} \rightarrow \mathcal{D}$ bezeichne $\mathrm{dom}(A)$ den Wertebereich von $A$ und $w \in \mathrm{dom}(A)$ ein Attributwert.
\end{defi}

Nun kann das Relationenschema und zugehörige Begriffe wie Relation und Tupel definiert werden.

\begin{defi}[Relationenschema, Relation, Tupel]
    \label{def:relation}
    Eine Menge $R \subseteq \mathcal{U}$ heißt Relationenschema über dem Universum $\mathcal{U}$. Für $R=\{A_1, \ldots, A_n \}$ ist eine Relation $r$ über $R$, kurz $r(R)$, als eine endliche Menge von Abbildungen
    \begin{equation*}
        t:R \rightarrow \bigcup_{i=1}^m D_i
    \end{equation*}
    definiert. Dabei gilt $t(A) \in \mathrm{dom}(A)$. Die Abbildungen $t$ werden Tupel genannt.
\end{defi}

Vereinfacht gesagt, setzt sich eine Datenbank als Menge von Relationen und ein Datenbankschema als Menge der zugehörigen Relationenschemata zusammen.

\begin{defi}[Datenbank, Datenbankschema, vgl.\cite{DBLP:books/daglib/0044627}]
    Für $p \in \mathbb{N}$ ist eine Menge von Relationenschemata $S=\{R_1, \ldots, R_p\}$ als Datenbankschema definiert. Eine Datenbank $d$ über dem Schema $S$, kurz $d(S)$, ist eine Menge von Relationen
    \begin{equation*}
        d=\{r_1, \ldots, r_p \}
    \end{equation*}
    mit $r_i(R_i)$ für $1 \leq i \leq p$. Eine Relation $r \in d$ wird Basisrelation genannt.
\end{defi}
\subsection{Die Anfragesprache SQL}

\subsection{Lineare Algebra in SQL}
\label{linalgsql}
In diesem Abschnitt wird eine Darstellungsform von Vektoren und Matrizen als Relationen vorgestellt. Weiter werden Ideen zur Umsetzung wichtiger Basisoperationen mit Vektoren und Matrizen in SQL beleuchtet, da diese mathematischen Objekte bei zahlreichen statistischen Analysen eingestzt werden.
\subsection{Matrixdarstellunng}
Im Folgenden wird das \textit{Coordinate scheme} \cite{martendiss} als Schema für die Darstellung  von Matrizen und Vektoren genutzt. Dieses Schema gestaltet sich als einfach und ist daher weit verbreitet \cite{saad1990sparskit}.
Für eine weiterführende Diskussion anderer Darstellungsmöglichkeiten, sei an dieser Stelle auf Marten \cite{martendiss} verwiesen. 
Das Coordinate Scheme beinhaltet 3 Arrays, welche den Zeilenindex, Spaltenindex und den Matrixeintrag als jeweilge Nicht-Null Werte strukturieren. 
\begin{bsp}
    \label{besipiel:_coordinate_sheme}
    Für $x \in \RR^n$ und $A \in \RR^{n \times m}$ ergeben sich die Relationen
    \begin{align*}
        X( &\underline{i} \; \; \mathrm{int}, \\
        &v \; \; \mathrm{double})
    \end{align*}
    für den Vektor $x$ und
    \begin{align*}
        A( &\underline{i} \; \; \mathrm{int}, \\
        &\underline{j} \; \;\mathrm{int},\\
        &v \; \; \mathrm{double})
    \end{align*} für die Matrix $A$.
    Ist
    \begin{equation*}
        A=\begin{pmatrix}
            1 & 2 \\
            -5 & 2 \\
            0 & 7 \\
        \end{pmatrix}
        \in \RR^{3 \times 2}
    \end{equation*}
    gegeben, so ergibt sich Coordinate Schema wie in Abbildung 2.1.
\end{bsp}

\begin{figure}[h]
    \label{coordinate_scheme_table}
    \centering
    \begin{tabular}{ |c|c|c|c|c|c|c| } 
     \hline
     Zeile $i$ &1 &1 &2 &2 &3 &3 \\ 
     \hline
     Spalte $j$ &1 &2 &1 &2 &1 &2 \\ 
     \hline
     Eintrag $a_{i,j}$ &1 &2 &-5 &2 &0 &7 \\ 
     \hline
    \end{tabular}
    \caption{Das Coordinate Schema zur Matrix $A$ aus Beispiel \ref{besipiel:_coordinate_sheme}.}
\end{figure}



\subsection{Basisoperationen}
\label{abs_basisoperationen}
In diesem Abschnitt werden typische Operationen mit Objekten der linearen Algebra beschrieben. Einfache Operationen wie Summation und Multiplikation für reelle Zahlen sind bereits im SQL-Standard enthalten. Seien nun Vektoren $x,y \in \RR^n$ sowie Matrizen $A,B \in \RR^{m \times n}$ und Skalare $r,s \in \RR$ gegeben. Die SQL-Anweisung für die Vektoraddition $rx+sy$ lautet
\begin{align*}
    & \mathbf{select} \; x.i \; \mathbf{as} \; i, \; r*x.v+(s*y.v ) \; \mathbf{as} \; v \\
    & \mathbf{from} \; x \; \mathbf{join} \; y \; \mathbf{on} \; x.i=y.i
\end{align*}
Ähnlich ergibt sich die Matrixaddition $rA+sB$ zu
\begin{align*}
    & \mathbf{select} \; A.i \; \mathbf{as} \; i, \; A.j \; \mathbf{as} \; j, r*A.v+(s*B.v ) \; \mathbf{as} \; v \\
    & \mathbf{from} \; A \; \mathbf{join} \; B \; \mathbf{on} \; A.i=B.i \; \mathbf{and} \; A.j=B.j
\end{align*}
Das Auftreten von \textbf{NULL}-Werten in den Relationen sei hierbei ausgeschlossen. 
Mit der Aggregation \textbf{SUM} können zudem Skalarprodukte und damit Längenbegriffe wie Normen und dadurch induzierte Abstandsbergiffe formuliert werden. Die entsprechenden SQL-Anfragen  sind im Anhang \ref{app:app_1} zu finden. \\
Weitere wichtige Oprationenen stellen die Matrixvektor- und Matrixmatrixmultiplikation dar. 
Durch Kombination vorheriger Basisoperationen ergeben sich entsprechende Transformationenen für die Matrixvektormultiplikation $Ax \in \RR^m$ einer Matrix $A \in \RR^{m \times n}$ und Vektors $x \in \RR^n$ zu

\begin{align*}
    & \mathbf{select} \; A.i \; \mathbf{as} \; i, \; \mathbf{sum} (A.v*x.v) \; \mathbf{as} \; v\\
    & \mathbf{from} \; A \; \mathbf{join} \; x \; \mathbf{on} \; A.j=x.i \; \\
    & \mathbf{group} \; \mathbf{by} \; A.i
\end{align*}

Für die Matrix $C=AB \in \RR^{m \times n}$ als Produkt zweier Matrizen $A \in \RR^{m \times k}$ und $B \in \RR^{k \times n}$ lautet die Anfrage

\begin{align*}
    & \mathbf{select} \; A.i \; \mathbf{as} \; i, \; B.j \; \mathbf{as} \; j, \; \mathbf{sum} (A.v*B.v) \; \mathbf{as} \; v\\
    & \mathbf{from} \; A \; \mathbf{join} \; B \; \mathbf{on} \; A.j=B.i \; \\
    & \mathbf{group} \; \mathbf{by} \; A.i, \, B.j
\end{align*}
Schließlich kann auch die Transponierte $A^T$ einer Matrix $A$ einfach berechnet werden, siehe dazu Anhang \ref{app:app_1}.

Zusammenfassend stellt sich heraus, dass wesentliche Objekte der linearen Algebra und damit verbundene fundamentale Operationen im SQL-Kern umgesetzt werden können.  
Im folgenden Abschnitt werden diese Resultate genutzt und im Zusammenhang mit einem Machine- Learning-Verfahren eingesetzt.
Problemstellung(Einleitung)

\begin{defi}
    \label{def:image}
    Eine Matrix $X \in [0,1]^{h \times b}$ heißt (Grauwert)-Bild mit der Höhe $h$ und Breite $b$. Mit $X_{i,j}$ wird der Grauwert des Pixels $p=(i,j)$ bezeichnet.
\end{defi}


Training, Aufgabe Leistung

supervisies, unsupervised erklären

Klassifikationsproblem

Merkmalsextraktion( 1FFT 2FFT, IFFT NFFT)

(Faltung)

(FFT Regeln insb convolution/coprrelation theorem mit FFT)

Trennbarkeit linear/nichtlinear Entscheidungsgrenzen Hyperebene

Perzeptron Theorem

numerische Minimierung, kurz Abstiegsverfahren in einfacher Version

falls nötig adaptive Verfahren

warum NN?

warum später CNN?


\section*{SQL}
Relationen, Tensoren

Marizen/Vektoren als Relationen

Basisoperationen


\chapter{Grundlagen neuronaler Netze}
\label{kapitel_neuralnetworks}

In diesem Abschnitt werden Künstliche Neuronale Netze\cite{dayhoff1990neural}, kurz KNN, als Forschungsgegenstand der Informatik eingeführt und deren mathematischen Grundlagen präzisiert. 
Sie stellen informationsverarbeitende Systeme nach dem Vorbild von tierischen beziehungsweise menschlichen Gehirnen dar und bestehen aus Neuronen in gewissen Zuständen und Schichten, die über gewichtete Verbindungen miteinander gekoppelt sind. Jene Gewichte sind als freie Parameter des neuronalen Netzes zu verstehen und können während des Trainingsprozesses so angepasst werden, um eine entsprechende Aufgabe zu lösen.  
Gelingt dies, so können neuronale Netze genutzt werden, um bestimmte Muster in Daten, typischerweise in Bildern, Audio oder Stromdaten, effizient zu erkennen\cite{pandya1995pattern, pao1989adaptive, urbaniak2021quality}.
Sie eignen sich daher für viele typischen Aufgaben des maschinellen Lernens, beispielsweise für die Klassifikation digitalisierter Objekte.

Im ersten Abschnitt wird das Perzeptron\cite{rosenblatt1958perceptron} als Grundeinheit eines neuronalen Netztes eingeführt. 
Im folgendem Abschnitt wird das Konzept der Multi-Layer-Perzeptronen\cite{werbos1988generalization} durch die Kopplung mehrerer Perzeptronen mit bestimmten Übertragungs- und Aktivierungsfunktion in einem Netz erläutert. Diese Repräsentierung eines KNN wird im weiteren Verlauf dieser Arbeit genutzt. 

\section{Das Perzeptron}
\label{perzeptron_abs}
Zunächst wird das \textit{Perzeptron} ähnlich wie in Minsky \cite{minsky2017perceptrons} als fundamentaler Baustein eines neuronalen Netzes eingeführt. Das Perzeptron wird oft als Basis moderner KNN angeführt und kann mithilfe des Perzeptron-Lernalgorithmus\cite{rosenblatt1958perceptron} trainiert werden, um das Problem der linearen Trennbarkeit \ref{prob:lin_trenn}von Punktmengen zu lösen.
\begin{defi}[Perzeptron]
    \label{def_neuron}
    Für eine gegebene Funktion $\psi: \RR \rightarrow \RR$, einen Vektor $w \in \Rnv$ und ein Skalar $\theta \in \RR$ wird die Funktion 
    \[ \
    \Psi: \RR^n \rightarrow \RR, \; \; \; x \mapsto \psi(w^T x +\theta)=:y,
    \]
    \textit{Perzeptron} genannt. Mit $x \in \Rnv$ wird die vektorwertige Eingabe und mit $y \in \RR$ die skalare Ausgabe des Perzeptrons bezeichnet. Dabei ist mit $w^Tx=\sum_{i=1}^n w_i x_i$ das Standardskalarprodukt im euklidischen Vektorraum $\Rnv$ gemeint. Die Komponenten von $w$ werden Gewichte und der Skalar $\theta$ Schwellwert oder auch Bias genannt.
\end{defi}
Die Funktionsweise eines Perzeptrons ist in Abbildung \ref{funktionsweise_neuron} dargestellt.
\begin{figure}[h]
    \includegraphics[width=0.8\textwidth]{pics/chapter_neuralnetworks/perzeptron.png}
    \centering
    \caption{Arbeitsweise eines Perzeptrons mit entsprechender Notation aus Definition \ref{def_neuron}.}
    \label{funktionsweise_neuron}
\end{figure}
Bei der Wahl der Funktion $\psi$ gibt es mehrere Möglichkeiten. Wird wie in Minsky\cite{minsky2017perceptrons} die Heavyside-Funktion
\begin{equation*}
    \psi: \RR \rightarrow \RR, \; \; \;
    \psi(x)=\begin{cases}
       1 &, x \geq 0 \\
       0 &, \text{sonst}
    \end{cases}
\end{equation*} 
genutzt, kann das Perzeptron als binärer Klassifikator wie in \ref{abs_linear_trenn} interpretiert werden. Dabei dient $w^Tx+\theta=0$ als trennende Hyperebene. Ist $w^Tx+\theta<0$, so ist $\psi(x)=0$ und $x$ wird der Klasse $K_{-1}$ zugeordnet. Gilt jedoch $w^Tx+\theta \geq 0$ und damit $\psi(x)=1$, so ist der Vektor $x$ der Klasse $K_1$ zugehörig. 

Für ein Klassifikationsproblem, bei dem die Klassen nicht linear trennbar sind, scheitern diese einfachen Perzeptronen. Hier wird oft das zweidimensionale XOR-Problem angeführt, bei denen die Punktmengen $P_{-1}=\{(0,0),(1,1)\}$ und $P_{1}=\{(1,0),(0,1)\}$ getrennt werden sollen. Um solche Aufgaben zu lösen, ist es notwendig, mehrere Perzeptronen geschickt zu verknüpfen, um komplexe Entscheidungsgrenzen zu erhalten.

\section{Multi-Layer-Perzeptron}
\label{MLP_abs}
In dieser Arbeit wird ein Künstliches Neuronales Netz als eine Menge von Perzeptronen, die in gewissen Schichten partitioniert und miteinander verbunden sind, notiert. Diese sogenannten \textit{Multi-Layer-Perzeptronen}, kurz MLP,  gelten als erste tiefe neuronale Netze und sind seit den späten 1980er Gegenstand der Forschung\cite{bourlard1990links,bounds1988multilayer,MLPbook}. Zunächst sind einige Definition notwendig, um eine lesbare Notation des MLPs zu geben.

\begin{defi}[Übertragungsfunktion]
    \label{def_net}
    Für eine gegebene Matrix $W \in \RR^{n \times m}$ und einen Vektor $b \in \RR^m$ ist 
    \[ 
    \Psi^{W,b}: \RR^n \rightarrow \RR^m, \; \; \; x \mapsto W^T x +b
    \]
    als Übertragungsfunktion definiert. Der Vektor $y=\Psi^{W,b}(x) \in \RR^m $ wird als Netzeingabe bezeichnet.
\end{defi}
Hierbei ist $W$ eine Gewichtsmatrix und $b$ ein Biasvektor, welche als freie Parameter fungieren und die Netzeingabe eines Eingabevektors $x \in \RR^n$ auf lineare Art und Weise beeinflussen. Um auch nichtlineare Zusammenhänge darzustellen, werden Aktivierungsfunktionen benutzt.

\begin{defi}[Aktivierungsfunktion]
    \label{def_act_f}
    Eine stetige, monoton steigende und nicht notwendigerweise lineare Funktion $\psi: \RR \rightarrow \RR$ wird als Aktivierungsfunktion bezeichnet.
\end{defi}
Es sei erwähnt, dass auch nicht monotone Aktivierungsfunktionen genutzt werden können, beispielsweise radiale Basisfunktionen\cite{radialbasis}, welche jedoch in dieser Arbeit nicht weiter von Interesse sind.
Typische Aktivierungsfunktionen, die oft verwendet werden, sind die:
\begin{align*}
    \text{Identität}: \; \;\psi(x)&=x, \\
    \text{Logistische Funktion}: \; \;\psi(x)&=\frac{1}{1+\mathrm{e}^{-x}}, \\
    \text{Tangens Hyperbolicus}: \; \;\psi(x)&=\tanh(x), \\
    \text{ReLU (rectified linear unit)}: \; \;\psi(x)&=\max\{0,x\}.
\end{align*}

\begin{bem}
    Ist $\psi$ eine Aktivierungsfunktion, so wird für $x \in \RR^n$ mit 
    \[\psi(x):=\left(\psi(x_1), \ldots, \psi(x_n)\right)^T \in \RR^n
    \]
    der Vektor bezeichnet, welcher sich durch die elementweise Auswertung der Aktivierungsfunktion $\psi$ an den Stellen $x_1, \ldots, x_n$ ergibt. 
\end{bem}

Bei Klassifikationsproblemen wird oft die \textit{Softmax}-Funktion\cite{denker1990transforming} genutzt, welche die gesamte Eingabe berücksichtigt. Im Abschnitt \ref{task_training} wird erläutert, warum sich in diesem Fall die Softmax-Funktion eignet.

\begin{defi}[Softmax]
    Für $x \in \RR^n$ wird die Funktion $\psi: \RR^n \rightarrow (0,1]^n$ mit 
    \[
        \psi(x):=\left(\frac{\mathrm{e}^{x_1}}{\sum_{i=1}^n \mathrm{e}^{x_i}}, \ldots,\frac{\mathrm{e}^{x_n}}{\sum_{i=1}^n \mathrm{e}^{x_i}} \right)^T
    \]
    als Softmax-Funktion definiert. Die Einträge des Vektors $\psi(x)$ summieren sich zu Eins.  
\end{defi}

Für den späteren Trainingsprozess ist es nützlich, die Ableitung der verwendeten Aktivierungsfunktion, sofern sie exisitiert, zur Verfügung zu haben. Zudem ist es möglich, für bestimmte Aktivierungsfunktionen die Ableitung nur mithilfe der verwendeten Funktion zu berechnen.

\begin{lem}
    \begin{itemize}
        \item[(i)] Für die ReLU $\psi(x)=\max\{0,x\}$ gilt
         \[\psi'(x)=\begin{cases}
            0 &, x <0 \\
            1 &, x >0
        \end{cases}. 
        \]
        An der Stelle 0 ist die Ableitung nicht definiert und wird oft mit $\psi'(0)=\frac{1}{2}$ festgelegt.
        \item[(ii)] Für die logistische Funktion $\psi(x)=\frac{1}{1+\mathrm{e}^{-x}}$ gilt
        \[ 
            \psi'(x)=\psi(x)(1-\psi(x)) 
        \]
        für alle $x \in \RR$.
        \item[(iii)] Für den Tangens Hyperbolicus $\psi(x)=\tanh(x)$ gilt
        \[ 
            \psi'(x)=1-\psi^2(x) 
        \]
        für alle $x \in \RR$.
    \end{itemize}
\end{lem}
\begin{proof}
    Einfaches Differenzieren liefert für $(i)$ und $(ii)$ die Resultate. Bei $(iii)$ wird die Darstellung $\tanh(x)=\frac{2}{\mathrm{e}^{2x}+1}$ genutzt und das Differenzieren mittels Quotientenregel liefert die Aussage.
\end{proof}


Ähnlich der Definition des Perzeptron \ref{def_neuron} wird nun eine Schicht als Verknüpfung von Übertragungsfunktion und Aktivierungsfunktion definiert.

\begin{defi}[Neuronenschicht]
    Ist $\Psi^{W,b}$ eine Übertragungsfunktion mit den Parametern $W \in \RR^{n \times m}, b \in \RR^m$ und $\psi$ eine Aktivierungsfunktion, so wird das Paar $(\Psi^{W,b}, \psi)$ als Schicht $\mathcal{S}$ bezeichnet. Für eine Eingabe $x \in \RR^n$ ist die Ausgabe $y \in \RR^m$ der Schicht $\mathcal{S}$ durch
    \[y=\psi \circ \Psi^{W,b}(x)= \psi\left(\Psi^{W,b}(x)\right)
        \] 
        gegeben. Die Komponenten $y_i$ werden für $1 \leq i \leq m$ Neuronen der Schicht $\mathcal{S}$ genannt und gleichen jeweils der Ausgabe eines einfachen Perzeptrons wie in Definition \ref{def_neuron}. Eine Schicht besteht aus $m$ Perzeptronen $\tilde{\Psi}_i$ mit $y_i=\tilde{\Psi}(x_i)=\psi(W_{i,:}^T x+b_i)$ für $1 \leq i \leq m$.
\end{defi}
Im Hinblick auf MLPs werden nun mehrere Schichten so verbunden, dass die Ausgabe einer Schicht $\mathcal{S}_k$ als Eingabe einer darüberliegenden Schicht $\mathcal{S}_{k+1}$ für ein $k \in \mathbb{N}$ dient. Die Anzahl der Neuronen kann dabei von Schicht zu Schicht variieren. Dementsprechend werden die Dimensionen der beteiligten Gewichtsmatrizen $W^{(k)}$ und Biasvektoren $b^{(k)}$ passend gewählt. 
Um die Notation übersichtlich zu halten, bezeichne $\Psi^{W^{(k)},b^{(k)},\psi_{k}}$ die Schicht $\mathcal{S}_k$ mit $\Psi^{W^{(k)},b^{(k)},\psi_{k}}(x):= \psi_{k} \left(\Psi^{W^{(k)},b^{(k)}}(x)\right)$.

\begin{defi}[Multi-Layer-Perzeptron, vgl. gruening]
    Für eine gegebene Anzahl $l \in \mathbb{N}, \; l>1$ von Schichten $\Psi^{W^{(1)},b^{(1)},\psi_{1}}, \ldots, \Psi^{W^{(l)},b^{(l)},\psi_{l}}$ bezeichne $s_l \in \mathbb{N}$ die Anzahl der Neuronen in Schicht $l$. Für eine Eingabe $x \in \RR^{s_0}$ lässt sich die Ausgabe $y \in \RR^{s_l}$ eines Multi-Layer-Perzeptron  $\Lambda_l: \RR^{s_0} \rightarrow \RR^{s_l}, \; x \mapsto y$ mit $l$ Schichten durch
    \[
        y=\Psi^{W^{(l)},b^{(l)},\psi_{l}} \circ \ldots \circ \Psi^{W^{(1)},b^{(1)},\psi_{1}}(x)
    \]
    berechnen. Dabei gelten für die Gewichtsmatrizen die Dimensionsbedingungen
    \[{}_1W^{(1)}=s_0, \; \; {}_2W^{(l)}=s_l, \; \; \forall i \in [l-1]: \; {}_2W^{(i)}={}_1W^{(i+1)}.
        \] 
    Die Eingabeschicht $\mathcal{S}_0$ besitzt keine Parameter $W$ und $b$ und besteht nur aus dem Eingabevektor $x \in \RR^{s_0}$. Die letzte Schicht $\Psi^{W^{(l)},b^{(l)},\psi_{l}}$ wird als Ausgabeschicht bezeichnet. Weiter werden die Schichten $\mathcal{S}_1, \ldots, \mathcal{S}_{l-1}$ als verdeckte Schichten definiert. Die Funktionsauswertung $\Lambda_l(x)$ für eine Eingabe $x$ wird Vorwärtsrechnung, engl. \textit{forward propagation}, genannt.
\end{defi}

\begin{algorithm}
    \caption{Vorwärtsrechnung}\label{alg:ff}
    \begin{algorithmic}
    \Require $ \text{MLP} \; \Lambda_l, \text{Eingabe} \; x_0 \in \RR^n$
    \Ensure $y = \Lambda_l(x) \in \RR^m$
    \State $x=x_0$
    \For{$i=1, \ldots l}$
    \State $u=W^{(i) \,T}x+b^{(i)}$
    \State $x=\psi_i(u)$
    \EndFor
    \State $y=x$
    \end{algorithmic}
\end{algorithm}
    


Das MLP-Modell wird im weiteren Verlauf dieser Arbeit repräsentativ als Künstliches Neuronales Netz bezeichnet. Die Funktionsauswertung eines KNN wird im Algorithmus Vorwärtsrechnung \ref{alg:ff} festgehalten. Das zuvor angesprochene XOR-Problem kann nun beispielsweise mithilfe eines KNN bestehend aus zwei Schichten gelöst werden\cite{Goodfellow-et-al-2016}.
Es lassen sich zwischen Modell- und Hyperparameter von KNN unterscheiden.

\begin{defi}[Hyper- und Modellparameter]
    Sei für $l \in \mathbb{N}$ ein KNN $\Lambda_l$ gegeben. Dann werden die Eingabe- und Ausgabedimension $s_0, s_l$, die Anzahl $l$ der (verdeckten) Schichten sowie die verwendeten Aktivierungsfunktion $\psi_l$ Hyperparameter des neuronalen Netzes genannt.
    Die Gewichtsmatrizen und Biasvektoren mit den entsprechend passenden Abmessungen stellen die Modellparameter $\mathcal{W}:=\{(W^{(i)},b^{(i)}): \; i=1, \ldots, l\}$ des neuronales Netzes dar. 
\end{defi}
Die Hyperparameter werden oft anwendungsspezifisch für das jeweilige Problem gewählt, während die Modellparameter dynamisch in einem Trainingsprozes angepasst werden, sodass die gegebene Aufgabe zufriedenstellend gelöst wird. Wie sies geschieht, wird im folgenden Abschnitt \ref{task_training} erläutert.

\section{Training neuronaler Netze}
\label{task_training}
Künstliche Neuronale Netze gehören zu den typischen Vertretern von maschinellen Lernalgorithmen, welche hinsichtlich einer bestimmten Aufgabe, engl. \textit{task T}, und einem Leistungsmaß, engl. \textit{perfomance P} an der Erfahrung, engl. \textit{experience E} lernen\cite{Goodfellow-et-al-2016}. Dabei ist mit Lernen gemeint, dass das Computerprogramm bezüglich der Aufgabe $T$ sein Leistungsmaß $P$ mit wachsener Erfahrung $E$ schrittweise steigert. Wie in Kapitel \ref{fund} erläutert, gibt es viele verschiedene Aufgaben, wie die Regression, Klassifikation oder Clusterung bestimmter Objekte. 

In den folgenden Abschnitten wird das Klassifikationsproblem asl \textit{task T} im Mittelpunkt stehen. Weiter werden KNNs als Modellschätzer aus der Wahrscheinlichkeitstheorie interpretiert und fundamentale Aussagen wie das \textit{Universal-Approximation-Theroem}\cite{HORNIK1989359} gegeben. Schließlich wird bezüglich der Klassifikationsaufgabe das Training neuronaler Netze erläutert.


\subsection{Neuronale Netze als universelle Schätzer}
\label{NN_estimators_abs}
Beim Klassifikationsproblem müssen bestimmte bedingte Wahrscheinlichkeiten, die in diesem Abschnitt erklärt werden, ermittelt werden. Oft wird dazu die Ausgabeschicht eines KNN als Wahrscheinlichkeit interpretiert und daher KNN als Schätzer der bedingten Wahrscheinlichkeiten eingesetzt. Zunächst werden Klassifikationsfunktion und -problem definiert.
\begin{defi}
    \label{def_classfun}
    Seien die Mengen $D \subset \RR^n$ und $\mathcal{C}=\{c_1, \ldots, c_m\}$ gegeben. Eine Funktion $f: D \rightarrow \mathcal{C}$, welche ein Element aus $D$ einer Klasse $c_i \in \mathcal{C}$ zuordnet, wird Klassifikationsfunktion genannt. Hier gibt es $m \in \mathbb{N}$ verschiedene Klassenlabels. Die Menge $D$ wird abstrakter Datensatz genannt.
\end{defi}

Das Ziel beim Klassifikationsproblem ist die Approximation einer nicht bekannten Klassifikationsfunktion $f:D \rightarrow \mathcal{C}$ durch ein Modell $\tilde{f}: D \rightarrow \mathcal{C}$. 
In dieser Arbeit werden dafür KNNs genutzt, welche als probabilistische Modelle auf folgende Weise genutzt werden. Auf der Ergebnismenge $\Omega= D \times \mathcal{C}$ sei die nicht bekannte gemeinsame (Wahrscheinlichkeits-) Verteilung $p_{Daten}(x,c)$, genannt Datenverteilung, gegeben. Ein Modell soll nun konstruiert werden, welches die a posterior-Verteilung $p_{Daten}(\cdot \; | \; x)$ der Klassen schätzt. 

In dieser Arbeit werden KNN so benutzt, dass die Klassenzugehörigkeit direkt anhand der Eingabe $x \in D$ geschätzt wird. Die Funktion $P_{Daten}: D \rightarrow [0,1]^m$ mit
\begin{equation}
    \label{eq:Pdaten}
    P_{Daten}(x):=\left(p_{Daten}(c_1 \; | \; x), \ldots, p_{Daten}(c_m \; | \; x)\right)^T \in \RR^m
\end{equation} soll für alle $x \in D$ approximiert werden. Dazu wird die Funktion $P_{Modell}: D \rightarrow [0,1]^m$ mit 
\begin{equation}
    \label{eq:Pmodell}
    P_{Modell}(x;\mathcal{W}):=\left(p_{Modell}(c_1 \; | \; x; \mathcal{W}), \ldots, p_{Modell}(c_m \; | \; x; \mathcal{W})\right)^T \in \RR^m
\end{equation}
 für alle $x \in D$  genutzt, welche von den Modellparametern $\mathcal{W}$ abhängig ist. Die Klassifikationsfunktion des Modells ergibt sich als
\begin{equation}
    \label{eq:f_modell}
    f_{Modell}(x):= \underset{c \in \mathcal{C}}{\mathrm{argmax}} \; p_{Modell}(c \; | \; x).
\end{equation}

Es stellt sich die Frage, inwiefern das MLP als Modell genutzt werden kann, um beliebige Datenverteilungen $P_{Daten}$ zu approximieren. Folgende Resultate liefern die Antwort.

\begin{satz}[Universal-Approximation-Theroem\cite{gruen}]
    \label{UAT}
    Sei $\psi_1$ eine nichtkonstante, beschränkkte Aktivierungsfunktion und $id: \RR \rightarrow \RR$ die Identität sowie $D \subset \RR^n$ kompakt. Dann existieren für alle $\varepsilon >0$ und stetigen Funktionen $f: D \rightarrow \RR$ Parameter $N \in \mathbb{N}, W^{(1)} \in \RR^{n \times N}, b^{(1)} \in \RR^N$ sowie $W^{(2)} \in \RR^{N \times 1}$, sodass
    \begin{equation}
        \label{UAP_eq}
        \left|f(x)-\Psi^{W^{(2)},0,id} \circ \Psi^{W^{(1)},b^{(1)},\psi_1}(x)\right| < \varepsilon, \; \; \forall x \in D
    \end{equation}
    gilt.
\end{satz}

\begin{proof}
    Ein Beweis kann in Hornik\cite{hornik1991approximation} nachgelesen werden.
\end{proof}

Das Universal-Approximation-Theroem kann ebenfalls auf unbeschränkte und nichtkonstante Funktion $f:D \rightarrow \RR^m$ erweitert werden. Heutzutage wird oft die ReLU-Funktion als Aktivierungsfunktion verwendet\cite{schmidt2020nonparametric,li2017convergence}.

\begin{kor}
    Mit den gleichen Vorraussetzungen wie in Satz \ref{UAT} gilt die Abschätzung \ref{UAP_eq} für $\psi_1(x)=\max\{0,x\}$.
\end{kor}

\begin{proof}
    Siehe Sonoda et. al.\cite{sonoda2017neural}.
\end{proof}

Hinsichtlich der Approximation von beliebigen Funktionen $P_{Daten}$ mithilfe eines neuronalen Netzes mit der Softmax-Funktion als Aktivierungsfunktion liefert Strauß\cite{strauss} folgendes Resultat.

\begin{kor}
    \label{kor_softmax}
    Ein MLP mit zwei Schichten, wobei $\psi_2$ die Softmax-Funktion ist, kann genutzt werden, um stetige Funktionen $f:K \rightarrow [0,1]^m$, welche von einem Kompaktum $K \subset \RR^n$ in eine (Wahrscheinlichkeits)-Verteilung über die Klassen $\mathcal{C}$ abbilden, beliebig genau zu approximieren.
\end{kor}

\begin{proof}
    Siehe \cite{strauss}.
\end{proof}

Die Aussage kann auf das MLP mit beliebig vielen Schichten erweitert werden.
In dieser Arbeit umfasst die Menge $D$ aus Definition \ref{def_classfun} digitalisierte Objekte und ist endlich und damit kompakt. Daher kann wegen Korollar \ref{kor_softmax} das MLP als Modell genutzt werden, um stetige Funktionen $P_{Daten}$ sinnvoll zu approximieren.

\subsection{Optimale Parameterwahl bei neuronalen Netzen}

Wird ein künstliches neuronales Netz als probabilistisches Modell genutzt und sind die Hyperparameter festgelegt, müssen die Modellparameter $\mathcal{W}$ gewählt werden. Um die Approximationsgüte, also die \textit{perfomance P}, bezüglich des Klassifikationsproblems messbar zu machen, werden Fehlerfunktionen eingeführt. 
Mit Trainingsdaten als \textit{experience E} und dem Gradientenverfahren\cite{nocedal1999numerical} sollen optimale Parameter $\mathcal{W}$ gefunden werden, sodass die gewählte Fehlerfunktion minimiert wird. Im folgenden sei $\Lambda_l$ ein KNN mit der Softmax-Funktion als Aktivierungsfunktion in der Ausgabeschicht, welches als parametrisiertes Modell $f_{Modell}$ wie in \ref{eq:f_modell} genutzt wird.

\begin{defi}[Trainingsmenge]
    Seien $p_{daten}$ eine Datenverteilung und $\Omega=D \times \mathcal{C}$. Dann heißt für ein $k \in \mathbb{N}$ die Menge
    \begin{equation*}
        \mathcal{T}:=\left\{ (x^{(i)},c^{(i)}) \; | \; i \in [k] \right\} \subset \Omega
    \end{equation*}
    Trainingsmenge bestehend aus $k$ Trainingsdaten, welche unabhängig durch $p_{Daten}$ generiert wurde.
\end{defi}

Die Güte des Modells $f_{Modell}$ wird als Likelihood gegeben einer Trainingsmenge $\mathcal{T}$ gemessen und lässt sich als 

\begin{equation}
    \label{eq:likelihood}
    L(\mathcal{T},\mathcal{W}):=\prod_{(x,c) \in \mathcal{T}} p_{Modell}(c \; | \; x; \mathcal{W})
\end{equation}
wie in Bishop\cite{bishop2006pattern} berechnen.
Für eine Trainingsmenge $\mathcal{T}$ soll das Produkt über alle Wahrscheinlichkeiten der korrekten Klassenzugehörigkeiten $c$ gegeben der Eingaben $x$ maximiert werden. Dieser Ansatz wird \textit{Maximum Likelihood-Methode}\cite{ruschendorf2014mathematische} genannt und eine Parameterwahl ist durch eine Lösung des Optimierungsproblems 
\begin{equation}
    \label{eq:opt_likelihood}
     \prod_{(x,c) \in \mathcal{T}} p_{Modell}(c \; | \; x; \mathcal{W}) \rightarrow \max
\end{equation}
gegeben. Dabei sei bemerkt, dass die Optimierung unabhängig von den Hyperparametern vorgenommen wird.

Für ein Trainingspaar $(x,c) \in \mathcal{T}$ bezeichne $t(x,c) \in \RR^m$ den Zielvektor der Klasse $c$ mit sogenannter (1 aus m)-Kodierung. Die Komponenten des Zielvektors sind
\begin{equation*}
    t_k(x,c):= \begin{cases}
        1 &, \text{wenn} \; k=c \\
        0 &, \text{sonst}
    \end{cases}, \; \; \forall k \in [m].
\end{equation*} 
Mit dieser Bezeichnung lässt sich das Optimierungsproblem \ref{eq:opt_likelihood} als Minimierungproblem mithilfe der \textit{negative log likelihood} schreiben.
\begin{defi}[negative log likelihood]
    Seien die Mengen $D$ und  $\mathcal{C}=\{c_1, \ldots, c_m\}$ mit einer dazugehörigen Trainingsmenge $\mathcal{T}$ sowie entsprechende Zielvektoren gegeben. Weiter seien die a posterior Wahrscheinlichkeiten $p_{Modell}(c \; | \; x; \mathcal{W})$ wie in Gleichung \ref{eq:Pmodell} gegeben. Die negative log likelihood ist als Funktion 
    \begin{equation}
        \label{eq:NLL}
        L_{NNL}(\mathcal{T},\mathcal{W}):= -\sum_{(x,c) \in \mathcal{T}}  \sum_{i=1}^m t_i(x,c) \log \left(p_{Modell}(c_i \; | \; x; \mathcal{W}) \right) 
    \end{equation}
    definiert.
\end{defi}

Das Minimieren der negative log likelihood ist äquivalent zur Maximierung der Likelihood aus \ref{eq:likelihood}, denn es gilt 
\begin{equation*}
    \log \left(\prod_{(x,c) \in \mathcal{T}} p_{Modell}(c \; | \; x; \mathcal{W})\right)= \sum_{(x,c) \in \mathcal{T}} \log \left(p_{Modell}(c \; | \; x; \mathcal{W}) \right)
\end{equation*}
und der natürliche Logarithmus ist monoton steigend. Wird zusätzlich angenommen, dass die Datenverteilung einer Normalverteilung mit konstanter Varianz entspricht, so ist das Maximieren von \ref{eq:likelihood} äquivalent zur Minimierung der mittleren quadratischen Abweichung
\begin{equation*}
    \label{eq:MSE}
    L_{MSE}(\mathcal{T},\mathcal{W}):=\frac{1}{2} \sum_{(x,c) \in \mathcal{T}} ||\hat{c}-t(x,c)||_2^2,
\end{equation*}
wobei $\hat{c}=f_{Modell}(x)$ und $t(x,c)$ der Zielvektor des Datenpaars $(x,c)$ ist, siehe Goodfellow\cite{Goodfellow-et-al-2016}.

 Das Problem \ref{eq:opt_likelihood} wird nun allgmemein mit Fehlerfunktionen definiert.
 \begin{defi}
    Seien $\mathcal{T}$ eine Trainingsmenge und $\mathcal{W}$ Modellparameter eines KNN. Mithilfe des Gradientenverfahrens soll das Problem
    \begin{equation}
        \label{eq:error_fun_opt}
        \mathcal{E}(\mathcal{T},\mathcal{W}) \rightarrow \min
    \end{equation}
    gelöst werden. Dabei wird $\mathcal{E}$ Fehlerfunktion gennant.
 \end{defi}
In dieser Arbeit wird $\mathcal{E}$ immer als stückweise stetig differenzierbare Funktion gewählt, damit das Gradientenverfahren angewendet werden kann. Sowohl die negative log likelihood $L_{NNL}$ als auch die mittlere quadratische Abweichung $L_{MSE}$ sind als Fehlerfunktion geeignet. Die Optimierung der Parameter geschieht iterativ und besteht jeweils aus zwei Schritten. Zuerst wird eine Abstiegsrichtung 
\begin{equation}
    \label{eq:gradE}
    \Delta_n :=\nabla_{\mathcal{W}} \mathcal{E}(\mathcal{T},\mathcal{W})
\end{equation} 
berechnet und dann die Parameter 
\begin{equation}
    \label{eq:step}
    \mathcal{W}_{n+1}:=\mathcal{W}_n- \lambda \Delta_n
\end{equation}
aktualisiert. Es werden also Gradienten der Fehlerfunktion bezüglich der Gewichtsmatrizen und Biasvektoren ermittelt und anschließend werden jene Parameter mit einer Lernrate $\lambda \in \RR$ angepasst. Hier wird der Gradient über alle Trainingspaare berechnet. Diese Variante nennt sich \textit{Offline-Version} des Gradientenverfahrens und ist besonders für große Trainingsmengen ineffizient. Die \textit{Online-Version} berechnet den Gradienten lediglich für ein Trainingspaar und passt die Parameter direkt an. In dieser Arbeit wird ein Kompromis aus beiden Verfahren verwendet und zwar das \textit{Mini-Batch-Verfahren} \ref{alg:minibatch}, bei dem die Gradienten über kleine Teilmengen der Trainingsmenge $\mathbb{T} \subset \mathcal{T}$ berechnet werden. Für eine tiefere Analyse des Gradientenverfahrens sei auf Ruder\cite{ruder2016overview} verwiesen. 

\begin{algorithm}
    \caption{Mini-Batch-Verfahren}\label{alg:minibatch}
    \begin{algorithmic}
    \Require $ \text{Trainingsmenge} \; \mathcal{T}, \text{Modellparameter} \; \mathcal{W}, \text{Fehlerfunktion} \; \mathcal{E}, \text{Batch-Größe} \; n$
    \Ensure $\text{optimierte Modellparamter} \; \mathcal{W}$
    \State $x=x_0$
    \For{$i=1, \ldots l}$
    \State $u=W^{(i) \,T}x+b^{(i)}$
    \State $x=\psi_i(u)$
    \EndFor
    \State $y=x$
    \end{algorithmic}
\end{algorithm}
    






\chapter{Die Faltung}
In diesem Kapitel wird erläutert, wie die Faltung (engl. \textit{convolution}) bei CNNs zu verstehen ist und wie diese Operation bei diesen neuronalen Netzen motiviert wird. In diesem Zusammenhang werden Begriffe wie Merkmalskarten (engl. \textit{feature maps}) und Filter (engl. \textit{kernels}) eingeführt. Des Weiteren wird die Arithmetik der Faltungsoperation für zweidimensionale Eingaben, repräsentiert durch Matrizen, erklärt und das Verfahren \textit{padding} beziehungsweise das Nutzen von \textit{strides} erläutert. Das gesamte Kapitel wird mit konkreten Beispielen begleitet, um die verschiedenen Effekte der Faltungsoperation zu beleuchten.

\section{Die Faltungsoperation}
In der Analysis ist die Faltung ein mathematischer Operator und liefert für zwei Funktionen $f$ und $g$ die Funktion $ f \ast g$, wobei mit dem Sternchen die Faltungsoperation gemeint ist.

\begin{defi}[Faltung]\label{allg_faltung}
    Für zwei Funktionen $f,g: \Rnv \rightarrow \mathbb{C}$ ist die Faltung als
    \begin{equation*}
        (f \ast g) (x) := \int_{\Rnv} f(\tau) g(x-\tau) \mathrm{d} \tau
    \end{equation*}
    definiert, wobei gefordert wird, dass das Integral für fast alle $x$ wohldefiniert ist.
   \end{defi}

Bei klassischen neuronalen Netzen, siehe Kapitel \ref{classicNN} werden Eingabedaten durch eine Verkettung von affinen Transformationen verarbeitet. Typischerweise wird die Eingabe als Vektor dargestellt und  mit einer Matrix multipliziert, gegebenenfalls mit einem Biasvektor manipuliert und schließlich so die Ausgabe generiert. Bilder-, Audio- oder Videoaufnahmen besitzen jedoch mehrere Merkmale in unterschiedlichen Achsen. Oft sind solche Eingabedaten im Bereich des Machine-Learnings als mehrdimensionale Arrays abgelegt, welche eine oder mehrere Achsen repräsentieren, wobei die Ordnung dieser eine Rolle spielt. Bei digitalisierten Bildern sind das bespielsweise die Höhe und Breite des Bildes, bei Audioaufnahmen gibt es nur eine Achse, und zwar die Zeitachse. Hinzu kommen Kanalachsen als weitere Verfeinerung der Daten, zum Beispiel besitzen RGB-Farbbilder drei Kanäle der Farben rot, grün und blau. 

Diese speziellen Eigenschaften können bei affinen Transformationen nicht berücksichtigt werden. Alle Merkmale sowie Achsen werden gewissermaßen gleich behandelt und die wesentliche topologische Struktur kann so nicht zum Vorteil ausgenutzt werden. Hier soll nun die sogenannte diskrete Faltung Abhilfe schaffen.

\begin{defi}[Diskrete Faltung]\label{disk_faltung}
    Für zwei Funktionen $f,g: D \rightarrow \mathbb{C}$ mit einem diskreten Definitionsbereich $D \subseteq \mathbb{Z}^n$ ist die diskrete Faltung als
    \begin{equation*}
        (f \ast g) (n) := \sum_{k \in D} f(k) g(x-k)
    \end{equation*}
    definiert. Hier wird über dem gesamten Definitionsbereich $D$ summiert. Ist $D$ beschränkt, werden $f$ beziehungsweise $g$ durch Nullen fortgesetzt. 
   \end{defi}

Besonders bei der Bildverarbeitung wird oft die diskrete Faltung als lineare Operation verwendet. 

\begin{defi}[Matrixfaltung, vgl. \cite{gruening}] \label{matrix_faltung}
    Für gegebene Matrizen $X \in \RR^{h \times w}$ und $K \in \RR^{k_h \times k_w}$ seien 
    \begin{equation*}
        h_l=\begin{cases}
           \lfloor k_h/2  \rfloor &, k_h \, \text{ungerade} \\
           k_h/2-1 &, \text{sonst}
        \end{cases}, \; \; 
        w_l=\begin{cases}
            \lfloor k_w/2 \rfloor &, k_w \, \text{ungerade} \\
            k_w/2 -1 &, \text{sonst}    
        \end{cases}.
    \end{equation*} 
    Die Matrixfaltung $K \ast X \in \RR^{h \times w}$ ist als 
    \begin{equation}
        \label{matrix_faltung_op}
        (K \ast X)_{i,j}:=\sum_{l=-h_l}^{\lfloor k_h/2  \rfloor} \sum_{m=-w_l}^{\lfloor k_w/2 \rfloor} K_{l+h_l+1, m+w_l+1} X_{i+l,j+m} \; \; \forall i \in [h], j \in [w]
    \end{equation} mit $X_{i,j}=0$ für $i \notin [h]$ und $j \notin [w]$ definiert.
    \end{defi}

\begin{bem}\label{bem_strides}
    Die Matrixfaltung kann durch viele weitere Parameter genauer spezifiziert werden. Sogannte \textit{strides} bestimmen die Reduktion bei der Faltung von $X$ in der Höhe $h$ beziehungsweise Breite $w$. Für strides $s_h, s_w \in \mathbb{N}$ ist die Matrixfaltung als
    \begin{equation*}
        (K \ast X)_{i,j}:=\sum_{l=-h_l}^{\lfloor k_h/2  \rfloor} \sum_{m=-w_l}^{\lfloor k_w/2 \rfloor} K_{l+h_l+1, m+w_l+1} X_{i \cdot s_h +l,j \cdot s_w +m} \; \; \forall i \in [h], j \in [w].
    \end{equation*}
    Für $s_h=s_w=1$ ergibt sich die Standardvariante wie in \ref{matrix_faltung_op}.
    \end{bem}

Im Folgenden werden konkrete Beispiele für verschiedene zweidimensionale Faltungen, welche in dieser Arbeit im Fokus stehen, gegeben. Dabei sind die Eingabe $X \in \RR^{h \times w}$ und der Filter $K \in \RR^{k_h \times k_w}$ immer als Matrizen zu verstehen. Das Ergebnis der Faltung $S= X \ast K$ wird als Merkmalskarte bezeichnet. Es sei angemerkt, dass oft $k_h=k_w$ sowie $k_h$ ungerade gewählt wird, z.B. $k_h=3$ oder $k_h=5$. Die Größe der Merkmalskarte wird durch die Parameter
\begin{itemize}
    \item $h,w$: Die Höhe und Breite der Eingabe,
    \item $k_h, k_w$: Die Abmessungen des Filters,
    \item $s_h, s_w$: Die Wahl der strides, 
    \item $p_h, p_w$: Die Größe des zero paddings
\end{itemize}
beeinflusst. Mit zero padding ist gemeint, dass künstliche Nullen um Randpixel der Eingabe $X$ eingefügt werden, damit die Berechnung mit dem Filter um jene Pixel gelingt. Ein Beispiel für das Verwenden von zero padding wird in Abbildung \ref{abb_simplematrixconv_padding} gezeigt. In Abbildung \ref{abb_simplematrixconv} ist die Berechnung einer einfachen zweidimensionalen Matrixfaltung dargestellt. Ein vorher festgelegter Filter (grau) bewegt sich über die Eingabe (blau) und berechnet jeweils die Einträge der Ausgabe(grün). 

\begin{figure}[h]
    \includegraphics[width=0.8\textwidth]{pics/abb_simpleconv}
    \centering
    \caption{Es wird die Merkmalskarte $S \in \RR^{3 \times 3}$ mit den Parametern ${h,w=5}, k_h=k_w=3, s_h=s_w=1$ und $p_h=p_w=1.$}
    \label{abb_simplematrixconv}
\end{figure}

\begin{figure}[h]
    \includegraphics[width=0.8\textwidth]{pics/abb_simplecov_padding}
    \centering
    \caption{Es wird die Merkmalskarte $S \in \RR^{3 \times 3}$ mit den Parametern ${h,w=5}, k_h=k_w=3, s_h=s_w=2$ und $p_h=p_w=1$ berechnet.}
    \label{abb_simplematrixconv_padding}
\end{figure}



\section{Motivation der Faltung}
..
Sie nutzt wichtige Konzepte zur Optimierung von Machine-Learning-Verfahren wie spärliche Konnektivität (engl. \textit{sparse connectivity}), \textit{Parameter Sharing} und \textit{äquivariante Repräsentation}, vgl. \cite{goodfellow}. Spärlicher Konnektivität bedeutet, dass die Ausgabeeinheit auf einer bestimmten Schicht nur durch wenige Eingabeeinheiten beeinflusst wird. Dies ist bei CNNs typisch, da meist die verwendeten Filter viel kleiner als die Eingabe ist. Noch mehr erklären + Abbildung

Mit Parameter Sharing ist die Nutzung von gleichen Parametern für mehrere Funktionen im neuronalen Netz gemeint. In herkömmlichen Feed-Forward-Netzen wird jedes Element der Gewichtsmatrizen für die Berechnung der Aktivierungen der jeweiligen Schichten verwendet. Anschließend werden diese Gewichte dann nicht mehr gebraucht. Im Zusammenhang von CNNs bedeutet Parameter Sharing während der Faltungsoperation, dass nur eine bestimmte Menge von Parametern erlernt werden müssen
Noch mehr erklären + Abbildung



\chapter{Weiteres Kapitel}
Hier wird dies und das vorgestellt. Unter anderem Fußnoten.\footnote{Dies ist eine Fußnote.}
\section{Umgebungen und Formeln}
\begin{defi}\label{defi1}
 \dots\ heißt \ac{rnn}. 
\end{defi}
\begin{bem}\label{bem1}
Bei jeder weiteren Verwendung der Abkürzung wird nur die Kurzform angezeigt: \ac{rnn}.
\end{bem}
\begin{bem}\label{bem2}
Die Verwendung des Symbolverzeichnisses ist analog der des Abkürzungsverzeichnisses, siehe \ac{cm}.
\end{bem}
\begin{annahme}\label{annahme1}
Eine kluge Annahme \dots
\end{annahme}
\begin{hsatz}\label{hsatz1}
Ein kluger Hilfssatz \dots
\end{hsatz}
\begin{satz}\label{satz1}
Ein kluger Satz \dots
\end{satz}
\begin{kor}\label{kor1}
Ein kluges Korollar \dots
\end{kor}
\begin{prop}\label{prop1}
Eine kluge Proposition \dots
\end{prop}
\begin{problem}\label{problem1}
Ein schweres Problem \dots
\end{problem}
\begin{bsp}\label{bsp1}
Ein anschauliches Beispiel \dots
\end{bsp}



\begin{defi}\label{defi2}
Seien $a,b\in\mathbb{C}$ definiere 
\begin{equation}\label{eq1}
a+b
\end{equation}
als  \dots
\end{defi}

Auf Formeln kann nun verwiesen werden (siehe \eqref{eq1}). Formeln können natürlich auch im normalen Text $a^2+b^2=c^2$ auftauchen. 


\begin{empheq}[box=\colorbox{mycolor},right=\empheqrbrace \text{ohne Sinn}]{align}
a^2+b^2&=c^2\\
f&=b-a
\end{empheq}
\section{Aufzählung und Nummerierung}
Für Literaturvereichnisse siehe Kapitel~\ref{ch:bib}, eine einfache Aufzählung geht so:
\begin{itemize}
 \item Eins
 \item Zwei
 \item Viele
\end{itemize}


\section{Tabellen}
\begin{table}[ht]
 \dots\ gibt es viele verschiedene, z.\,B.\ Tab.~\ref{tab:reservoir_segmentlabelling} und Tab.~\ref{tab:dlv_restr_all}. 
\caption{Einfache Tabelle}
\label{tab:reservoir_segmentlabelling}
\centering
  \begin{tabular}{ccrr}
        \toprule
             Column 1 & Column 2 & Column 3 & Column 4 \\
            \midrule
            Nein & Softmax & \SI{85,0}{\%} & \cellcolor{hellgrau} \SI{87,0}{\%} \\
            Nein & Linear & \cellcolor{hellgrau} \SI{88,8}{\%} & \SI{85,9}{\%} \\
            Ja & Softmax & \SI{80,0}{\%} & \cellcolor{hellgrau} \SI{89,1}{\%} \\
            Ja & Linear& \SI{84,6}{\%} & \cellcolor{hellgrau} \SI{89,8}{\%} \\
            \bottomrule
  \end{tabular}
\end{table} 

\begin{table}[ht]
\caption{Nicht mehr ganz so einfache Tabelle}\label{tab:dlv_restr_all}
\centering
 \begin{tabular}{llcccccc}
  \hline\noalign{\smallskip}
  	&source prior				&\multicolumn{2}{c}{\texttt{abs}}&\multicolumn{2}{c}{\texttt{prior}} & \multicolumn{2}{c}{\texttt{da}}\\	
	&source posterior 			&\texttt{path}&	\texttt{ctc}&   \texttt{path}&	\texttt{ctc}&\texttt{path}&\texttt{ctc}\\
	\noalign{\smallskip}
	\hline
	\noalign{\smallskip}
	\multirow{ 2}{*}{gAP}	&\texttt{normed}	&94.81&	94.89&	95.36&	\textbf{95.42}&	94.99&	95.04\\
				&\texttt{unnormed}	&94.77&	&	91.73&	91.87&	92.58&	\\%\hline
	\noalign{\smallskip}
	\multirow{ 2}{*}{mAP}	&\texttt{normed}	&89.71&	\textbf{89.90}&	89.58&  89.76&	89.63&89.82	\\
				&\texttt{unnormed}	&89.42&	&	88.59&	88.89&	89.13&	\\
	\noalign{\smallskip}
	\multirow{ 2}{*}{gNDCG}	&\texttt{normed}	&96.72&	96.78&	96.78&	\textbf{96.83}&	96.73&96.77	\\
				&\texttt{unnormed}	&96.69&	&	96.34&	96.41&	96.46	&\\
	\noalign{\smallskip}
	\multirow{ 2}{*}{mNDCG}	&\texttt{normed}	&90.77&	\textbf{90.97}&	90.66&  90.85&	90.70&90.89 \\
				&\texttt{unnormed}	&90.61&	&	89.96&	90.25&	90.36& \\	
	\hline
 \end{tabular}
\end{table}



\section{Bilder}
\subsection{Einzelnes Bild}
\begin{figure}[ht]
\centering
\includegraphics[width=0.5\textwidth]{pics/SCHNECKE}
\caption{Vektorgrafiken sind toll. Scrolle mal in mich rein!}
\label{fig:schnecke}
\end{figure}

Das ist Text. Das ist Text. Das ist Text über Abb.~\ref{fig:schnecke}. Das ist Text.\footnote{Dieser Textteil
ist von wesentlicher Bedeutung}

\subsection{Mehrere Bilder}
\begin{figure}[h!t]
\centering
\begin{subfigure}{0.4\textwidth}
\includegraphics[width=\textwidth]{pics/SCHNECKE}
\subcaption{Schnecke 1}
\label{fig:schnecke:a}
\end{subfigure}
\qquad
\begin{subfigure}{0.4\textwidth}
\includegraphics[width=\textwidth]{pics/SCHNECKE}
\subcaption{Schnecke 2}
\label{fig:schnecke:b}
\end{subfigure}
\caption{Vergleich verschiedener Schnecken}
\label{fig:seam}
\end{figure}

Die Schnecke aus Abb.~\ref{fig:schnecke:a} ist hübscher anzusehen als die aus Abb.~\ref{fig:schnecke:b}.

\section{TikZ}
TikZ bietet ein mächtiges Werkzeug Grafiken selber zu erzeugen.
\subsection{Einfache Grafiken}
Es gibt viele, viele Tutorials und Beispiele die leicht im Internet zu finden sind. Aber ein Beispiel sei an dieser Stelle trotzdem eingefügt, siehe Abb.~\ref{fig:tikz}.
\begin{figure}[ht]
\centering
\begin{tikzpicture}[minimum size=5mm,inner sep=0pt]
\node (v1) at (1,.5) [circle,fill=white,draw] {};
\node (v2) at (1.75,.5) [circle,fill=white,draw] {};
\node (v3) at (2.5,.5) [circle,fill=white,draw] {};
\node (v4) at (3.25,.5) [circle,fill=white,draw] {};
\node (v5) at (4,.5) [circle,fill=white,draw] {};
\begin{scope}[on background layer]
\node (vA) [fit=(v1)(v2)(v3)(v4)(v5), minimum width=2cm, inner sep=2pt,fill=white,draw,rounded corners=5pt] {};
\end{scope}
\node (t1) at (0,.5) [] {$\textbf{v}$};
\node (t2) at (1.5,1.25) [] {$Q(\textbf{h}^1|\textbf{v})$};
\node (t3) at (3.5,1.25) [] {$P(\textbf{v}|\textbf{h}^1)$};

\node (h1) at (1,2) [circle,fill=white,draw] {};
\node (h2) at (1.75,2) [circle,fill=white,draw] {};
\node (h3) at (2.5,2) [circle,fill=white,draw] {};
\node (h4) at (3.25,2) [circle,fill=white,draw] {};
\node (h5) at (4,2) [circle,fill=white,draw] {};
\begin{scope}[on background layer]
\node (hA) [fit=(h1)(h2)(h3)(h4)(h5), minimum width=2cm, inner sep=2pt,fill=white,draw,rounded corners=5pt] {};
\end{scope}
\node (t4) at (0,2) [] {$\textbf{h}^1$};
\node (t5) at (1.5,2.75) [] {$Q(\textbf{h}^2|\textbf{h}^1)$};
\node (t6) at (3.5,2.75) [] {$P(\textbf{h}^1|\textbf{h}^2)$};

\draw[thick,->] (hA.300) -- (vA.60);
\draw[thick,->,dotted] (vA.120) -- (hA.240);

\node (h6) at (1,3.5) [circle,fill=white,draw] {};
\node (h7) at (1.75,3.5) [circle,fill=white,draw] {};
\node (h8) at (2.5,3.5) [circle,fill=white,draw] {};
\node (h9) at (3.25,3.5) [circle,fill=white,draw] {};
\node (h10) at (4,3.5) [circle,fill=white,draw] {};
\begin{scope}[on background layer]
\node (hB) [fit=(h6)(h7)(h8)(h9)(h10), minimum width=2cm, inner sep=2pt,fill=white,draw,rounded corners=5pt] {};
\end{scope}
\node (t7) at (0,3.5) [] {$\textbf{h}^2$};
\node (t8) at (3.5,4.25) [] {$P(\textbf{h}^2,\textbf{h}^3)$};

\draw[thick,->] (hB.300) -- (hA.60);
\draw[thick,->,dotted] (hA.120) -- (hB.240);

\node (h11) at (1,5) [circle,fill=white,draw] {};
\node (h12) at (1.75,5) [circle,fill=white,draw] {};
\node (h13) at (2.5,5) [circle,fill=white,draw] {};
\node (h14) at (3.25,5) [circle,fill=white,draw] {};
\node (h15) at (4,5) [circle,fill=white,draw] {};
\begin{scope}[on background layer]
\node (hC) [fit=(h11)(h12)(h13)(h14)(h15), minimum width=2cm, inner sep=2pt,fill=white,draw,rounded corners=5pt] {};
\end{scope}
\node (t9) at (0,5) [] {$\textbf{h}^3$};

\draw[thick,<->] (hB) -- (hC);
\end{tikzpicture}
\caption{Beispiel eines mit TikZ erzeugten Bildes}
\label{fig:tikz}
\end{figure}
\subsection{Graphen und ähnliches}
Wer keine Lust hat z.\,B. Achsenbeschriftungen eines Matlab-Plots auf Font etc. des \LaTeX -Dokuments anzupassen, kann Datenreihen auch einfach mittels TikZ darstellen, siehe dazu Abb.~\ref{fig:ohstats}. Es ist natürlich auch möglich aus z.\,B. Matlab oder Gnuplot Tikz Grafiken zu exportieren!
\begin{figure}[ht]
\centering
\begin{tikzpicture}
		\begin{axis}[width=0.9\textwidth,height=0.3\textheight,
			xtick={0,25,50,75,100,125,150,175,200},
			x tick label style={/pgf/number format/1000 sep=},
			xlabel={Epochs},
			y tick label style={/pgf/number format/1000 sep=},
			ylabel={LBL-Error},
			enlarge x limits=0.0,
			ymin = 0.0,
			ymax = 0.5]		
			\addplot[mark=none,red] table[x=x, y=prime] {tikz/ohdata.txt}; \addlegendentry{System 1}
			\addplot[mark=none,black] table[x=x, y=fixed] {tikz/ohdata.txt}; \addlegendentry{System 2}
			\addplot[mark=none,yellow] table[x=x, y=sma] {tikz/ohdata.txt}; \addlegendentry{System 3}
			\addplot[mark=none,green] table[x=x, y=med] {tikz/ohdata.txt}; \addlegendentry{System 4}
			\addplot[mark=none,blue] table[x=x, y=big] {tikz/ohdata.txt}; \addlegendentry{System 5}
		\end{axis}
\end{tikzpicture}
\caption{Datenreihen mittels TikZ visualisiert}
\label{fig:ohstats}
\end{figure}

\chapter{Ein letztes Kapitel}
\begin{kor}\label{kor2}
Wird f\"{u}r die Festlegung  \dots
\end{kor}
\section{Weiteres Korollar}
In diesem Abschnitt  \dots
\begin{kor}\label{kor3}
Wird f\"{u}r die Festlegung  \dots
\end{kor}

\begin{bem}\label{bem3}
Eine vollst\"{a}ndige  \dots
\end{bem}

\section{Pseudocode}
In vielen Fällen ist es notwendig, Programmteile als Pseudocode darzustellen. Algorithmus \ref{alg:pseudo} stellt ein einfaches Beispiel dar. Es gibt weitere Pakete zur Darstellung von Pseudocode, \texttt{algorithm} + \texttt{algpseudocode} sei an dieser Stelle erwähnt.

\begin{algorithm}[ht]
 \KwData{this text}
 \KwResult{how to write algorithm with \LaTeX2e }
 initialization\;
 \While{not at end of this document}{
  read current\;
  \eIf{understand}{
   go to next section\;
   current section becomes this one\;
   }{
   go back to the beginning of current section\;
  }
 }
 \caption{How to write algorithms}\label{alg:pseudo}
\end{algorithm}


\section{Zitate}
\label{ch:bib}
Umfangreichen Quellenangaben sollte man in einer Literaturdatenbank pflegen. Um diese in \LaTeX\ zu verwenden bietet sich das Paket biblatex mit dem Sortierprogramm
 Biber an, da es gewisse Vorteile gegenüber dem klassischen Bib\TeX\ besitzt. 
Die Verweise liegen in einer separaten Datei (hier: \texttt{literatur.bib}) und werden mit \begin{verbatim}\addbibresource{<nameDerDatei>}\end{verbatim} eingefügt. %Das jeweilige Aussehen bestimmt der Befehl: \begin{verbatim} \bibliographystyle{style} \end{verbatim}
Zitiert wird dann mittels \begin{verbatim}\cite{key}\end{verbatim} was in unserem Beispiel dann so aussieht \cite{forster1983analysis}. 

\noindent ACHTUNG: Beim ändern der {\texttt{.bib}}-Datei und/oder der Zitate muss mehrfach compiliert werden, damit die änderungen auch wirksam werden. Sicher geht man, wenn man die folgende Reihenfolge beachtet: 
\begin{enumerate}
 \item \LaTeX
 \item Biber
 \item \LaTeX
 \item \LaTeX
\end{enumerate}

Eine genauere Beschreibung findet Ihr im Anhang \ref{sec:biber}.
\appendix

\printbibliography %hier Bibliographie ausgeben lassen

%Wenn gewünscht Danksagung einfügen
%\addchap*{Danksagungen}

%%%%%%%%%%%%%%%%%%%%%%%%%%%%%%%%%%%%%%%%%%%%%%%%%
%%%%%%%%%%%%%%%%%% Formalia %%%%%%%%%%%%%%%%%%%%%
%%%%%%%%%%%%%%%%%%%%%%%%%%%%%%%%%%%%%%%%%%%%%%%%%
% Muss in jedem Fall in die Arbeit!
\eigenstaenigkeitserklaerung

\cleardoublepage

%%%%%%%%%%%%%%%%%%%%%%%%%%%%%%%%%%%%%%%%%%%%%%%%%
\pagenumbering{roman}
\chapter{Anhang}
\section{Listings}

\lstset{language=C}
 \begin{lstlisting}[caption=C Code - direkt eingefügt, label=list:C]
#include <stdio.h>
#define N 10
/* Block
 * comment */

int main()
{
    int i;

    // Line comment.
    puts("Hello world!");
    
    for (i = 0; i < N; i++)
    {
        puts("LaTeX is also great for programmers!");
    }

    return 0;
}
\end{lstlisting}




\lstset{language=Java}


\lstinputlisting[label=list:java,caption=Java Code - über externe Datei eingefügt]{code/HelloWorld.java}
% \lstinputlisting[caption=Scheduler, language=C]{hello.c}

\section{Biber}
\label{sec:biber}
\includepdf[pages={1-2},scale=.9]{AnleitungBibLatexBiber.pdf}


\end{document}
