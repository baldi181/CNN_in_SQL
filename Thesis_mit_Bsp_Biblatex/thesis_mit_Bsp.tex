\documentclass[12pt,DIV=15,BCOR=15mm,twoside,headsepline,abstract=true,listof=totoc,bibliography=totoc]{scrreprt}

%%%%%%%%%%%%%%%%%%%%%%%%%%%%%%%%%%%%%%%%%%%%%%%%%%%%%%%%%%%
%%%%%%%%%%%%% Uni Rostock Thesis Style %%%%%%%%%%%%%%%%%%%%
%%% bei Problemen, Mail an susann.dittmer@uni-rostock.de
%%%%%%%%%%%%%%%%%%%%%%%%%%%%%%%%%%%%%%%%%%%%%%%%%%%%%%%%%%%
% enthält schon viele wichtige Pakete
\usepackage[mnf]{thesis_uro} %Fakultät wählen: uni (Standard),inf,msf,ief,mnf,mef,juf,wsf,auf,thf,phf
\usepackage{algorithm}
%\usepackage{algorithmic}
\usepackage{algpseudocode}
\usepackage{ifsym}
\addbibresource{literatur.bib} %Bibliographiedateien laden

%Definition von Umgebungen
\newtheorem{kor}{Korollar}
\newtheorem{hsatz}{Hilfssatz}
\newtheorem{satz}{Satz}
\newtheorem{prop}{Proposition}
\newtheorem{defi}{Definition}
\newtheorem{lem}{Lemma}
\newtheorem{annahme}{Annahme}
\newtheorem{problem}{Problem}
%\theoremstyle{remark}	%Styleänderung (Text aufrecht, ...)
\newtheorem{bem}{Bemerkung}
\newtheorem{bsp}{Beispiel}

%\newtheoremstyle{boldremark}
%    {\dimexpr\topsep/2\relax} % space above
%    {\dimexpr\topsep/2\relax} % space below
%    {}          % body font
%    {}          % indent amount
%    {\bfseries} % theorem head font
 %   {.}         % punctuation after theorem head
 %   {.5em}      % space after theorem head
 %%   {}          % theorem hed spec. (empty = "normal")

%\theoremstyle{boldremark}
%\newtheorem{brem}{T}[section] % remarks are numbered within sections
%Beispiel für eigene Kommandos
\newcommand{\ol}{\overline} %Kurzform für \overline definiert 
\newcommand{\olsi}[1]{\,\overline{\!{#1}}} % overline short italic
\newcommand{\ols}[1]{\mskip.5\thinmuskip\overline{\mskip-.5\thinmuskip {#1} \mskip-.5\thinmuskip}\mskip.5\thinmuskip} % overline short

%\hyphenation{Online-Back-pro-pa-ga-tions-algo-rithmus Online-Backpropagationsalgorithmus}



%\newcommand{\RR}{\mathbb{R}}
%newcommand{\RRn}{\mathbb{R}^n}

\newcommand{\RR}{\ensuremath{\mathbb{R}}}
\newcommand{\Rnv}{\ensuremath{\mathbb{R}^{n}}}
\newcommand{\Rnn}{\ensuremath{\mathbb{R}^{n\times n}}}

\newcommand{\NN}{\ensuremath{\mathbb{N}}}

\newcommand{\mK}{\ensuremath{\mathcal{K}}}
\newcommand{\mKp}{\ensuremath{\mathcal{K}}^+}
\newcommand{\Code}{\ensuremath{C^{\textnormal{dgl}}}}

% Daten für die Titelseite
\institut{Institut für Mathematik} %auskommentieren, wenn nicht benötigt
\arbeit{Masterarbeit} %Bachelorarbeit, Masterarbeit oder Abschlussarbeit (wenn Staatsexamensarbeit geschrieben wird, kann man, z. B. bei Untertitel "Wissenschaftliche Abschlussarbeit\\ im Rahmen  des Ersten Staatsexamens" eintragen)
\autor{Florian Baldauf}
\betreuerGutachter{Prof. Dr. Konrad Engel \newline Universität Rostock\newline Institut für Mathematik} %hier \newline als Zeilenwechsel, da Tabelle mit \parbox im Hintergrund
\gutachter{Prof. Dr. Andreas Heuer\newline Universität Rostock\newline Institut für Informatik}				%dto.
\date{19.08.2021}
\matrNr{217\,203729}
\titel{Datenbankgestützte Erkennung \\ von Mustern in\\
trainierten gefalteten \\neuronalen Netzen (CNN)\\}
\untertitel{Database-driven Recognition of Patterns in trained \\Convolutional Neural Networks (CNN)} %auskommentieren, wenn nicht benötigt

\begin{document}
\hypersetup{pageanchor=false}
\begin{titlepage}
\mytitle   %hier werden Daten für die Titelseite gesetzt
\end{titlepage}

\zusammenfassung{
  Platz für eine kurze Zusammenfassung.\\
}
\pagenumbering{Roman}
\tableofcontents % Inhaltsverzeichnis
\listoffigures
\listoftables
\addcontentsline{toc}{chapter}{Algorithmenverzeichnis}
\listofalgorithms
\lstlistoflistings

% Abkürzungsverzeichnis --------ggfs auskommentieren
%\chapter*{Abkürzungsverzeichnis}
%\addcontentsline{toc}{chapter}{Abkürzungsverzeichnis}
%\begin{acronym}[SEPSEP]
 \acro{rnn}[RNN]{Rekurrentes Neuronales Netz}
\end{acronym} 


% Symbolverzeichnis ------------ggfs auskommentieren
\chapter*{Symbolverzeichnis}
\addcontentsline{toc}{chapter}{Symbolverzeichnis}
\begin{acronym}[SEPSEP]
 \acro{cm}[$\mathcal{C}$]{Confidence Matrix}
\end{acronym}
\cleardoublepage
\hypersetup{pageanchor=true}
\pagenumbering{arabic}

%mainmatter
\chapter{Einleitung}
Es mag euch wundern, dass die Einleitung in einem separaten File abgelegt ist. Dies muss natürlich nicht so sein. Es könnte aber bei einer langen Abschlussarbeit durchaus die Übersichtlichkeit erhöhen, wenn ihr für verschiedene Kapitel einzelne Dateien anlegt und diese mittels 
\begin{verbatim}\input{<DateiName>}\end{verbatim}

\bigskip
oder

\bigskip
\begin{verbatim}\include{<DateiName>}\end{verbatim}
einfügt.

\bigskip
\noindent\textcolor{red}{Verwendet keine Umlaute oder Leerzeichen in Dateinamen.}

\noindent\texttt{input} fügt den Text direkt an die Stelle des \texttt{input}-Befehls ein.

\noindent\texttt{include} fügt den Text auf einer neuen Seite ein. 

\chapter{Grundlagen}

\section{Mathe/ ML Learning}

\section{Relationale Datenbanksysteme}
Relationale Datenbanksysteme gehören zu den erfolgreichsten und verbreitetsten Datenbanken, welche zur elektronischen Datenverwaltung in Computersystemen eingesetzt werden\cite{DBLP:books/daglib/0044627}. In diesem Abschnitt werden wichtige Grundbegriffe relationaler Datenbanksysteme erläutert und erklärt, wie Daten repräsentiert und verarbeitet werden können. Die Notation und Bezeichnungen basieren auf Heuer et al. \cite{DBLP:books/daglib/0044627}. Zum Abfragen und Manipulieren der Daten wird die Datenbanksparche SQL eingeführt und deren theoretische Grundlage im Abschnitt \ref{abs:SQL_intro} beleuchtet. Schließlich werden im Abschnitt \ref{abs:SQL_linalg} Methoden vorgestellt, um Objekte der linearen Algebra als Relationen darzustellen und damit verbundene Operationen, beispielsweise die Matrixvektormultiplikation, umzusetzen.

\subsection{Das Relationenmodell}
Der Grundbaustein relationaler Datenbanksysteme bildet die Relation. Sie stellt eine mathematische Beschreibung einer Tabelle, welche aus Attributen und zugehörigen Domänen besteht, dar. Ein Datenbanksystem ist dann eine Menge dieser Tabellen. 

\begin{defi}[Universum, Attribut, Domäne]
    \label{def:universum}
    Bezeichne die endliche Menge $\mathcal{U} \neq \emptyset$ das Universum. Ein Element $A \in \mathcal{U}$ heißt Attribut. Für $m \in \mathbb{N}$ sei $\mathcal{D}=\{D_1, \ldots, D_m\}$ eine Menge nichtleere Mengen. Ein Element $D_i \in \mathcal{D}$ wird Domäne genannt. Für eine Funktion $\mathrm{dom}: \mathcal{U} \rightarrow \mathcal{D}$ bezeichne $\mathrm{dom}(A)$ den Wertebereich von $A$ und $w \in \mathrm{dom}(A)$ ein Attributwert.
\end{defi}

Nun kann das Relationenschema und zugehörige Begriffe wie Relation und Tupel definiert werden.

\begin{defi}[Relationenschema, Relation, Tupel]
    \label{def:relation}
    Eine Menge $R \subseteq \mathcal{U}$ heißt Relationenschema über dem Universum $\mathcal{U}$. Für $R=\{A_1, \ldots, A_n \}$ ist eine Relation $r$ über $R$, kurz $r(R)$, als eine endliche Menge von Abbildungen
    \begin{equation*}
        t:R \rightarrow \bigcup_{i=1}^m D_i
    \end{equation*}
    definiert. Dabei gilt $t(A) \in \mathrm{dom}(A)$. Die Abbildungen $t$ werden Tupel genannt.
\end{defi}

Vereinfacht gesagt, setzt sich eine Datenbank als Menge von Relationen und ein Datenbankschema als Menge der zugehörigen Relationenschemata zusammen.

\begin{defi}[Datenbank, Datenbankschema, vgl.\cite{DBLP:books/daglib/0044627}]
    Für $p \in \mathbb{N}$ ist eine Menge von Relationenschemata $S=\{R_1, \ldots, R_p\}$ als Datenbankschema definiert. Eine Datenbank $d$ über dem Schema $S$, kurz $d(S)$, ist eine Menge von Relationen
    \begin{equation*}
        d=\{r_1, \ldots, r_p \}
    \end{equation*}
    mit $r_i(R_i)$ für $1 \leq i \leq p$. Eine Relation $r \in d$ wird Basisrelation genannt.
\end{defi}
\subsection{Die Anfragesprache SQL}

\subsection{Lineare Algebra in SQL}
\label{linalgsql}
In diesem Abschnitt wird eine Darstellungsform von Vektoren und Matrizen als Relationen vorgestellt. Weiter werden Ideen zur Umsetzung wichtiger Basisoperationen mit Vektoren und Matrizen in SQL beleuchtet, da diese mathematischen Objekte bei zahlreichen statistischen Analysen eingestzt werden.
\subsection{Matrixdarstellunng}
Im Folgenden wird das \textit{Coordinate scheme} \cite{martendiss} als Schema für die Darstellung  von Matrizen und Vektoren genutzt. Dieses Schema gestaltet sich als einfach und ist daher weit verbreitet \cite{saad1990sparskit}.
Für eine weiterführende Diskussion anderer Darstellungsmöglichkeiten, sei an dieser Stelle auf Marten \cite{martendiss} verwiesen. 
Das Coordinate Scheme beinhaltet 3 Arrays, welche den Zeilenindex, Spaltenindex und den Matrixeintrag als jeweilge Nicht-Null Werte strukturieren. 
\begin{bsp}
    \label{besipiel:_coordinate_sheme}
    Für $x \in \RR^n$ und $A \in \RR^{n \times m}$ ergeben sich die Relationen
    \begin{align*}
        X( &\underline{i} \; \; \mathrm{int}, \\
        &v \; \; \mathrm{double})
    \end{align*}
    für den Vektor $x$ und
    \begin{align*}
        A( &\underline{i} \; \; \mathrm{int}, \\
        &\underline{j} \; \;\mathrm{int},\\
        &v \; \; \mathrm{double})
    \end{align*} für die Matrix $A$.
    Ist
    \begin{equation*}
        A=\begin{pmatrix}
            1 & 2 \\
            -5 & 2 \\
            0 & 7 \\
        \end{pmatrix}
        \in \RR^{3 \times 2}
    \end{equation*}
    gegeben, so ergibt sich Coordinate Schema wie in Abbildung 2.1.
\end{bsp}

\begin{figure}[h]
    \label{coordinate_scheme_table}
    \centering
    \begin{tabular}{ |c|c|c|c|c|c|c| } 
     \hline
     Zeile $i$ &1 &1 &2 &2 &3 &3 \\ 
     \hline
     Spalte $j$ &1 &2 &1 &2 &1 &2 \\ 
     \hline
     Eintrag $a_{i,j}$ &1 &2 &-5 &2 &0 &7 \\ 
     \hline
    \end{tabular}
    \caption{Das Coordinate Schema zur Matrix $A$ aus Beispiel \ref{besipiel:_coordinate_sheme}.}
\end{figure}



\subsection{Basisoperationen}
\label{abs_basisoperationen}
In diesem Abschnitt werden typische Operationen mit Objekten der linearen Algebra beschrieben. Einfache Operationen wie Summation und Multiplikation für reelle Zahlen sind bereits im SQL-Standard enthalten. Seien nun Vektoren $x,y \in \RR^n$ sowie Matrizen $A,B \in \RR^{m \times n}$ und Skalare $r,s \in \RR$ gegeben. Die SQL-Anweisung für die Vektoraddition $rx+sy$ lautet
\begin{align*}
    & \mathbf{select} \; x.i \; \mathbf{as} \; i, \; r*x.v+(s*y.v ) \; \mathbf{as} \; v \\
    & \mathbf{from} \; x \; \mathbf{join} \; y \; \mathbf{on} \; x.i=y.i
\end{align*}
Ähnlich ergibt sich die Matrixaddition $rA+sB$ zu
\begin{align*}
    & \mathbf{select} \; A.i \; \mathbf{as} \; i, \; A.j \; \mathbf{as} \; j, r*A.v+(s*B.v ) \; \mathbf{as} \; v \\
    & \mathbf{from} \; A \; \mathbf{join} \; B \; \mathbf{on} \; A.i=B.i \; \mathbf{and} \; A.j=B.j
\end{align*}
Das Auftreten von \textbf{NULL}-Werten in den Relationen sei hierbei ausgeschlossen. 
Mit der Aggregation \textbf{SUM} können zudem Skalarprodukte und damit Längenbegriffe wie Normen und dadurch induzierte Abstandsbergiffe formuliert werden. Die entsprechenden SQL-Anfragen  sind im Anhang \ref{app:app_1} zu finden. \\
Weitere wichtige Oprationenen stellen die Matrixvektor- und Matrixmatrixmultiplikation dar. 
Durch Kombination vorheriger Basisoperationen ergeben sich entsprechende Transformationenen für die Matrixvektormultiplikation $Ax \in \RR^m$ einer Matrix $A \in \RR^{m \times n}$ und Vektors $x \in \RR^n$ zu

\begin{align*}
    & \mathbf{select} \; A.i \; \mathbf{as} \; i, \; \mathbf{sum} (A.v*x.v) \; \mathbf{as} \; v\\
    & \mathbf{from} \; A \; \mathbf{join} \; x \; \mathbf{on} \; A.j=x.i \; \\
    & \mathbf{group} \; \mathbf{by} \; A.i
\end{align*}

Für die Matrix $C=AB \in \RR^{m \times n}$ als Produkt zweier Matrizen $A \in \RR^{m \times k}$ und $B \in \RR^{k \times n}$ lautet die Anfrage

\begin{align*}
    & \mathbf{select} \; A.i \; \mathbf{as} \; i, \; B.j \; \mathbf{as} \; j, \; \mathbf{sum} (A.v*B.v) \; \mathbf{as} \; v\\
    & \mathbf{from} \; A \; \mathbf{join} \; B \; \mathbf{on} \; A.j=B.i \; \\
    & \mathbf{group} \; \mathbf{by} \; A.i, \, B.j
\end{align*}
Schließlich kann auch die Transponierte $A^T$ einer Matrix $A$ einfach berechnet werden, siehe dazu Anhang \ref{app:app_1}.

Zusammenfassend stellt sich heraus, dass wesentliche Objekte der linearen Algebra und damit verbundene fundamentale Operationen im SQL-Kern umgesetzt werden können.  
Im folgenden Abschnitt werden diese Resultate genutzt und im Zusammenhang mit einem Machine- Learning-Verfahren eingesetzt.
Problemstellung(Einleitung)

\begin{defi}
    \label{def:image}
    Eine Matrix $X \in [0,1]^{h \times b}$ heißt (Grauwert)-Bild mit der Höhe $h$ und Breite $b$. Mit $X_{i,j}$ wird der Grauwert des Pixels $p=(i,j)$ bezeichnet.
\end{defi}


Training, Aufgabe Leistung

supervisies, unsupervised erklären

Klassifikationsproblem

Merkmalsextraktion( 1FFT 2FFT, IFFT NFFT)

(Faltung)

(FFT Regeln insb convolution/coprrelation theorem mit FFT)

Trennbarkeit linear/nichtlinear Entscheidungsgrenzen Hyperebene

Perzeptron Theorem

numerische Minimierung, kurz Abstiegsverfahren in einfacher Version

falls nötig adaptive Verfahren

warum NN?

warum später CNN?


\section*{SQL}
Relationen, Tensoren

Marizen/Vektoren als Relationen

Basisoperationen


\chapter{Grundlagen neuronaler Netze}
\label{kapitel_neuralnetworks}

In diesem Abschnitt werden Künstliche Neuronale Netze\cite{dayhoff1990neural}, kurz KNN, als Forschungsgegenstand der Informatik eingeführt und deren mathematischen Grundlagen präzisiert. 
Sie stellen informationsverarbeitende Systeme nach dem Vorbild von tierischen beziehungsweise menschlichen Gehirnen dar und bestehen aus Neuronen in gewissen Zuständen und Schichten, die über gewichtete Verbindungen miteinander gekoppelt sind. Jene Gewichte sind als freie Parameter des neuronalen Netzes zu verstehen und können während des Trainingsprozesses so angepasst werden, um eine entsprechende Aufgabe zu lösen.  
Gelingt dies, so können neuronale Netze genutzt werden, um bestimmte Muster in Daten, typischerweise in Bildern, Audio oder Stromdaten, effizient zu erkennen\cite{pandya1995pattern, pao1989adaptive, urbaniak2021quality}.
Sie eignen sich daher für viele typischen Aufgaben des maschinellen Lernens, beispielsweise für die Klassifikation digitalisierter Objekte.

Im ersten Abschnitt wird das Perzeptron\cite{rosenblatt1958perceptron} als Grundeinheit eines neuronalen Netztes eingeführt. 
Im folgendem Abschnitt wird das Konzept der Multi-Layer-Perzeptronen\cite{werbos1988generalization} durch die Kopplung mehrerer Perzeptronen mit bestimmten Übertragungs- und Aktivierungsfunktion in einem Netz erläutert. Diese Repräsentierung eines KNN wird im weiteren Verlauf dieser Arbeit genutzt. 

\section{Das Perzeptron}
\label{perzeptron_abs}
Zunächst wird das \textit{Perzeptron} ähnlich wie in Minsky \cite{minsky2017perceptrons} als fundamentaler Baustein eines neuronalen Netzes eingeführt. Das Perzeptron wird oft als Basis moderner KNN angeführt und kann mithilfe des Perzeptron-Lernalgorithmus\cite{rosenblatt1958perceptron} trainiert werden, um das Problem der linearen Trennbarkeit \ref{prob:lin_trenn}von Punktmengen zu lösen.
\begin{defi}[Perzeptron]
    \label{def_neuron}
    Für eine gegebene Funktion $\psi: \RR \rightarrow \RR$, einen Vektor $w \in \Rnv$ und ein Skalar $\theta \in \RR$ wird die Funktion 
    \[ \
    \Psi: \RR^n \rightarrow \RR, \; \; \; x \mapsto \psi(w^T x +\theta)=:y,
    \]
    \textit{Perzeptron} genannt. Mit $x \in \Rnv$ wird die vektorwertige Eingabe und mit $y \in \RR$ die skalare Ausgabe des Perzeptrons bezeichnet. Dabei ist mit $w^Tx=\sum_{i=1}^n w_i x_i$ das Standardskalarprodukt im euklidischen Vektorraum $\Rnv$ gemeint. Die Komponenten von $w$ werden Gewichte und der Skalar $\theta$ Schwellwert oder auch Bias genannt.
\end{defi}
Die Funktionsweise eines Perzeptrons ist in Abbildung \ref{funktionsweise_neuron} dargestellt.
\begin{figure}[h]
    \includegraphics[width=0.8\textwidth]{pics/chapter_neuralnetworks/perzeptron.png}
    \centering
    \caption{Arbeitsweise eines Perzeptrons mit entsprechender Notation aus Definition \ref{def_neuron}.}
    \label{funktionsweise_neuron}
\end{figure}
Bei der Wahl der Funktion $\psi$ gibt es mehrere Möglichkeiten. Wird wie in Minsky\cite{minsky2017perceptrons} die Heavyside-Funktion
\begin{equation*}
    \psi: \RR \rightarrow \RR, \; \; \;
    \psi(x)=\begin{cases}
       1 &, x \geq 0 \\
       0 &, \text{sonst}
    \end{cases}
\end{equation*} 
genutzt, kann das Perzeptron als binärer Klassifikator wie in \ref{abs_linear_trenn} interpretiert werden. Dabei dient $w^Tx+\theta=0$ als trennende Hyperebene. Ist $w^Tx+\theta<0$, so ist $\psi(x)=0$ und $x$ wird der Klasse $K_{-1}$ zugeordnet. Gilt jedoch $w^Tx+\theta \geq 0$ und damit $\psi(x)=1$, so ist der Vektor $x$ der Klasse $K_1$ zugehörig. 

Für ein Klassifikationsproblem, bei dem die Klassen nicht linear trennbar sind, scheitern diese einfachen Perzeptronen. Hier wird oft das zweidimensionale XOR-Problem angeführt, bei denen die Punktmengen $P_{-1}=\{(0,0),(1,1)\}$ und $P_{1}=\{(1,0),(0,1)\}$ getrennt werden sollen. Um solche Aufgaben zu lösen, ist es notwendig, mehrere Perzeptronen geschickt zu verknüpfen, um komplexe Entscheidungsgrenzen zu erhalten.

\section{Multi-Layer-Perzeptron}
\label{MLP_abs}
In dieser Arbeit wird ein Künstliches Neuronales Netz als eine Menge von Perzeptronen, die in gewissen Schichten partitioniert und miteinander verbunden sind, notiert. Diese sogenannten \textit{Multi-Layer-Perzeptronen}, kurz MLP,  gelten als erste tiefe neuronale Netze und sind seit den späten 1980er Gegenstand der Forschung\cite{bourlard1990links,bounds1988multilayer,MLPbook}. Zunächst sind einige Definition notwendig, um eine lesbare Notation des MLPs zu geben.

\begin{defi}[Übertragungsfunktion]
    \label{def_net}
    Für eine gegebene Matrix $W \in \RR^{n \times m}$ und einen Vektor $b \in \RR^m$ ist 
    \[ 
    \Psi^{W,b}: \RR^n \rightarrow \RR^m, \; \; \; x \mapsto W^T x +b
    \]
    als Übertragungsfunktion definiert. Der Vektor $y=\Psi^{W,b}(x) \in \RR^m $ wird als Netzeingabe bezeichnet.
\end{defi}
Hierbei ist $W$ eine Gewichtsmatrix und $b$ ein Biasvektor, welche als freie Parameter fungieren und die Netzeingabe eines Eingabevektors $x \in \RR^n$ auf lineare Art und Weise beeinflussen. Um auch nichtlineare Zusammenhänge darzustellen, werden Aktivierungsfunktionen benutzt.

\begin{defi}[Aktivierungsfunktion]
    \label{def_act_f}
    Eine stetige, monoton steigende und nicht notwendigerweise lineare Funktion $\psi: \RR \rightarrow \RR$ wird als Aktivierungsfunktion bezeichnet.
\end{defi}
Es sei erwähnt, dass auch nicht monotone Aktivierungsfunktionen genutzt werden können, beispielsweise radiale Basisfunktionen\cite{radialbasis}, welche jedoch in dieser Arbeit nicht weiter von Interesse sind.
Typische Aktivierungsfunktionen, die oft verwendet werden, sind die:
\begin{align*}
    \text{Identität}: \; \;\psi(x)&=x, \\
    \text{Logistische Funktion}: \; \;\psi(x)&=\frac{1}{1+\mathrm{e}^{-x}}, \\
    \text{Tangens Hyperbolicus}: \; \;\psi(x)&=\tanh(x), \\
    \text{ReLU (rectified linear unit)}: \; \;\psi(x)&=\max\{0,x\}.
\end{align*}

\begin{bem}
    Ist $\psi$ eine Aktivierungsfunktion, so wird für $x \in \RR^n$ mit 
    \[\psi(x):=\left(\psi(x_1), \ldots, \psi(x_n)\right)^T \in \RR^n
    \]
    der Vektor bezeichnet, welcher sich durch die elementweise Auswertung der Aktivierungsfunktion $\psi$ an den Stellen $x_1, \ldots, x_n$ ergibt. 
\end{bem}

Bei Klassifikationsproblemen wird oft die \textit{Softmax}-Funktion\cite{denker1990transforming} genutzt, welche die gesamte Eingabe berücksichtigt. Im Abschnitt \ref{task_training} wird erläutert, warum sich in diesem Fall die Softmax-Funktion eignet.

\begin{defi}[Softmax]
    Für $x \in \RR^n$ wird die Funktion $\psi: \RR^n \rightarrow (0,1]^n$ mit 
    \[
        \psi(x):=\left(\frac{\mathrm{e}^{x_1}}{\sum_{i=1}^n \mathrm{e}^{x_i}}, \ldots,\frac{\mathrm{e}^{x_n}}{\sum_{i=1}^n \mathrm{e}^{x_i}} \right)^T
    \]
    als Softmax-Funktion definiert. Die Einträge des Vektors $\psi(x)$ summieren sich zu Eins.  
\end{defi}

Für den späteren Trainingsprozess ist es nützlich, die Ableitung der verwendeten Aktivierungsfunktion, sofern sie exisitiert, zur Verfügung zu haben. Zudem ist es möglich, für bestimmte Aktivierungsfunktionen die Ableitung nur mithilfe der verwendeten Funktion zu berechnen.

\begin{lem}
    \begin{itemize}
        \item[(i)] Für die ReLU $\psi(x)=\max\{0,x\}$ gilt
         \[\psi'(x)=\begin{cases}
            0 &, x <0 \\
            1 &, x >0
        \end{cases}. 
        \]
        An der Stelle 0 ist die Ableitung nicht definiert und wird oft mit $\psi'(0)=\frac{1}{2}$ festgelegt.
        \item[(ii)] Für die logistische Funktion $\psi(x)=\frac{1}{1+\mathrm{e}^{-x}}$ gilt
        \[ 
            \psi'(x)=\psi(x)(1-\psi(x)) 
        \]
        für alle $x \in \RR$.
        \item[(iii)] Für den Tangens Hyperbolicus $\psi(x)=\tanh(x)$ gilt
        \[ 
            \psi'(x)=1-\psi^2(x) 
        \]
        für alle $x \in \RR$.
    \end{itemize}
\end{lem}
\begin{proof}
    Einfaches Differenzieren liefert für $(i)$ und $(ii)$ die Resultate. Bei $(iii)$ wird die Darstellung $\tanh(x)=\frac{2}{\mathrm{e}^{2x}+1}$ genutzt und das Differenzieren mittels Quotientenregel liefert die Aussage.
\end{proof}


Ähnlich der Definition des Perzeptron \ref{def_neuron} wird nun eine Schicht als Verknüpfung von Übertragungsfunktion und Aktivierungsfunktion definiert.

\begin{defi}[Neuronenschicht]
    Ist $\Psi^{W,b}$ eine Übertragungsfunktion mit den Parametern $W \in \RR^{n \times m}, b \in \RR^m$ und $\psi$ eine Aktivierungsfunktion, so wird das Paar $(\Psi^{W,b}, \psi)$ als Schicht $\mathcal{S}$ bezeichnet. Für eine Eingabe $x \in \RR^n$ ist die Ausgabe $y \in \RR^m$ der Schicht $\mathcal{S}$ durch
    \[y=\psi \circ \Psi^{W,b}(x)= \psi\left(\Psi^{W,b}(x)\right)
        \] 
        gegeben. Die Komponenten $y_i$ werden für $1 \leq i \leq m$ Neuronen der Schicht $\mathcal{S}$ genannt und gleichen jeweils der Ausgabe eines einfachen Perzeptrons wie in Definition \ref{def_neuron}. Eine Schicht besteht aus $m$ Perzeptronen $\tilde{\Psi}_i$ mit $y_i=\tilde{\Psi}(x_i)=\psi(W_{i,:}^T x+b_i)$ für $1 \leq i \leq m$.
\end{defi}
Im Hinblick auf MLPs werden nun mehrere Schichten so verbunden, dass die Ausgabe einer Schicht $\mathcal{S}_k$ als Eingabe einer darüberliegenden Schicht $\mathcal{S}_{k+1}$ für ein $k \in \mathbb{N}$ dient. Die Anzahl der Neuronen kann dabei von Schicht zu Schicht variieren. Dementsprechend werden die Dimensionen der beteiligten Gewichtsmatrizen $W^{(k)}$ und Biasvektoren $b^{(k)}$ passend gewählt. 
Um die Notation übersichtlich zu halten, bezeichne $\Psi^{W^{(k)},b^{(k)},\psi_{k}}$ die Schicht $\mathcal{S}_k$ mit $\Psi^{W^{(k)},b^{(k)},\psi_{k}}(x):= \psi_{k} \left(\Psi^{W^{(k)},b^{(k)}}(x)\right)$.

\begin{defi}[Multi-Layer-Perzeptron, vgl. gruening]
    Für eine gegebene Anzahl $l \in \mathbb{N}, \; l>1$ von Schichten $\Psi^{W^{(1)},b^{(1)},\psi_{1}}, \ldots, \Psi^{W^{(l)},b^{(l)},\psi_{l}}$ bezeichne $s_l \in \mathbb{N}$ die Anzahl der Neuronen in Schicht $l$. Für eine Eingabe $x \in \RR^{s_0}$ lässt sich die Ausgabe $y \in \RR^{s_l}$ eines Multi-Layer-Perzeptron  $\Lambda_l: \RR^{s_0} \rightarrow \RR^{s_l}, \; x \mapsto y$ mit $l$ Schichten durch
    \[
        y=\Psi^{W^{(l)},b^{(l)},\psi_{l}} \circ \ldots \circ \Psi^{W^{(1)},b^{(1)},\psi_{1}}(x)
    \]
    berechnen. Dabei gelten für die Gewichtsmatrizen die Dimensionsbedingungen
    \[{}_1W^{(1)}=s_0, \; \; {}_2W^{(l)}=s_l, \; \; \forall i \in [l-1]: \; {}_2W^{(i)}={}_1W^{(i+1)}.
        \] 
    Die Eingabeschicht $\mathcal{S}_0$ besitzt keine Parameter $W$ und $b$ und besteht nur aus dem Eingabevektor $x \in \RR^{s_0}$. Die letzte Schicht $\Psi^{W^{(l)},b^{(l)},\psi_{l}}$ wird als Ausgabeschicht bezeichnet. Weiter werden die Schichten $\mathcal{S}_1, \ldots, \mathcal{S}_{l-1}$ als verdeckte Schichten definiert. Die Funktionsauswertung $\Lambda_l(x)$ für eine Eingabe $x$ wird Vorwärtsrechnung, engl. \textit{forward propagation}, genannt.
\end{defi}

\begin{algorithm}
    \caption{Vorwärtsrechnung}\label{alg:ff}
    \begin{algorithmic}
    \Require $ \text{MLP} \; \Lambda_l, \text{Eingabe} \; x_0 \in \RR^n$
    \Ensure $y = \Lambda_l(x) \in \RR^m$
    \State $x=x_0$
    \For{$i=1, \ldots l}$
    \State $u=W^{(i) \,T}x+b^{(i)}$
    \State $x=\psi_i(u)$
    \EndFor
    \State $y=x$
    \end{algorithmic}
\end{algorithm}
    


Das MLP-Modell wird im weiteren Verlauf dieser Arbeit repräsentativ als Künstliches Neuronales Netz bezeichnet. Die Funktionsauswertung eines KNN wird im Algorithmus Vorwärtsrechnung \ref{alg:ff} festgehalten. Das zuvor angesprochene XOR-Problem kann nun beispielsweise mithilfe eines KNN bestehend aus zwei Schichten gelöst werden\cite{Goodfellow-et-al-2016}.
Es lassen sich zwischen Modell- und Hyperparameter von KNN unterscheiden.

\begin{defi}[Hyper- und Modellparameter]
    Sei für $l \in \mathbb{N}$ ein KNN $\Lambda_l$ gegeben. Dann werden die Eingabe- und Ausgabedimension $s_0, s_l$, die Anzahl $l$ der (verdeckten) Schichten sowie die verwendeten Aktivierungsfunktion $\psi_l$ Hyperparameter des neuronalen Netzes genannt.
    Die Gewichtsmatrizen und Biasvektoren mit den entsprechend passenden Abmessungen stellen die Modellparameter $\mathcal{W}:=\{(W^{(i)},b^{(i)}): \; i=1, \ldots, l\}$ des neuronales Netzes dar. 
\end{defi}
Die Hyperparameter werden oft anwendungsspezifisch für das jeweilige Problem gewählt, während die Modellparameter dynamisch in einem Trainingsprozes angepasst werden, sodass die gegebene Aufgabe zufriedenstellend gelöst wird. Wie sies geschieht, wird im folgenden Abschnitt \ref{task_training} erläutert.

\section{Training neuronaler Netze}
\label{task_training}
Künstliche Neuronale Netze gehören zu den typischen Vertretern von maschinellen Lernalgorithmen, welche hinsichtlich einer bestimmten Aufgabe, engl. \textit{task T}, und einem Leistungsmaß, engl. \textit{perfomance P} an der Erfahrung, engl. \textit{experience E} lernen\cite{Goodfellow-et-al-2016}. Dabei ist mit Lernen gemeint, dass das Computerprogramm bezüglich der Aufgabe $T$ sein Leistungsmaß $P$ mit wachsener Erfahrung $E$ schrittweise steigert. Wie in Kapitel \ref{fund} erläutert, gibt es viele verschiedene Aufgaben, wie die Regression, Klassifikation oder Clusterung bestimmter Objekte. 

In den folgenden Abschnitten wird das Klassifikationsproblem asl \textit{task T} im Mittelpunkt stehen. Weiter werden KNNs als Modellschätzer aus der Wahrscheinlichkeitstheorie interpretiert und fundamentale Aussagen wie das \textit{Universal-Approximation-Theroem}\cite{HORNIK1989359} gegeben. Schließlich wird bezüglich der Klassifikationsaufgabe das Training neuronaler Netze erläutert.


\subsection{Neuronale Netze als universelle Schätzer}
\label{NN_estimators_abs}
Beim Klassifikationsproblem müssen bestimmte bedingte Wahrscheinlichkeiten, die in diesem Abschnitt erklärt werden, ermittelt werden. Oft wird dazu die Ausgabeschicht eines KNN als Wahrscheinlichkeit interpretiert und daher KNN als Schätzer der bedingten Wahrscheinlichkeiten eingesetzt. Zunächst werden Klassifikationsfunktion und -problem definiert.
\begin{defi}
    \label{def_classfun}
    Seien die Mengen $D \subset \RR^n$ und $\mathcal{C}=\{c_1, \ldots, c_m\}$ gegeben. Eine Funktion $f: D \rightarrow \mathcal{C}$, welche ein Element aus $D$ einer Klasse $c_i \in \mathcal{C}$ zuordnet, wird Klassifikationsfunktion genannt. Hier gibt es $m \in \mathbb{N}$ verschiedene Klassenlabels. Die Menge $D$ wird abstrakter Datensatz genannt.
\end{defi}

Das Ziel beim Klassifikationsproblem ist die Approximation einer nicht bekannten Klassifikationsfunktion $f:D \rightarrow \mathcal{C}$ durch ein Modell $\tilde{f}: D \rightarrow \mathcal{C}$. 
In dieser Arbeit werden dafür KNNs genutzt, welche als probabilistische Modelle auf folgende Weise genutzt werden. Auf der Ergebnismenge $\Omega= D \times \mathcal{C}$ sei die nicht bekannte gemeinsame (Wahrscheinlichkeits-) Verteilung $p_{Daten}(x,c)$, genannt Datenverteilung, gegeben. Ein Modell soll nun konstruiert werden, welches die a posterior-Verteilung $p_{Daten}(\cdot \; | \; x)$ der Klassen schätzt. 

In dieser Arbeit werden KNN so benutzt, dass die Klassenzugehörigkeit direkt anhand der Eingabe $x \in D$ geschätzt wird. Die Funktion $P_{Daten}: D \rightarrow [0,1]^m$ mit
\begin{equation}
    \label{eq:Pdaten}
    P_{Daten}(x):=\left(p_{Daten}(c_1 \; | \; x), \ldots, p_{Daten}(c_m \; | \; x)\right)^T \in \RR^m
\end{equation} soll für alle $x \in D$ approximiert werden. Dazu wird die Funktion $P_{Modell}: D \rightarrow [0,1]^m$ mit 
\begin{equation}
    \label{eq:Pmodell}
    P_{Modell}(x;\mathcal{W}):=\left(p_{Modell}(c_1 \; | \; x; \mathcal{W}), \ldots, p_{Modell}(c_m \; | \; x; \mathcal{W})\right)^T \in \RR^m
\end{equation}
 für alle $x \in D$  genutzt, welche von den Modellparametern $\mathcal{W}$ abhängig ist. Die Klassifikationsfunktion des Modells ergibt sich als
\begin{equation}
    \label{eq:f_modell}
    f_{Modell}(x):= \underset{c \in \mathcal{C}}{\mathrm{argmax}} \; p_{Modell}(c \; | \; x).
\end{equation}

Es stellt sich die Frage, inwiefern das MLP als Modell genutzt werden kann, um beliebige Datenverteilungen $P_{Daten}$ zu approximieren. Folgende Resultate liefern die Antwort.

\begin{satz}[Universal-Approximation-Theroem\cite{gruen}]
    \label{UAT}
    Sei $\psi_1$ eine nichtkonstante, beschränkkte Aktivierungsfunktion und $id: \RR \rightarrow \RR$ die Identität sowie $D \subset \RR^n$ kompakt. Dann existieren für alle $\varepsilon >0$ und stetigen Funktionen $f: D \rightarrow \RR$ Parameter $N \in \mathbb{N}, W^{(1)} \in \RR^{n \times N}, b^{(1)} \in \RR^N$ sowie $W^{(2)} \in \RR^{N \times 1}$, sodass
    \begin{equation}
        \label{UAP_eq}
        \left|f(x)-\Psi^{W^{(2)},0,id} \circ \Psi^{W^{(1)},b^{(1)},\psi_1}(x)\right| < \varepsilon, \; \; \forall x \in D
    \end{equation}
    gilt.
\end{satz}

\begin{proof}
    Ein Beweis kann in Hornik\cite{hornik1991approximation} nachgelesen werden.
\end{proof}

Das Universal-Approximation-Theroem kann ebenfalls auf unbeschränkte und nichtkonstante Funktion $f:D \rightarrow \RR^m$ erweitert werden. Heutzutage wird oft die ReLU-Funktion als Aktivierungsfunktion verwendet\cite{schmidt2020nonparametric,li2017convergence}.

\begin{kor}
    Mit den gleichen Vorraussetzungen wie in Satz \ref{UAT} gilt die Abschätzung \ref{UAP_eq} für $\psi_1(x)=\max\{0,x\}$.
\end{kor}

\begin{proof}
    Siehe Sonoda et. al.\cite{sonoda2017neural}.
\end{proof}

Hinsichtlich der Approximation von beliebigen Funktionen $P_{Daten}$ mithilfe eines neuronalen Netzes mit der Softmax-Funktion als Aktivierungsfunktion liefert Strauß\cite{strauss} folgendes Resultat.

\begin{kor}
    \label{kor_softmax}
    Ein MLP mit zwei Schichten, wobei $\psi_2$ die Softmax-Funktion ist, kann genutzt werden, um stetige Funktionen $f:K \rightarrow [0,1]^m$, welche von einem Kompaktum $K \subset \RR^n$ in eine (Wahrscheinlichkeits)-Verteilung über die Klassen $\mathcal{C}$ abbilden, beliebig genau zu approximieren.
\end{kor}

\begin{proof}
    Siehe \cite{strauss}.
\end{proof}

Die Aussage kann auf das MLP mit beliebig vielen Schichten erweitert werden.
In dieser Arbeit umfasst die Menge $D$ aus Definition \ref{def_classfun} digitalisierte Objekte und ist endlich und damit kompakt. Daher kann wegen Korollar \ref{kor_softmax} das MLP als Modell genutzt werden, um stetige Funktionen $P_{Daten}$ sinnvoll zu approximieren.

\subsection{Optimale Parameterwahl bei neuronalen Netzen}

Wird ein künstliches neuronales Netz als probabilistisches Modell genutzt und sind die Hyperparameter festgelegt, müssen die Modellparameter $\mathcal{W}$ gewählt werden. Um die Approximationsgüte, also die \textit{perfomance P}, bezüglich des Klassifikationsproblems messbar zu machen, werden Fehlerfunktionen eingeführt. 
Mit Trainingsdaten als \textit{experience E} und dem Gradientenverfahren\cite{nocedal1999numerical} sollen optimale Parameter $\mathcal{W}$ gefunden werden, sodass die gewählte Fehlerfunktion minimiert wird. Im folgenden sei $\Lambda_l$ ein KNN mit der Softmax-Funktion als Aktivierungsfunktion in der Ausgabeschicht, welches als parametrisiertes Modell $f_{Modell}$ wie in \ref{eq:f_modell} genutzt wird.

\begin{defi}[Trainingsmenge]
    Seien $p_{daten}$ eine Datenverteilung und $\Omega=D \times \mathcal{C}$. Dann heißt für ein $k \in \mathbb{N}$ die Menge
    \begin{equation*}
        \mathcal{T}:=\left\{ (x^{(i)},c^{(i)}) \; | \; i \in [k] \right\} \subset \Omega
    \end{equation*}
    Trainingsmenge bestehend aus $k$ Trainingsdaten, welche unabhängig durch $p_{Daten}$ generiert wurde.
\end{defi}

Die Güte des Modells $f_{Modell}$ wird als Likelihood gegeben einer Trainingsmenge $\mathcal{T}$ gemessen und lässt sich als 

\begin{equation}
    \label{eq:likelihood}
    L(\mathcal{T},\mathcal{W}):=\prod_{(x,c) \in \mathcal{T}} p_{Modell}(c \; | \; x; \mathcal{W})
\end{equation}
wie in Bishop\cite{bishop2006pattern} berechnen.
Für eine Trainingsmenge $\mathcal{T}$ soll das Produkt über alle Wahrscheinlichkeiten der korrekten Klassenzugehörigkeiten $c$ gegeben der Eingaben $x$ maximiert werden. Dieser Ansatz wird \textit{Maximum Likelihood-Methode}\cite{ruschendorf2014mathematische} genannt und eine Parameterwahl ist durch eine Lösung des Optimierungsproblems 
\begin{equation}
    \label{eq:opt_likelihood}
     \prod_{(x,c) \in \mathcal{T}} p_{Modell}(c \; | \; x; \mathcal{W}) \rightarrow \max
\end{equation}
gegeben. Dabei sei bemerkt, dass die Optimierung unabhängig von den Hyperparametern vorgenommen wird.

Für ein Trainingspaar $(x,c) \in \mathcal{T}$ bezeichne $t(x,c) \in \RR^m$ den Zielvektor der Klasse $c$ mit sogenannter (1 aus m)-Kodierung. Die Komponenten des Zielvektors sind
\begin{equation*}
    t_k(x,c):= \begin{cases}
        1 &, \text{wenn} \; k=c \\
        0 &, \text{sonst}
    \end{cases}, \; \; \forall k \in [m].
\end{equation*} 
Mit dieser Bezeichnung lässt sich das Optimierungsproblem \ref{eq:opt_likelihood} als Minimierungproblem mithilfe der \textit{negative log likelihood} schreiben.
\begin{defi}[negative log likelihood]
    Seien die Mengen $D$ und  $\mathcal{C}=\{c_1, \ldots, c_m\}$ mit einer dazugehörigen Trainingsmenge $\mathcal{T}$ sowie entsprechende Zielvektoren gegeben. Weiter seien die a posterior Wahrscheinlichkeiten $p_{Modell}(c \; | \; x; \mathcal{W})$ wie in Gleichung \ref{eq:Pmodell} gegeben. Die negative log likelihood ist als Funktion 
    \begin{equation}
        \label{eq:NLL}
        L_{NNL}(\mathcal{T},\mathcal{W}):= -\sum_{(x,c) \in \mathcal{T}}  \sum_{i=1}^m t_i(x,c) \log \left(p_{Modell}(c_i \; | \; x; \mathcal{W}) \right) 
    \end{equation}
    definiert.
\end{defi}

Das Minimieren der negative log likelihood ist äquivalent zur Maximierung der Likelihood aus \ref{eq:likelihood}, denn es gilt 
\begin{equation*}
    \log \left(\prod_{(x,c) \in \mathcal{T}} p_{Modell}(c \; | \; x; \mathcal{W})\right)= \sum_{(x,c) \in \mathcal{T}} \log \left(p_{Modell}(c \; | \; x; \mathcal{W}) \right)
\end{equation*}
und der natürliche Logarithmus ist monoton steigend. Wird zusätzlich angenommen, dass die Datenverteilung einer Normalverteilung mit konstanter Varianz entspricht, so ist das Maximieren von \ref{eq:likelihood} äquivalent zur Minimierung der mittleren quadratischen Abweichung
\begin{equation*}
    \label{eq:MSE}
    L_{MSE}(\mathcal{T},\mathcal{W}):=\frac{1}{2} \sum_{(x,c) \in \mathcal{T}} ||\hat{c}-t(x,c)||_2^2,
\end{equation*}
wobei $\hat{c}=f_{Modell}(x)$ und $t(x,c)$ der Zielvektor des Datenpaars $(x,c)$ ist, siehe Goodfellow\cite{Goodfellow-et-al-2016}.

 Das Problem \ref{eq:opt_likelihood} wird nun allgmemein mit Fehlerfunktionen definiert.
 \begin{defi}
    Seien $\mathcal{T}$ eine Trainingsmenge und $\mathcal{W}$ Modellparameter eines KNN. Mithilfe des Gradientenverfahrens soll das Problem
    \begin{equation}
        \label{eq:error_fun_opt}
        \mathcal{E}(\mathcal{T},\mathcal{W}) \rightarrow \min
    \end{equation}
    gelöst werden. Dabei wird $\mathcal{E}$ Fehlerfunktion gennant.
 \end{defi}
In dieser Arbeit wird $\mathcal{E}$ immer als stückweise stetig differenzierbare Funktion gewählt, damit das Gradientenverfahren angewendet werden kann. Sowohl die negative log likelihood $L_{NNL}$ als auch die mittlere quadratische Abweichung $L_{MSE}$ sind als Fehlerfunktion geeignet. Die Optimierung der Parameter geschieht iterativ und besteht jeweils aus zwei Schritten. Zuerst wird eine Abstiegsrichtung 
\begin{equation}
    \label{eq:gradE}
    \Delta_n :=\nabla_{\mathcal{W}} \mathcal{E}(\mathcal{T},\mathcal{W})
\end{equation} 
berechnet und dann die Parameter 
\begin{equation}
    \label{eq:step}
    \mathcal{W}_{n+1}:=\mathcal{W}_n- \lambda \Delta_n
\end{equation}
aktualisiert. Es werden also Gradienten der Fehlerfunktion bezüglich der Gewichtsmatrizen und Biasvektoren ermittelt und anschließend werden jene Parameter mit einer Lernrate $\lambda \in \RR$ angepasst. Hier wird der Gradient über alle Trainingspaare berechnet. Diese Variante nennt sich \textit{Offline-Version} des Gradientenverfahrens und ist besonders für große Trainingsmengen ineffizient. Die \textit{Online-Version} berechnet den Gradienten lediglich für ein Trainingspaar und passt die Parameter direkt an. In dieser Arbeit wird ein Kompromis aus beiden Verfahren verwendet und zwar das \textit{Mini-Batch-Verfahren} \ref{alg:minibatch}, bei dem die Gradienten über kleine Teilmengen der Trainingsmenge $\mathbb{T} \subset \mathcal{T}$ berechnet werden. Für eine tiefere Analyse des Gradientenverfahrens sei auf Ruder\cite{ruder2016overview} verwiesen. 

\begin{algorithm}
    \caption{Mini-Batch-Verfahren}\label{alg:minibatch}
    \begin{algorithmic}
    \Require $ \text{Trainingsmenge} \; \mathcal{T}, \text{Modellparameter} \; \mathcal{W}, \text{Fehlerfunktion} \; \mathcal{E}, \text{Batch-Größe} \; n$
    \Ensure $\text{optimierte Modellparamter} \; \mathcal{W}$
    \State $x=x_0$
    \For{$i=1, \ldots l}$
    \State $u=W^{(i) \,T}x+b^{(i)}$
    \State $x=\psi_i(u)$
    \EndFor
    \State $y=x$
    \end{algorithmic}
\end{algorithm}
    






\chapter{Gefaltete neuronale Netze}
\label{kap:CNN}

Feed-Forward-Netze gelten als leistungsstarke maschinelle Lernmethoden, da sie so trainiert werden können, um beliebige komplexe Funktionen abhängig von einer vektorwertigen Eingabe zu approximieren. Ist die Dimension der Eingabeschicht jedoch zu groß, treten bei klassischen FFN Probleme hinsichtlich der Paramteranzahl auf. Im weiteren Verlauf dieser Arbeit sollen digitalisierte Bilder klassifiziert werden. Wird ein MLP mit $100$ Ausgabeneuronen genutzt und jeder Pixel eines Bildes mit den Abmessungen $1000 \times 1000$ als Merkmal genutzt, so ergeben sich bereits $10^8+100$ freie Parameter. Stehen nur relativ wenige Traingsdaten zur Verfügung, ist die Struktur des FFN zu komplex und dies kann zur Überanpassung führen\cite{caruana2000overfitting,bilbao2017overfitting}. Die Parameteranzahl muss also deutlich reduziert werden. Konzepte wie \textit{Parameter Sharing} und spärliche Konnektivität, engl. \textit{sparse connectivity} erlauben diese Reduktion, vgl. Goodfellow\cite{Goodfellow-et-al-2016} und werden in den folgenden Abschnitten erläutert.

Ein weiterer Nachteil des FFN ergibt sich dadurch, dass Korrelationen von benachbarten Eingabeneuronen, z.B. Bilsegmente wie Kanten oder Ecken, nicht miteinbezogen werden. Es muss also eine Modell entwickelt werden, welches diese lokalen Muster extrahiert und sie miteinander verknüpft. Das Modell sollte zudem äquivariant gegenüber Translationen sein. 

%Schließlich sind beim FFN die Eingabe- und Ausgabedimension fixiert. Eine flexible Wahl diser Hyperparameter ist bei Problemen der Computergrafik oft erwünscht. 

In diesem Kapitel wird erläutert, wie gefaltete neuronale Netze die erwähnten Nachteile von FFN umgehen. CNN sind in der Lage, lokale Muster zu erkennen, sind äquivariant gegenüber Translationen und realisieren Konzepte wie Parameter Sharing, um die Anzahl der freien Parameter drastisch zu reduzieren. So gelingt es, besonders bei Aufgaben der Computergrafik\cite{DBLP:conf/nips/KrizhevskySH12, DBLP:journals/pieee/LeCunBBH98,DBLP:conf/cvpr/CiresanMS12} die Generalisierungsrate gegenüber klassichen FFN zu erhöhen. 

Gefaltete neuronale Netze unterscheiden sich von FFN bei der Berechnung der Übertragungsfunktion. Dazu wird die gefaltete Übertragungsfunktion definiert, welche das Konzept der diskreten Faltung nutzt. Im folgenden Abschnitt \ref{abs:conv_theorie}wird zunächst die Faltung als mathematische Operation eingeführt und deren Zusammenhang zur Fourier-Transformation\cite{werner2011funktionalanalysis} erläutert. Anschließend wird im Abschnitt \ref{abs:conv_def} das CNN-Modell definiert. 

%bei CNNs zu verstehen ist und wie diese Operation bei diesen neuronalen Netzen motiviert wird. In diesem Zusammenhang werden Begriffe wie Merkmalskarten (engl. \textit{feature maps}) und Filter (engl. \textit{kernels}) eingeführt. Des Weiteren wird die Arithmetik der Faltungsoperation für zweidimensionale Eingaben, repräsentiert durch Matrizen, erklärt und das Verfahren \textit{padding} beziehungsweise das Nutzen von \textit{strides} erläutert. Das gesamte Kapitel wird mit konkreten Beispielen begleitet, um die verschiedenen Effekte der Faltungsoperation zu beleuchten.

\section{Die Faltungsoperation}
\label{abs:conv_theorie}
In der Analysis ist die Faltung ein mathematischer Operator und liefert für zwei Funktionen $f$ und $g$ die Funktion $ f \ast g$, wobei mit dem Sternchen die Faltungsoperation gemeint ist.

\begin{defi}[Faltung]\label{allg_faltung}
    Für zwei Funktionen $f,g: \Rnv \rightarrow \mathbb{C}$ ist die Faltung als
    \begin{equation*}
        (f \ast g) (x) := \int_{\Rnv} f(\tau) g(x-\tau) \mathrm{d} \tau
    \end{equation*}
    definiert, wobei gefordert wird, dass das Integral für fast alle $x$ wohldefiniert ist. Für $f,g \in L^1(\RR^n)$ ist dies der Fall.
   \end{defi}

Für die Faltung gelten einige Rechenregeln.

\begin{lem}
    \label{lem:convrules}
    Seien $f,g,h \in L^1(\RR^n)$ und $a \in \mathbb{C}$. Dann gelten
    \begin{align*}
         (i) \; \; &f \ast g = g \ast f \; \; &( \text{Kommutativität}) \\
         (ii) \; \; &f \ast (g \ast h) = (f \ast g) \ast h  \; \;& (\text{Assoziativität}) \\
         (iii) \; \; &f \ast (g+h) = (f+g) \ast h \; \; &(\text{Distributivität}) \\ 
         (iv) \; \; &a(f \ast g) = (af) \ast g = f \ast (ag) \; \; &(\text{Assoziativität mit skalarer Multiplikation}) 
    \end{align*}
\end{lem}

\begin{proof}
    Eine Beweis dieser Rechenregeln kann in Werner\cite{werner2011funktionalanalysis} nachgelsen werden.
\end{proof}

%Bei klassischen neuronalen Netzen, siehe Kapitel \ref{classicNN} werden Eingabedaten durch eine Verkettung von affinen Transformationen verarbeitet. Typischerweise wird die Eingabe als Vektor dargestellt und  mit einer Matrix multipliziert, gegebenenfalls mit einem Biasvektor manipuliert und schließlich so die Ausgabe generiert. Bilder-, Audio- oder Videoaufnahmen besitzen jedoch mehrere Merkmale in unterschiedlichen Achsen. Oft sind solche Eingabedaten im Bereich des Machine-Learnings als mehrdimensionale Arrays abgelegt, welche eine oder mehrere Achsen repräsentieren, wobei die Ordnung dieser eine Rolle spielt. Bei digitalisierten Bildern sind das bespielsweise die Höhe und Breite des Bildes, bei Audioaufnahmen gibt es nur eine Achse, und zwar die Zeitachse. Hinzu kommen Kanalachsen als weitere Verfeinerung der Daten, zum Beispiel besitzen RGB-Farbbilder drei Kanäle der Farben rot, grün und blau. 

%Diese speziellen Eigenschaften können bei affinen Transformationen nicht berücksichtigt werden. Alle Merkmale sowie Achsen werden gewissermaßen gleich behandelt und die wesentliche topologische Struktur kann so nicht zum Vorteil ausgenutzt werden. Hier soll nun die sogenannte diskrete Faltung Abhilfe schaffen.
In der digitalen Signal- und Bildverarbeitung werden meist diskrete Funktionen analysiert und daher die diskrete Faltung genutzt, bei der statt der Intgration eine Summation auftaucht. Die Regeln aus Lemma \ref{lem:convrules} gelten analog.

\begin{defi}[Diskrete Faltung]\label{disk_faltung}
    Für zwei Funktionen $f,g: D \rightarrow \mathbb{C}$ mit einem diskreten Definitionsbereich $D \subseteq \mathbb{Z}^n$ ist die diskrete Faltung als
    \begin{equation*}
        (f \ast g) (n) := \sum_{k \in D} f(k) g(x-k)
    \end{equation*}
    definiert. Hier wird über dem gesamten Definitionsbereich $D$ summiert. Ist $D$ beschränkt, werden $f$ beziehungsweise $g$ durch Nullen fortgesetzt.  
\end{defi}

Ist für $f,g: D \rightarrow \mathbb{C}$ der Definitionsbereich $D$ endlich, so können die Funktionen als zeitdiskrete Signale $f=(f_0, \ldots, f_{n-1})^T \in \mathbb{C}^{n}$ und $g=(g_0, \ldots, g_{n-1}) \in \mathbb{C}^{n}$ aufgefasst werden. Durch das Fortsetzen mit Nullen besitzen die Vektoren $f$ und $g$ die gleiche Länge. In diesem Fall kann die Faltung als Matrix-Vektor-Produkt mit einer zyklischen Matrix ausgedrückt werden. 

\begin{defi}[Zyklische Matrix, vgl. Gray\cite{gray2006toeplitz}]
    Eine quadratische Matrix heißt zyklisch im Vektor $a=(a_0, \ldots, a_{n-1})^T \in \RR^n$, wenn sie die Gesatlt
    \begin{equation*}
        \mathrm{zyk}(a):=
        \begin{pmatrix}
            a_0 & a_{n-1} &a_{n-2} &\ldots &a_1 \\ 
            a_1 & a_0 &a_{n-1} & \ldots &a_2 \\
            a_2 & a_1 &a_0 & \ldots &a_3 \\
             &\ddots &\ddots &\ddots & \\
            a_{n-1} &a_{n-2} &a_{n-3} &\ldots &a_0
        \end{pmatrix}
    \end{equation*}
    besitzt.
\end{defi}    
    
\begin{bem}
Für ein zeitdiskrete Signal $f=(f_0, \ldots, f_{n-1})^T \in \mathbb{C}^{n}$ sei $F=\mathrm{zyk}(f)$ die zyklische Matrix im Vektor $f$. Sei weiter $g=(g_0, \ldots, g_{n-1}) \in \mathbb{C}^{n}$. Dann lässt sich mit
    \begin{equation*}
        (F g)_k=\sum_{j=0}^{n-1}  f_{k-j} g_j,  \; \; k=0, \ldots, n-1
    \end{equation*}
    die diskrete Faltung von $f$ und $g$ darstellen. Dabei werden Indizies außerhalb von $0, \ldots, n-1$ zyklisch durch Modulo-Rechnung ($\mathrm{mod} \; n)$ in den gültigen Indexbereich abgebildet.
\end{bem}

In Hinblick auf die Klassifikation von digitalisierten Bildern, dargestellt als zweidimensionale Signale, wird die zweidimensionale Faltung mit sogenannten quadratischen Kernen $K \in \RR^{k \times k}$ mit ungeradem $k \in \mathbb{N}$ definiert.
 
%Um die Notation einfach zu halten, sei ein Kern $K$ ebenfalls eine $h \times b$ - Matrix, indem $K$ mit Nullen aufgefüllt wird.

%\begin{defi}[Zweidimensionale Faltung]
    %Sei $X \in \RR^{h \times b}$ und $K \in \RR^{k \times k}$. Das Ergebnis der zweidimensionalen Faltung ist die Matrix $Y= X \ast K \in \RR^{h \times b}$ mit

%    \begin{equation}
%        \label{eq:2dmatrixconv}
%        Y_{i,j}= \sum_{p \in [k]} \sum_{q \in [k]} X_{p,q} K_{p-i+3,q-j+3}.
%    \end{equation}
%    Dabei besitzt $Y$ die selben Abmessungen wie $X$. 
%\end{defi}

\begin{defi}[Zweidimensionale Faltung, vgl. \cite{gruening}] \label{def:matrix_faltung}
    Für gegebene Matrizen $X \in \RR^{h \times b}$ und $K \in \RR^{k \times k}$ sei $h=\lfloor k/2  \rfloor$.
    %\begin{equation*}
    %   h_l=\begin{cases}
    %       \lfloor k_h/2  \rfloor &, k_h \, \text{ungerade} \\
    %       k_h/2-1 &, \text{sonst}
    %    \end{cases}, \; \; 
    %    w_l=\begin{cases}
    %        \lfloor k_w/2 \rfloor &, k_w \, \text{ungerade} \\
    %        k_w/2 -1 &, \text{sonst}    
    %    \end{cases}.
    %\end{equation*} 
    Die zweidimensionale Faltung  $Y=X \ast K \in \RR^{h \times w}$ ist als 
    \begin{equation}
        \label{eq:matrix_faltung}
        (Y)_{i,j}:=\sum_{l=-h}^{h} \sum_{m=-h}^{h} X_{i+l,j+m} K_{l+h_l+1, m+w_l+1}\; \; \forall i \in [h], j \in [b]
    \end{equation} mit $X_{i,j}=0$ für $i \notin [h]$ und $j \notin [b]$ definiert. In der Literatur wird das Aüffüllen mit Nullen am Rand von $X$ mit \textit{zero padding} bezeichnet. In dieser Definition besitzt das Ergebnis $Y$ der Faltung die gleichen Abmessungen wie $X$.
\end{defi}

Bei gefalteten neuronalen Netzen wird oft eine Reduktion der Dimensionen angestrebt. Dafür werden natürliche Zahlen als Schrittweiten, engl. \textit{strides}, genutzt.
\begin{bem}\label{bem_strides}
    Für Schrittweiten $s_h, s_b \in \mathbb{N}$ ergibt sich die reduzierte zweidimensionale Faltung $Y=X \ast K$ zu
    \begin{equation*}
        (Y)_{i,j}:=\sum_{l=-h}^{h} \sum_{m=-h}^{h} X_{i \cdot s_h +l,j \cdot s_b +m} K_{l+h_l+1, m+w_l+1}\; \; \forall i \in [\lceil h/s_h \rceil], j \in [\lceil b/s_b \rceil].
    \end{equation*}
    Für $s_h=s_b=1$ ergibt sich die Standardvariante wie in \ref{eq:matrix_faltung}.
    \end{bem}

\section{CNN Architektur}

Beim maschinellen Lernen sind Eingabedaten oft als mehrdimensionale Arrays abgelegt, welche eine oder mehrere Achsen repräsentieren, wobei die Ordnung dieser eine Rolle spielt. Bei digitalisierten Bildern sind das bespielsweise die Höhe und Breite des Bildes, welche als Raumachsen bezeichnet werden. Hinzu kommen Kanalachsen als weitere Verfeinerung der Daten, zum Beispiel besitzen Grauwert-Bilder einen Farbkanal, während RGB-Farbbilder drei Kanäle der Farben rot, grün und blau besitzen. Dementsprechend werden Grauwert-Bilder wie in Definition \ref{def:image} nun als dreidimensionale Arrays $X \in [0,1]^{h \times b \times 1}$ dargestellt. Dies erlaubt die Definition der gefalteten Übertragungsfunktion, wie in Gruening\cite{gruening}

\begin{defi}[Gefaltete Übertragungsfunktion]
    Sei ein vierdimensionales Array $K \in \RR^{k \times k \times z_{in} \times z_{out}}$ und ein Biasvektor $b \in \RR^{z_{out}}$ gegeben. Die Funktion 
    \begin{equation*}
        \Psi_{conv}^{K,b}: \RR^{\cdot \times \cdot \times z_{in}} \rightarrow \RR^{\cdot \times \cdot\times z_{out}}
    \end{equation*}
    mit
    \begin{equation*}
        \Psi_{conv}^{K,b}(X)_{:,:,l}:= \sum_{p=1}^{z_{in}} K_{:,:,p,l} \ast X_{:,:,p} +b_l \; \forall l \in [z_{out}]
    \end{equation*}
    wird gefaltete Übertragungsfunktion bezeichnet. Mit $\ast$ ist die zweidimensionale Faltung wie in Definition \ref{def:matrix_faltung} und mit $\cdot$ beliebige Raumachsen gemeint.
\end{defi}


\begin{bem}
    Ist $\psi: \RR \rightarrow \RR$ eine Aktivierungsfunktion wie in Definition \ref{def_act_f}, so wird für $X \in \RR^{\cdot \times \cdot \times z}$ mit 
    \[\psi(X)_{i,j,:}:=\left(\psi(X_{i,j,1}), \ldots, \psi(X_{i,j,z})\right)^T \in \RR^z \; \; \forall i \in [{}_1 X], j \in [{}_2 X] 
    \]
    der Vektor bezeichnet, welcher sich durch die elementweise Auswertung der Aktivierungsfunktion $\psi$ ergibt.
\end{bem}

Ähnlich der Definition \ref{def:NNlayer} wird nun eine Faltungsschicht als Verknüpfung von gefalteter Übertragungsfunktion und Aktivierungsfunktion definiert.

\begin{defi}[Faltungsschicht]
    \label{def:convlayer}
    Ist $\Psi_{conv}^{K,b}$ eine gefaltete Übertragungsfunktion und $\psi$ eine Aktivierungsfunktion, so wird das Paar $(\Psi_{conv}^{K,b}, \psi)$ als Faltungsschicht $\mathcal{S}_{conv}$ bezeichnet. Für eine Eingabe $X \in \RR^{\cdot \times \cdot \times z_{in}}$ ist die Ausgabe $Y \in \RR^{\cdot \times \cdot \times z_{out}}$ der Schicht $\mathcal{S}_{conv}$ durch
    \[Y=\psi \circ \Psi_{conv}^{K,b}(X)= \psi\left(\Psi_{conv}^{K,b}(X)\right)
        \] 
        gegeben. Die Matrizen $Y_{:,:,p}$ werden für $1 \leq p \leq z_{out}$ Merkmalskarten genannt.
\end{defi}

Im Folgenden werden konkrete Beispiele für verschiedene zweidimensionale Faltungen, welche in dieser Arbeit im Fokus stehen, gegeben. Dabei sind die Eingabe $X \in \RR^{h \times w}$ und der Filter $K \in \RR^{k_h \times k_w}$ immer als Matrizen zu verstehen. Das Ergebnis der Faltung $S= X \ast K$ wird als Merkmalskarte bezeichnet. Es sei angemerkt, dass oft $k_h=k_w$ sowie $k_h$ ungerade gewählt wird, z.B. $k_h=3$ oder $k_h=5$. Die Größe der Merkmalskarte wird durch die Parameter
\begin{itemize}
    \item $h,w$: Die Höhe und Breite der Eingabe,
    \item $k_h, k_w$: Die Abmessungen des Filters,
    \item $s_h, s_w$: Die Wahl der strides, 
    \item $p_h, p_w$: Die Größe des zero paddings
\end{itemize}
beeinflusst. Mit zero padding ist gemeint, dass künstliche Nullen um Randpixel der Eingabe $X$ eingefügt werden, damit die Berechnung mit dem Filter um jene Pixel gelingt. Ein Beispiel für das Verwenden von zero padding wird in Abbildung \ref{abb_simplematrixconv_padding} gezeigt. In Abbildung \ref{abb_simplematrixconv} ist die Berechnung einer einfachen zweidimensionalen Matrixfaltung dargestellt. Ein vorher festgelegter Filter (grau) bewegt sich über die Eingabe (blau) und berechnet jeweils die Einträge der Ausgabe(grün). 

\begin{figure}[h]
    \includegraphics[width=0.8\textwidth]{pics/abb_simpleconv}
    \centering
    \caption{Es wird die Merkmalskarte $S \in \RR^{3 \times 3}$ mit den Parametern ${h,w=5}, k_h=k_w=3, s_h=s_w=1$ und $p_h=p_w=1.$}
    \label{abb_simplematrixconv}
\end{figure}

\begin{figure}[h]
    \includegraphics[width=0.8\textwidth]{pics/abb_simplecov_padding}
    \centering
    \caption{Es wird die Merkmalskarte $S \in \RR^{3 \times 3}$ mit den Parametern ${h,w=5}, k_h=k_w=3, s_h=s_w=2$ und $p_h=p_w=1$ berechnet.}
    \label{abb_simplematrixconv_padding}
\end{figure}



\section{Motivation der Faltung}
..
Sie nutzt wichtige Konzepte zur Optimierung von Machine-Learning-Verfahren wie spärliche Konnektivität (engl. \textit{sparse connectivity}), \textit{Parameter Sharing} und \textit{äquivariante Repräsentation}, vgl. \cite{goodfellow}. Spärliche Konnektivität bedeutet, dass Neuronen auf einer Schicht $\mathcal{s}_{l+1}$ nur durch wenige Neuronen der Schicht $\mathcal{S}_l$ beeinflusst wird. Dies ist bei CNNs typisch, da meist die verwendeten Filter viel kleiner als die Eingabe ist. Noch mehr erklären + Abbildung

Mit Parameter Sharing ist die Nutzung von gleichen Parametern für mehrere Funktionen im neuronalen Netz gemeint. In herkömmlichen Feed-Forward-Netzen wird jedes Element der Gewichtsmatrizen für die Berechnung der Aktivierungen der jeweiligen Schichten verwendet. Anschließend werden diese Gewichte dann nicht mehr gebraucht. Im Zusammenhang von CNNs bedeutet Parameter Sharing während der Faltungsoperation, dass nur eine bestimmte Menge von Parametern erlernt werden müssen
Noch mehr erklären + Abbildung


\chapter{Datenbankgestützte Implementierung von CNN}
\label{kap:CNN_in_SQL}
In diesem Kapitel werden Ideen zur Umsetzung der datenbankgestützten Mustererkennung durch tranierte gefaltete neuronale Netze vorgestellt. Der Schlüssel liegt dabei in der effektiven Implementierung der Faltungsoperation als SQL-Anfrage. Gelingt dies, so kann die Vorwärtsrechnung eines CNN durch Komposition von Faltungs-, Pooling- und Matrixvektoroperationen umgesetzt werden. Im Abschnitt \ref{abs:conv_in_sql} werden drei Implementierungsmöglichkeiten der Matrixfaltung, siehe \ref{def:matrix_faltung}, in SQL beleuchtet. Dabei stellt sich heraus, dass es sich lohnt, die diskrete Faltung als lineare Operationen mithilfe der in Kapitel \ref{kap:fund} vorgestellten Basisoperationen umzusetzen. So gelingt es, die Problemstellung \ref{prob:conv_in_sql} zu beantworten und auch für relativ große Grauwertbilder akzeptable Laufzeiten zu erreichen. 

Im Abschnitt \ref{abs_CNN_in_SQL} wird hinsichtlich Problem \ref{prob:ffCCN} eine (objekt-) relationale Umsetzung der Vorwärtsrechnung für das Modell \ref{modell} zur Ziffernerkennung vorgestellt. Dieses Modell sei mithilfe des Backpropagationsalgorithmus \ref{alg:cnn_online} bereits traniert, siehe Abschnitt \ref{abs:model_mnist}.
\section{Die Faltungsoperation in SQL}
\label{abs:conv_in_sql}
Die Faltung zweier zeitdiskrete Signale stellt die Kernoperation innerhalb von CNN dar. Sind die entsprechenden Signale zweidimensional wie in Problemstellung \ref{prob:conv_in_sql}, so muss die Matrixfaltung $X \ast K$ für $X \in \RR^{h \times b}$ und $K \in \RR^{k \times k}$ mithilfe des im Abschnitt \ref{abs:relation_intro} vorgestellten SQL-Kerns umgesetzt werden. Dazu werden in diesem Abschnitt drei Varianten vorgestellt, welche das Coordinate-Schema bzw. das Spaltenkompression-Schema zur Darstellung der Matrizen $X$ und $K$ nutzen. 
\subsection{Faltung mit Nachbarschaften}
\label{abs:naive_app}
Der erste Ansatz beruht auf den Nachbarschaftsbeziehungen der Pixel innerhalb der Matrix $X$, die durch die Faltung mit einem gedrehten Kern $K$ mit den Abmessungen $k \times k$ gegeben sind. Dazu werden einige Bezeichnungen im Folgenden eingeführt. Seien $l=\lfloor k/2 \rfloor$ und die Mengen
\begin{align*}
    N:=\{(i,j) \; :\; 1 \leq i \leq b, 1 \leq j \leq h \}
\end{align*} 
sowie
\begin{align*}
    F:=\{(i,j) \; :\; -l \leq i \leq l, -l \leq j \leq l \}
\end{align*} 
gegeben, welche die Positionen der Matrizen $X$ und $K$ widerspiegeln. Hier wird wieder die spezielle Indizierung des Kerns $K$ wie in Bemerkung \ref{bem:K_conv_komp} genutzt. Nun kann für jeden Pixel $(i,j)$ von $X$ eine Umgebung $U(i,j)$  in Abhängigkeit von $F$ definiert werden, und zwar
\begin{equation}
    \label{eq:neighborhood}
    U(i,j):=\{(i', j') \; : \; (i'-i, j'-j) \in F \}.
\end{equation}
Die Menge $U(i,j)$ wird auch als Nachbarschaft des Pixels $(i,j)$ bezeichnet.
Die Matrixfaltung $Y= X \ast K$ lässt sich mit den Umgebungen (\ref{eq:neighborhood}) durch
\begin{equation}
    \label{eq:naive_approach}
    Y_{i,j}=\sum_{(i',j') \in U(i,j)} X_{i', j'} K_{i'-i, j'-j}, \; \; \forall i \in [h], \forall j \in [b]
\end{equation}
berechnen. Dabei wird $X$ außerhalb des Definitionsbereiches mit Nullen aufgefüllt, sodass das Ergebnis $Y$ wieder die gleichen Abmessungen wie $X$ besitzt.

Zur Umsetzung der Faltungsoperation sind also zunächst die Nachbarschaften für jeden Pixel von $X$ zu berechnen. In SQL kann dies mithilfe des kartesischem Produkts implementiert werden. Dazu bezeichnen $\mathbf{X}$ und $\mathbf{K}$ die Relation zur Darstellung der Matrizen $X \in \RR^{h \times b}$ und $K \in \RR^{k \times k}$ im Coordinate-Schema. Eine rein relationale Umsetzung von (\ref{eq:naive_approach}) ist durch die SQL-Anfrage \ref{sql:naive} gegeben. Zur Übersicht sind die SQL-Operationen blau gekennzeichnet. 

TODO systemspecs

\lstinputlisting[label=sql:naive, caption=SQL-Code zur Umsetzung der Faltung mit Nachbarschaften, language=SQL]{sql_code/neighborhood.sql}

In der Anfrage (Zeile 4-14) werden die Nachbarschaften aller Pixel in der temporären Relation $\mathbf{kreuz}$ berechnet. Dabei wird die \textbf{BETWEEN}-Funktion zur kompakten Darstellung der Konstantenselektion bezüglich $l=\lfloor k/2 \rfloor$ genutzt. In den Zeilen 15-17 werden dann die Nachbarschaften über die entsprechenden Indizies der Filtermatrix verbunden und schließlich die \textbf{SUM}-Funktion genutzt, um das Faltungsergebnis zu berechnen. Dieser naive Ansatz nutzt keine lineare Algebra in Form von Matrixvektor- oder Matrixmatrixmultiplikation und ist hinsichtlich des Problems \ref{prob:conv_in_sql} schon für verhältnismäßig kleine Grauwertbilder ineffizient. Zu Erkennen ist dies in der Abbildung \ref{abb:laufzeit_naive}, bei der die Laufzeiten der SQL-Anfrage \ref{sql:naive} in Abhängigkeit von der Dimension $n$ für allgemeine Matrizen $X \in \RR^{n \times n}$ dargestellt ist.

TODO Grafik ergebnis erklären.

Eine Verbesserung der Laufzeiten kann mithilfe von Umsetzungstabellen, engl. \textit{Lookup tables}, erreicht werden. Da die Dimensionen aller vorkommenden Merkmalskarten und Kerne eines CNN von Anfang an durch die Wahl der Hyperparameter festgelegt werden, müssen die Nachbarschaften für die Faltung und das Pooling nur einmalig berechnet werden. Dies kann zudem vor der eigentlichen Vorwärtsrechnung durchgeführt werden. So wird zwar der Speicheraufwand erhöht, aber der Zeitaufwand hinsichtlich der Faltungs- und Poolingoperation deutlich vermindert. Dazu werden pro Faltungs- und Poolingschicht jeweils eine Umsetzungstabelle \textbf{U} in der Form 
\begin{align*}
    \mathbf{U}( &\underline{i} \; \; \mathrm{int}, \\
    &\underline{j} \; \;\mathrm{int},\\
    &\underline{\text{id}} \; \; \mathrm{int}, \\
    &\text{istrich} \; \; \mathrm{int},\\
    &\text{jstrich}\; \; \mathrm{int})
\end{align*}
mit dem zusammengesetzten Schlüssel $(i,j,\text{id})$ benötigt. Hier werden für jeden Pixel $(i,j)$ die Nachbarpixel $(i', j') \in U$ mit einer entsprechenden Nummer $\text{id}$ zur Identifikation  hinterlegt. So wird die zeitintensive Berechnung der Nachbarschaften mit dem  kartesischen Produkt in der Anfrage \ref{sql:naive} umgangen. Die Laufzeiten der verbesserten Anfrage \ref{sql:naive_better} in Abhängigkeit von der Dimension $n$ sind in Abbildung \ref{abb:laufzeit_naive_verb} dargestellt. Dabei ist der erhöhte Speicheraufwand für die Umsetzungstabellen sichtbar.

\lstinputlisting[label=sql:naive_better, caption=SQL-Code zur Umsetzung der Faltung mit Umsetzungstabellen, language=SQL]{sql_code/neigh_verbesster.txt}

\subsection{Faltung als Matrixvektorprodukt}
\label{abs:conv_using_sparse}
Zwei beliebige Funktionen $f,g: D \rightarrow \mathbb{R}$ mit endlichem Definitionsbereich $D$ können als zeitdiskrete Signale $f=(f_0, \ldots, f_{n-1})^T \in \mathbb{R}^{n}$ und $g=(g_0, \ldots, g_{n-1})^T \in \mathbb{R}^{n}$ aufgefasst werden. %Durch das Fortsetzen mit Nullen besitzen die Vektoren $f$ und $g$ die gleiche Länge. 
Die zyklische Faltung dieser eindimensionalen Signale wird im Folgenden erklärt.

\begin{defi}[Zyklische Faltung]
    \label{def:cycconv}
    Die zyklische Faltung $y=f \circledast g \in \RR^{n}$ zweier zeitdiskreter Signale $f \in \RR^n$ und $g \in \RR^n$ ist durch
    \begin{equation*}
        y_k:=\sum_{j=0}^{n-1} f_j g_{(k-j) \, \mathrm{mod} \; n},  \; \; 0 \leq k \leq n-1
    \end{equation*}
    definiert. Dabei werden Indizies außerhalb von $0, \ldots, n-1$ zyklisch durch Modulo-Rechnung ($\mathrm{mod} \; n)$ in den gültigen Indexbereich abgebildet.
\end{defi}
In diesem Fall kann die zyklische Faltung von $f$ und $g$ als Matrixvektorprodukt mit speziellen Toeplitz-Matrizen formuliert werden. 

\begin{defi}[Toeplitz-Matrix, Zyklische Matrix]
    \label{def:toeplitzM}
    Eine diagonalkonstante Matrix $A \in \RR^{n \times n}$ der Gestalt
    \begin{equation*}
    A=
    \begin{pmatrix}
        a_0 & a_{-1} &a_{-2} &\ldots &\ldots &a_{-(n-1)} \\ 
        a_1 & a_0 &a_{-1} &\ddots & &\vdots \\
        a_2 & a_1 &\ddots &\ddots &\ddots &\vdots\\
        \vdots & \ddots &\ddots &\ddots &a_{-1} &a_{-2}\\
        \vdots & &\ddots &a_1 &a_0 &a_{-1} \\
        a_{n-1} &\ldots &\ldots &a_{2} &a_{1} &a_0
    \end{pmatrix}
\end{equation*}
    wird Toeplitz\footnote{Otto Toeplitz 1881-1940}-Matrix genannt. Hierbei gilt $A_{i,j}=A_{i+1,j+1}=a_{i-j}$ für alle Indizies $0 \leq i, j \leq n-1$. Eine quadratische Toeplitzmatrix ist damit durch ihre erste Zeile und Spalte eindeutig bestimmt. Für den Spezialfall $a_i=a_{-(n-i)}=a_{i-n}$ für alle $0 \leq i \leq n-1$ wird $A$ zyklische Matrix genannt. Zyklische Matrizen sind damit eindeutig durch einen Vektor $a \in \RR^n$ charakterisiert.
\end{defi}

%\begin{defi}[Zyklische Matrix, vgl. Gray\cite{gray2006toeplitz}]
  %  \label{def:zykM}
   % Eine quadratische Matrix heißt zyklisch im Vektor $a=(a_0, \ldots, a_{n-1})^T \in \RR^n$, wenn sie die Gesatlt
    %\begin{equation*}
        %\mathrm{zyk}(a):=
        %\begin{pmatrix}
            %a_0 & a_{n-1} &a_{n-2} &\ldots &a_1 \\ 
            %a_1 & a_0 &a_{n-1} & \ldots &a_2 \\
            %a_2 & a_1 &a_0 & \ldots &a_3 \\
            % &\ddots &\ddots &\ddots & \\
            %a_{n-1} &a_{n-2} &a_{n-3} &\ldots &a_0
        %\end{pmatrix}
    %\end{equation*}
   % besitzt.
%\end{defi}    
    
Für zeitdiskrete Signale $f,g \in \mathbb{R}^{n}$ kann die zyklische Faltung $y=f \circledast g$ als Matrixvektorprodukt mit einer zyklischen Matrix dargestellt werden. Es gilt 
\begin{equation*}
    y=\underbrace{\begin{pmatrix}
        g_0 & g_{n-1} &g_{n-2} &\ldots &g_1 \\ 
        g_1 & g_0 &g_{n-1} &\ldots &g_2 \\
        g_2 &g_1 &g_0 &\ddots &g_3\\
        &\ddots &\ddots &\ddots & & \\
        g_{n-1} &g_{n-2} &g_{n-3} &\ldots &g_0
    \end{pmatrix}}_{=:\mathrm{zyk}(g)}
    \begin{pmatrix}
        f_0 \\
        f_1 \\
        f_2 \\
        \vdots \\
        \vdots \\
        f_{n-1}
    \end{pmatrix}
\end{equation*}
Die Faltung zweidimensionaler Signale kann mithilfe von doppelt zyklischen Blockmatrizen dargestellt werden.
    %Matrix im Vektor $f$. Sei weiter $g \in \mathbb{R}^{n}$. Dann lässt sich mit
%\begin{equation*}
 %       (F g)_k=\sum_{j=0}^{n-1}  f_{k-j} g_j,  \; \; k=0, \ldots, n-1
  %  \end{equation*}
   % die diskrete Faltung von $f$ und $g$ darstellen. Dabei werden Indizies außerhalb von $0, \ldots, n-1$ zyklisch durch Modulo-Rechnung ($\mathrm{mod} \; n)$ in den gültigen Indexbereich abgebildet. Zyklische Matrizen aus Definition \ref{def:zykM} stellen Spezialfälle von Toeplitz-Matrizen dar.
\begin{defi}[Doppelt zyklische Blockmatrix]
    \label{def:double_circ}
    Eine Blockmatrix $A \in \RR^{n^2 \times n^2}$ bestehend aus Blockmatrizen $B_{i,j} \in \RR^{n \times n}$ heißt zyklische Blockmatrix, genau dann wenn die Matrizen $B_{i,j}$ für alle $ 1 \leq i, j \leq n$ zyklisch im Sinne von Definition \ref{def:toeplitzM} sind. Ist zusätzlich $A$ eine zyklische Matrix, so wird $A$ doppelt zyklischen Blockmatrix genannt.
\end{defi}

Die Konstruktion solcher zyklischen Blockmatrizen soll im Folgenden beleuchtet werden. Dazu seien die Matrizen $X \in \RR^{n \times n}$ und $K \in \RR^{k \times k}$ gegeben. Zunächst wird der Kern $K$ in eine $n \times n$-Matrix eingebettet, die wieder mit $K \in \RR^{n \times n}$ bezeichnet wird. Dazu wird der Kern von unten und von rechts mit Nullen aufgefüllt, siehe Beispiel \ref{bsp:Kzeropad}.
Weiter bezeichne $\mathrm{vec}(X)$ die Transformation der Matrix $X$ in einen Vektor der Länge $n^2$, indem die Spalten von $X$ untereinander geschrieben werden, ähnlich wie bei der Flatten-Funktion \ref{def:flatten} aus Abschnitt \ref{abs:CNN_arch}. Das folgende Lemma liefert die Darstellung der Matrixfaltung, vgl. Bemerkung \ref{bem:K_conv_komp}, als Matrixvektorprodukt.

\begin{lem}[vgl. Jain\cite{jain1989fundamentals}, \cite{DBLP:journals/corr/abs-1805-10408}]
    Für einen gedrehten Kern $K \in \RR^{n \times n}$ wird die zyklische Blockmatrix $A \in \RR^{n^2 \times n^2}$ als
    \begin{equation*}
        A=\begin{bmatrix}
            \mathrm{zyk}(K_{1,:}) &\mathrm{zyk}(K_{2,:}) &\ldots &\mathrm{zyk}(K_{n,:}) \\
            \mathrm{zyk}(K_{n,:}) &\mathrm{zyk}(K_{1,:}) &\ldots & \; \; \;\mathrm{zyk}(K_{n-1,:})\\
            \vdots &\vdots &\ddots &\vdots\\
            \mathrm{zyk}(K_{2,:}) &\mathrm{zyk}(K_{3,:}) &\ldots &\mathrm{zyk}(K_{1,:})
        \end{bmatrix}
    \end{equation*}
    konstruiert. Sei die zweidimensionale Faltung  von $X$ und $K$ als
    \begin{equation*}
        Y_{i,j}= \sum_{u=0}^{n-1} \sum _{v=0}^{n-1} X_{i+u,j+v} K_{u,v} 
    \end{equation*}
    mit $X_{i,j}=0$ für $i \notin\{0,\ldots, n-1\}$ und $j \notin \{0, \ldots, n-1\}$ gegeben. Dann gilt der Zusammenhang
    $\mathrm{vec}(Y)=A \, \mathrm{vec}(X)$. Die Matrix $A$ ist dünnbesetzt. 
\end{lem}

\begin{proof}
    Ein Beweis ist von Jain \cite{jain1989fundamentals} gegeben. Das Auffüllen von $K \in \RR^{k \times k}$ zu $K \in \RR^{n \times n}$ mit Nullen führt zur dünnbesetzten Struktur der Matrix $A$. 
\end{proof}
\begin{bsp}
    \label{bsp:Kzeropad}
    Seien die Matrizen
    \begin{equation*}
        X=\begin{pmatrix}
            x_1 & x_2 &x_3 \\
            x_4 & x_5 &x_6 \\
            x_7 & x_8 &x_9 \\
        \end{pmatrix}, \; \;
        K=\begin{pmatrix}
            k_1 & k_2 &0\\
            k_3 &k_4 &0 \\
            0 &0 &0
        \end{pmatrix}
    \end{equation*}
    gegeben. Mit 
    \begin{align*}
    A \, \mathrm{vec}(X) &=    
    \begin{pmatrix}
        k_1 & k_2 & 0 &k_3 &k_4 &0 &0 &0 &0 \\
        0 & k_1 & k_2 &0 &k_3 &k_4 &0 &0 &0 \\
        0 & 0 & 0 &k_1 &k_2 &0 &k_3 &k_4 &0 \\
        0 & 0 & 0 &0 &k_1 &k_2 &0 &k_3 &k_4 
    \end{pmatrix}
    \begin{pmatrix}
        x_1 \\
        x_2 \\
        x_3 \\
        x_4 \\
        x_5 \\
        x_6 \\
        x_7 \\
        x_8 \\
        x_9
    \end{pmatrix} \\
    &=\begin{pmatrix}
        k_1 x_1+ k_2 x_2 +k_3 x_4 +k_4 x_5 \\
        k_1 x_2+ k_2 x_3 +k_3 x_5 +k_4 x_6 \\
        k_1 x_4+ k_2 x_5 +k_3 x_7 +k_4 x_8 \\
        k_1 x_5+ k_2 x_6 +k_3 x_8 +k_4 x_9 
    \end{pmatrix}=\mathrm{vec} \, (Y),
\end{align*}
kann die Matrixfaltung $Y = X \ast K$ ohne zero padding, welche im Modell \ref{modell} benötigt wird, als dünnbesetzte Matrixvektormultiplikation berechnet werden. Bei diesem Besipiel ist zu beachten, dass $A$ keine doppelt zyklische Blockmatrix ist, da sich die Abmessungen der Ergebnismatrix $ Y \in \RR^{ 4 \times 4}$ verkleinern. Es fehlen also gewisse Zeilen in $A$, sodass $A$ nicht zyklisch ist.
\end{bsp}
Die genaue Konstruktion doppelt zyklischer Blockmatrizen für die Faltungsoperation mit bzw. ohne zero padding ist ein technisches Detail und wird in dieser Arbeit nicht weiter beleuchtet. Für eine mögliche Implementierung sei auf den MATLAB-Code \ref{mat:code:zyk} im Anhang verwiesen. 

Die datenbankgestützte Umsetzung der Matrixvektormultiplikation mit dünnbesetzter Matrix wurde bereits im Abschnitt \ref{abs:SQL_linalg} erläutert. Bei CNN lassen sich die dünnbesetzten zyklischen Blockmatrizen für die Kerne bereits vor der Vorwärtsrechnung in Relationen speichern. Seien $X \in \RR^{n \times n}$ und $K \in \RR^{k \times k}$ sowie \textbf{vecX} und \textbf{K} entsprechende Relationen des Vektors $\mathrm{vec}(X)$ bzw. der eingebetteten $n \times n$-Matrix $K$ im Coordinate-Schema. Die Matrixfaltung $Y = X \ast K$ lässt sich mit den obigen Resultaten in der SQL-Anfrage \ref{sql:sparseapp} berechnen. In der Abbildung \ref{abb:sparseapp} sind numerische Resultate hinsichtlich des Speicher- und Zeitaufwands dargestellt. 

\lstinputlisting[label=sql:sparseapp, caption=SQL-Code zur Umsetzung der Matrixfaltung als Matrixvektorprodukt, language=SQL]{sql_code/sparsecnn.txt}

\subsection{Diskrete Fourier-Transformation}
Die Diskrete Fourier\footnote{Jean Baptiste Joseph Fourier 1786-1830}-Transformation ist eine Methode aus dem Bereich der Fourier-Analysis. Dabei werden zeitdiskrete endliche Signale auf sogenannte diskrete, periodische Frequenzspektren abgebildet. In diesem Kontext wird zwischen Zeitbereich und Frequenzbereich unterschieden.

\begin{defi}[Diskrete Fourier-Transformation]
    \label{def:dft}
    Im Zeitbereich sei ein diskretes Signal $x=(x_0, \ldots, x_{n-1})^T \in \RR^n$ gegeben. Dann wird mit $\hat{x}=(\hat{x}_0, \ldots, \hat{x}_{n-1}) \in \mathbb{C}^n$ das Ergebnis der diskreten Fourier-Transformation, kurz $\hat{x}=\mathrm{DFT}(x)$, bezeichnet. Die sogenannten Fourier-Koeffizienten sind als 
    \begin{equation*}
        \hat{x}_k:=\sum_{j=0}^{n-1} \mathrm{e}^{- \frac{2 \pi i}{n} j k} \cdot x_j
    \end{equation*}
    für $0 \leq k \leq n-1$ definiert. Das komplexe Signal $\hat{x}$ ist dem Frequenzbereich zugeordnet.
\end{defi}
Mit der inversen Fourier-Transformation kann aus dem Signal im Frequenzbereich das Signal im Zeitbereich rekonstruiert werden. Damit ist es möglich, Signale im Frequenzbereich zu manipulieren und zwischen Zeit- und Frequenzbereich beliebig zu wechseln.
\begin{defi}[Inverse Diskrete Fourier-Transformation]
    Sei $\hat{x}=(\hat{x}_0, \ldots, \hat{x}_{n-1}) \in \mathbb{C}^n$. Mit den Koeffizienten
    \begin{equation*}
        x_j:= \frac{1}{n} \sum_{k=0}^{n-1} \mathrm{e}^{\frac{2 \pi i}{n} j k} \cdot \hat{x}_k, \; \; 0 \leq j \leq n-1
    \end{equation*}
    lässt sich die inverse diskrete Fourier-Transformation, kurz $x=\mathrm{iDFT}(\hat{x})$, angeben. Das Paar $(x ,\hat{x})$ wird Fourier-Paar genannt.
\end{defi}
Die diskrete Fourier-Transformation aus Definition \ref{def:dft} lässt sich in ein Matrixvektorprodukt $\hat{x}=Fx$ überführen, wobei $F \in \mathbb{C}^{n \times n}$ eine symmetrische Matrix der Gestalt
\begin{equation}
    \label{eq:FM}
    F=\begin{pmatrix}
        &1 &1 &1 &\ldots &1 \\
        &1 &\omega_n &\omega_n^2 &\ldots &\omega_n^{(n-1)} \\
        &1 &\omega_n^2 &\omega_n^4 &\ldots &\omega_n^{2 (n-1)} \\
        &\vdots &\vdots & &\ddots &\vdots \\
        &1 &\omega_n^{(n-1)} &\omega_n^{2(n-1)}  &\ldots &\omega_n^{(n-1)(n-1)}
    \end{pmatrix}
\end{equation}
mit 
\begin{equation*}
    \omega_n^{j}:=\mathrm{e}^{- \frac{2 \pi i}{n} j}, \; \; 0 \leq j \leq n-1
\end{equation*}
ist. Diese Matrix wird Fourier-Matrix genannt und deren Einträge $\omega_n^{j}$ als $n$-te Einheitswurzeln bezeichnet. Es gilt $\omega_n^n=1$.
\begin{lem}
    \label{lem:Finv}
    Es gilt $F^H=\bar{F}$ und die Matrix $\frac{1}{\sqrt{n}} F$ ist unitär. Für $x \in \RR^n$ sei $\hat{x}=Fx$. Dann gilt $F^{-1}=\frac{1}{n} \bar{F}$ und $x= \frac{1}{n}\bar{F} \hat{x}$.
\end{lem}
\begin{proof}
    Wegen $F=F^T$ gilt 
    \begin{equation*}
        F^H=\bar{{F}}^T=\bar{F}.
    \end{equation*}
    Mit $W:=F\bar{F}$ gilt $W_{k,j}=\sum_{l=0}^{n-1} \omega_n^{(j-k)l}$. Ist $k=j$ so ergeben sich die Einträge auf der Hauptdiagonalen von $W$ zu $n$. Ist $k \neq j$, so ist $\omega_0:=\omega_n^{j-k} \neq 1$ eine $n$-te Einheitswurzel.
    Mit der geometrischen Summenformel gilt
    \begin{equation*}
        W_{k,j}=\sum_{l=0}^{n-1} \omega_0^l=\frac{1-\omega_0^n}{1-\omega_0}=0.
    \end{equation*} 
    Also ist $\left(\frac{1}{\sqrt{n}}\right) F\left(\frac{1}{\sqrt{n}}\right) F^H=I$ und damit $\frac{1}{\sqrt{n}} F$ unitär. Schließlich gilt $\frac{1}{n} \bar{F} F=\frac{1}{n} F \bar{F}= I$ und damit $\hat{x}=Fx \Leftrightarrow \frac{1}{n}\bar{F} \hat{x}=x$.
\end{proof} 
Die inverse diskrete Fourier-Transformation lässt sich mithilfe der diskreten Fourier-Transformation berechnen. Dieser Zusammenhang wird insbesondere für die spätere datenbankgestützte Implementierung der Fourier-Transformationen genutzt.
\begin{lem}
    \label{lem:inversedftasdft}
    Sei das Fourier-Paar
    \begin{align*}
        \mathrm{DFT}(x)&: \; \;\hat{x}_k=\sum_{j=0}^{n-1} \mathrm{e}^{- \frac{2 \pi i}{n} j k} \cdot x_j, \\ 
        \mathrm{iDFT}(\hat{x})&:\; \; x_j= \frac{1}{n} \sum_{k=0}^{n-1} \mathrm{e}^{\frac{2 \pi i}{n} j k} \cdot \hat{x}_k
    \end{align*}
    für $x=(x_0, \ldots, x_{n-1})^T \in \mathbb{R}^n$ gegeben. Dann gilt $x=\frac{1}{n} (DFT(\hat{x}^*))^*$. Hierbei ist mit ${}^*$ die komplexe Konjugation gemeint.
\end{lem}
\begin{proof}
  Für alle $0 \leq j \leq n-1$  gilt
  \begin{align*}
    x_j^{*}&=\frac{1}{n} \sum_{k=0}^{n-1} \mathrm{e}^{-\frac{2 \pi i}{n} j k} \cdot \hat{x}^*_k \\
    &=\frac{1}{n} \mathrm{DFT}(\hat{x}^*)_j.
  \end{align*}
  Die Konjugation auf beiden Seiten liefert die Aussage.
\end{proof}
Wird ein zweidimensionales diskretes Signal in Form einer Matrix $X \in \RR^{n \times n}$ betrachtet, lässt sich die zweidimensionale diskrete Fourier-Transformation definieren. 
\begin{defi}
    Die zweidimensionale diskrete Fourier-Transformation für $X \in \RR^{n \times n}$, kurz $\hat{X}=2\mathrm{DFT}(X)$, ist als
    \begin{align*}
        \hat{X}_{u,v}:&= \sum_{l=0}^{n-1} \sum_{j=0}^{n-1} X_{l,j} \cdot \mathrm{e}^{\frac{2 \pi i}{n} -(lu+jv)} \\
        &=\sum_{l=0}^{n-1} \mathrm{e}^{-\frac{2 \pi i}{n} l u} \left(\sum_{j=0}^{n-1} X_{l,j} \cdot \mathrm{e}^{-\frac{2 \pi i}{n} j v}\right), \; \; 0 \leq u, v \leq n-1
    \end{align*}
    definiert.
\end{defi}
Die zweidimensionale diskrete Fourier-Transformation ist als Hintereinanderausführung von zwei eindimensionalen Fourier-Transformationen, vgl. Definition \ref{def:dft}, zu verstehen. Zuerst wird die $\mathrm{DFT}$ der Zeilen und anschließend die $\mathrm{DFT}$ der Spalten von $X$ berechnet. So lässt sich $\hat{X}=FXF^T$ als Matrixmatrixprodukt mit der Matrix $F$ aus (\ref{eq:FM}) darstellen.

Zwischen der zyklischen Faltung \ref{def:cycconv} und der diskreten Fourier-Transformation \ref{def:dft} besteht ein fundamentaler Zusammenhang, und zwar das Faltungstheorem. Eine Version davon wird im weiteren Verlauf dieser Arbeit genutzt, um die Matrixfaltung, vgl. Definition \ref{def:matrix_faltung}, mithilfe von Fourier-Transformationen zu berechnen.

\begin{satz}[Zyklisches Faltungstheorem]
    \label{satz:conv_theorem}
    Seien Vektoren $f,g \in \RR^{n}$ gegeben und $y= f \circledast g$ das Ergebnis der zyklischen Faltung. Dann gilt
    \begin{equation}
        \mathrm{DFT}(y)=\mathrm{DFT}(f) \odot \mathrm{DFT}(g).
    \end{equation}
    Dabei bezeichne $\odot$ die elementweise Multiplikation der Einträge von den beteiligten Vektoren.
\end{satz}
\begin{proof}
    Ein Beweis ist von Smith\cite{smith2007mathematics} gegeben.
\end{proof}
\begin{bem}
    Seien die Matrizen $X \in \RR^{n \times n}$ und $K \in \RR^{k \times k}$ mit ungeradem $k$ gegeben. Weiter sei der Kern $K$ gedreht und $l=\lfloor k/2 \rfloor$. Die Matrixfaltung $Y= X \ast K$ ist durch
    \begin{equation*}
        Y_{i,j}=\sum_{u=-l}^l \sum_{v=-l}^l X_{i+u, j+v} K_{u,v}, \; \; 1 \leq i, j \leq n
    \end{equation*}
    erklärt, vgl. Bemerkung \ref{bem:K_conv_komp}.
    Der Kern $K$ wird in eine $n \times n$-Matrix ähnlich wie in \ref{bsp:Kzeropad} eingebettet und diese wird wieder mit $K \in \RR^{n \times n}$ bezeichnet wird. In der Matrix $K$ werden zusätzlich bestimmte Zeilen zyklisch verschoben, sodass die zyklische Randbedingungen der Faltung im Zeitbereich eingehalten werden. Stehen die zweidimensionalen diskreten Fourier-Transformationen $\hat{X}=\mathrm{2DFT}(X)$ und $\hat{K}=\mathrm{2DFT}(K)$ zur Verfügung, so gilt mit dem Faltungstheorem \ref{satz:conv_theorem} der Zusammenhang
    \begin{align*}
        \mathrm{2DFT}(Y)&=\mathrm{2DFT}(X) \odot \mathrm{2DFT}(K). 
        %\Leftrightarrow &=\mathrm{i2DFT}(\hat{X} \odot \hat{K}).
    \end{align*}
    % bereits um 180 Grad gedreht, vgl. Bemerkung \ref{bem:K_conv_komp}, 
\end{bem}
Für eine detailliertere Beschreibung der Konstruktion der eingebetteten Matrizen $K \in \RR^{n \times n}$ sei aufgrund dessen Umfangs auf Jain \cite{jain1989fundamentals} verwiesen. Eine Implementierungsmöglichkeit \ref{matlab:boundary} in MATLAB ist im Anhang gegeben.
Die Matrixfaltung innerhalb einer Faltungsschicht eines CNN lässt sich mit den obigen Resultaten in drei Schritten berechnen.
\begin{itemize}
    \item[1.] Es sind jeweils die zweidimensionalen diskreten Fourier-Transformationen $\hat{X}$ und $\hat{K}$ für die Eingabe $X$ und den gedrehten Kern $K$ zu berechnen.
    \item[2.] Die Matrix $\hat{Y}= \hat{X} \odot \hat{K}$, welche sich aus der elementweisen Multiplikation ergibt, ist zu bestimmen.
    \item[3.] Schließlich stellt $Y=\mathrm{i2DFT}(\hat{Y})$ die Matrixfaltung dar, welche mithilfe der inversen diskreten Fourier-Transformation ermittelt wird.    
\end{itemize} 
Für Schritt 1 und Schritt 3 kann wegen Lemma \ref{lem:Finv} und Lemma \ref{lem:inversedftasdft} die Fourier-Matrix $F$ benutzt werden. Die Aufgabe besteht nun darin, die Berechnungen in Algorithmus \ref{alg:conv_as_2dft} datenbankgestützt umzusetzen. 
\begin{algorithm}[h]
    \caption{Matrixfaltung mit diskreten Fourier-Transformationen}
    \label{alg:conv_as_2dft}
    \begin{algorithmic}
    \Require  Eingabematrix $X \in \RR^{n \times n}$, eingebetteter Kern $K \in \RR^{n \times n}$, Fourier-Matrix $F \in \RR^{n \times n}$ 
    \Ensure Matrixfaltung $Y= X \ast K$
    \State Berechne die zweidimensionalen diskreten Fourier-Transformationen:
    \State $\hat{X}=F X F^T$
    \State $\hat{K}=F K F^T$ 
    \State Berechne das Produkt mit der elementweisen Multiplikation:
    \State $\hat{Y}= \hat{X} \odot \hat{K}$
    \State Bestimme die inverse Fourier-Transformation:  
    \State $Z=\hat{Y}^*$
    \State $\hat{Z}=F Z F^T$
    \State $Y=\frac{1}{n^2}\hat{Z}^*$
    \end{algorithmic}
\end{algorithm}

Dies gelingt, da ausschließlich Basisoperationen wie die Matrixmatrixmultiplikation sowie das Adjungieren von Matrizen benötigt werden. Die elementweise Multiplikation und die Konjugation können als einfache skalare Funktionen implementiert werden. Die Berechnung der eindimsenionalen DFT mit der Matrix $F$ benötigt $n(2n-1)$ Fließkomma-Operationen. Da die Matrix $F$ nur von den Dimensionen der beteiligten Matrizen abhängt und diese durch die Hyperparameter des verwendeten CNN festgelegt sind, kann $F$ vor der Vorwärtsrechnung in einer Relation gespeichert werden. 
Die Matrix $F$ besitzt komplexwertige Einträge und daher wird das Attribut \textbf{v} im Coordinate-Schema in die Attribute \textbf{re} und \textbf{im} aufgeteilt, um den Real- und Imaginärteil getrennt zu speichern. Die Multiplikation zweier komplexer Zahlen $z_1$ und $z_2$ ist durch
\begin{align*}
    z_1 \cdot z_2 &=(\Re(z_1)+ i \Im(z_1))(\Re(z_2)+i \Im(z_2))\\
    &=(\Re(z_1) \cdot \Re(z_2)-\Im(z_1) \cdot \Im(z_2))+ i (\Re(z_1) \cdot \Im(z_2)+ \Im(z_1) \cdot \Re(z_2))
\end{align*}
erklärt.
Darüber hinaus können auch die Fourier-Transformierten $\hat{K}$ für alle beteiligten Kerne $K$ bereits vor der Erkennungsphase bestimmt werden. So kann die Berechnungszeit verkürtzt werden.

Seien $X \in [0,1]^{n \times n}$ und \textbf{X} sowie \textbf{F} die Relation zur Darstellung der Matrizen $X$ und $F$ im Coordinate-Schema. Im ersten Berechnungsschritt ist lediglich die Matrix $\mathrm{2DFT}(X)$ zu bestimmen. Die entsprechende SQL-Anfrage \ref{sql:dft} lässt sich formulieren. Dabei werden \textit{Common-Table-Expressions}, zu Erkennen am Schlagwort \textbf{WITH}, welche als temporäre Tabellen zu verstehen sind, verwendet. So ist es möglich, zum einen Zwischenergebnisse übersichtlich darzustellen und zum anderen jene Ergebnisse weiter zu verwenden. Damit gelingt die iterative Berechnung der Matrixmatrixprodukte. Die erste temporäre Tabelle \textbf{FT}
in den Zeilen 1-6 ermittelt die Matrix $F^T$. In den Zeilen 7-15 wird $X F^T$ und anschließend in den Zeilen 16-24 das Produkt $F X F^T$ in der temporären Relation \textbf{FXFT} berechnet. 
Schließlich folgt in den letzten Zeilen die Ausgabe der zweidimensionalen diskreten Fourier-Transformation. Numerische Resultate zum Zeit- und Speicheraufwand sind der Abbildung \ref{abb:dft_insql} zu entnehmen.

%\noindent \begin{minipage}{\linewidth}
    \lstinputlisting[label=sql:dft, caption=SQL-Code zur Umsetzung der zweidimensionalen diskreten Fourier-Transformation, language=SQL]{sql_code/dft_ansatz_FXFT.txt}
%\end{minipage}

Im zweiten Berechnungsschritt ist eine elementweise Multiplikation auszuführen. Angenommenen die Matrizen $\hat{X}$ und $\hat{K}$ sind als Relationen \textbf{X}
und \textbf{K} im Coordinate-Schema hinterlegt. Dann wird $\hat{X} \odot \hat{K}$ in der SQL-Anfrage \ref{sql:elementweise} implementiert.
Der dritte Berechnungsschritt lässt sich analog zum ersten Schritt darstellen. Dabei ist an zwei Stellen lediglich das Adjungieren einer komplexen Matrix, siehe Anhang \ref{app:app_1}, notwendig.

\lstinputlisting[label=sql:elementweise, caption=SQL-Code zur Umsetzung der elementweisen Multiplikation, language=SQL]{sql_code/dft_ansatz_elemwise.txt}

Wird zur Berechnung der DFT die sogenannte schnelle Fourier-Transformation (engl. \textit{Fast-Fourier-Transformation}, kurz: FFT) genutzt, können die Zeitkosten von $\mathcal{O}(n^2)$ auf $\mathcal{O}(n \log n)$ vermindert werden. In dieser Arbeit wird ausschließlich die diskrete zweidimensionale Fourier-Transformation wie in Algorithmus \ref{alg:conv_as_2dft} behandelt. 

\subsection{Zusammenfassung}
\label{abs:sum_conc_in_sql}
TODO

\section{Datenbankgestützte Vorwärtsrechnung für CNN}
\label{abs_CNN_in_SQL}
In diesem Abschnitt steht das Problem \ref{prob:ffCCN} im Vordergrund und es wird eine relationale Umsetzung der Vorwärtsrechnung für ein trainiertes CNN-Modell diskutiert. In Abhängigkeit von den Hyperparameter des verwendeten Modells und dem Anwendungsszenario sind die im Abschnitt \ref{abs:conv_in_sql} vorgestellten Resultate zu nutzen.    
Die konkreten Implementierungen in SQL beziehen sich auf das im Abschnitt \ref{abs:model_mnist} vorgestellte Modell \ref{modell}, welches als laufende Beispielmodell dienen soll. In diesem Abschnitt tauchen immer wieder Modellparameter auf, welche als traniert und damit optimal im Sinne der gestellten Klassifikationsaufgabe voraussgesetzt werden. Nur dann ist eine sinnvolle datenbankgestützte Mustererkennung möglich.  

\subsection*{Gefaltete Übertragungsfunktion}
Bei ML-Verfahren wird oft mit mehrdimensionalen Arrays gearbeitet. Diese sind problemlos mit dem Relationenmodell vereinbar. Das Coordinate-Schema wird einfach mit entsprechenden Attributen, welche die Achsen darstellen, erweitert. Für einen Kern $K \in \RR^{q \times p \times k \times k}$ ergibt sich die Relation \textbf{K} zu

\begin{align*}
    \mathbf{K}\, ( 
    &\underline{q} \; \; \mathrm{int},\\
    &\underline{p} \; \; \mathrm{int},\\
    &\underline{i} \; \; \mathrm{int}, \\    
    &\underline{j} \; \; \mathrm{int}, \\
    &v \; \; \mathrm{double}),
\end{align*}
wobei $q$ und $p$ die neuen Achsen präsentieren. In diesem Kontext wird von einem erweiterten Coordinate-Schema gesprochen.

Zur Berechnung der Merkmalskarten durch die gefaltete Übertragungsfunktion, vgl. Definition \ref{eq:convlogit} sind (gewichtete) Summen über Matrixfaltungen zu bestimmen. Um dies kompakt darzustellen, werden rekursive Anfragen benutzt, welche seit 1999 in SQL standardisiert sind. An dieser Stelle sei bemerkt, dass je nach Wahl des Datenbankmanagementsystems diese Anfragen unterstützt beziehungsweise nicht unterstützt werden. In den meisten kommerziellen und freien Systemen, unter anderem PostgreSQL, sind rekursive Anfragen möglich. Ein Beispiel zur rekursiven Berechnung von
\begin{equation}
   \label{eq:sum_n}
\sum_{n=1}^{100} n    
\end{equation} 
ist in Anfrage \ref{sql:recursive} gegeben. 

\lstinputlisting[label=sql:recursive, caption=Ein Beispiel einer rekursiven SQL-Anfrage, language=SQL]{sql_code/recursive_sum_n.txt}

Eine rekursive Anfrage besteht immer aus einem nicht-rekursiven Term (im Beispiel Zeile 3), der Vereinigung \textbf{UNION} bzw. \textbf{UNION ALL} und einem rekursiven Term (Zeile 4-6), welcher den Selbstverweis auf die entsprechende Relation beinhaltet. Der \textbf{UNION ALL}-Operator entfernt im Gegensatz zum \textbf{UNION}-Operator keine Duplikate. Die erste \textbf{SELECT}-Klausel definiert die Rekursionsbasis und darf sich nicht auf die zu definierende Relation, im Beispiel \textbf{t}, beziehen. Durch diese Anweisung werden zudem die Datentypen der Attribute der zu definierenden Relation festgelegt. In den Zeilen 7 und 8 wird die Summe (\ref{eq:sum_n}) über die Partialsummen in der Relation \textbf{t} berechnet. 

Bei der gefalteten Übertragungsfunktion \ref{eq:convlogit} mit einer Aktivierungfunktion $\psi$ ist eine Summe der Form
\begin{equation}
    \label{eq:rec_conv}
    \psi \left(\sum_{p=1}^{z_{in}} \alpha_{qp} A_p+ b_q \mathbf{1} \right)
\end{equation}
zu berechnen. Dabei sind die Matrizen $A_p$ Ergebnisse von Matrixfaltungen, $\alpha_{qp} \in \RR$ beliebige Gewichte sowie $b \in \RR^{z_{out}}$ ein Biasvektor. Zukünftig werden Aktivierungsfunktionen im SQL-Code mit $T$ bezeichnet. Zur Berechnung von (\ref{eq:rec_conv}) wird eine rekursive SQL-Anfrage \ref{sql:conv_rec} genutzt. Die entsprechenden Relationen liegen jeweils im erweiterten Coordinate-Schema vor. In der Anfrage gilt für alle Gewichte $\alpha_{qp}=1$. Die Erweiterung für beliebige Gewichte kann mit einer zusätzlichen Relation für die Gewichtsparameter gelingen. 

\lstinputlisting[label=sql:conv_rec, caption=Die rekursive SQL-Anfrage zur Berechnung der gefalteten Übertragungsfunktion, language=SQL]{sql_code/conv_recursive.txt}

Die datenbankgestützte Matrixfaltung ist bereits im vorherigen Abschnitt \ref{abs:conv_in_sql} beleuchtet worden. Zur Veranschaulichung wird die SQL-Implementierung \ref{sql:C1} zur Berechnung der sechs Merkmalskarten in der Faltungschicht $C^1$ des Modells \ref{modell} dargestellt. Dabei wurde der im Abschnitt \ref{abs:naive_app} diskutierte verbesserte Nachbarschaftsansatz mit Umgebungstabellen und das erweiterte Coordinate-Schema benutzt. Hier ist im Gegensatz zur Schicht $C^2$ keine rekursive Berechnung notwendig. Die rekursiven Anfragen für $C^2$ sind im Anhang \ref{app:app_2} zu finden. Es kommen wieder Common-Table-Expressions zum Einsatz. In den Zeilen 1-19 werden die Matrixfaltungen $X \ast K^{1}_p$ mit zero-padding mithilfe einer Nachbarschaftstabelle \textbf{U} für $1 \leq p \leq 6$ berechnet. In den Zeilen 20 bis 27 wird nur das Teilergebnis behalten, welches ohne zero-padding bestimmt wurde. Schließlich folgt in den Zeilen 28-34 die Manipulation mit den Schwellwerten $b_p$ und die Berechnung der Aktivierung mit der logistischen Funktion.

\lstinputlisting[label=sql:C1, caption=Die SQL-Anfrage zur Berechnung von $C^1$, language=SQL]{sql_code/C1_sql.txt}

\subsection*{Pooling}
Die datenbankgestützte Umsetzung von Pooling-Schichten kann direkt durch die Nutzung von Nachbarschaften wie in Abschnitt \ref{abs:naive_app} gelingen. Da die Dimensionen aller vorkommenden Merkmalskarten und Kerne eines CNN durch die Wahl der Hyperparameter festgelegt werden, müssen die Nachbarschaften beim Pooling nur einmalig berechnet werden. Des Weiteren werden Funktionen wie \textbf{MAX} und \textbf{MEAN} im SQL-Standard unterstützt, sodass Maximum-Pooling und Mittelwert-Pooling implementiert werden können.

Eine weitere Möglichkeit besteht darin, das Mittelwert-Pooling mit den Schrittweiten $p:=p_h=p_b$ als Faltungsoperation mit dem sogenannten Mittelwert-Kern $\frac{1}{p^2} \mathbf{1} \in \RR^{p \times p}$ zu implementieren. An dieser Stelle können Resultate aus dem vorherigen Abschnitt \ref{abs:conv_in_sql} genutzt werden, um abzuschätzen, welcher Ansatz für das Anwendungsszenario besser geeignet ist. Schließlich ist auch die Flatten-Funktion, vgl. Definition \ref{def:flatten}, leicht in SQL umsetzbar. 

Eine Kombination aus Mittelwert-Pooling mit Schrittweite $p=2$ und der Flatten-Funktion $T_f$ ist für das konkrete Modell \ref{modell} in der SQL-Anfrage \ref{sql:pool_flatten} gegeben. Alle beteiligten Relationen liegen entsprechend ihrer Namen im erweitertem Coordinate-Schema vor. Es wird der oben diskutierte Nachbarschaftsansatz gewählt.

\lstinputlisting[label=sql:pool_flatten, caption=SQL-Code zur Umsetzung des Mittelwert-Poolings und anschließender Flatten-Operation, language=SQL]{sql_code/pool_flatten.txt}

In den Zeilen 1 bis 11 wird das Mittelwert-Pooling mithilfe der Aggregatfunktion \textbf{AVG} und der Nachbarschaften in der Relation \textbf{U} bestimmt. Dieses Ergebnis wird mit den Schrittweiten $s_h=s_b=2$, vgl. Bemerkung \ref{bem_strides}, in den Zeilen 12 bis 19 in der Dimension verkleinert. Dabei werden Funktionen wie das Aufrunden \textbf{CEIL} und Modulo-Rechnen \textbf{MOD} zur passenden Selektierung der Indizes genutzt. Die Doppelpunkte sorgen dafür, dass die Indizes zu natürlichen Zahlen umgewandelt werden. Im Englischen wird diese Typumwandlung als \textit{type-casting} bezeichnet. Alle verwendeten Operationen werden im SQL-Standard unterstützt. Schließlich wird der Vektor $f=T_f(P^2) \in \RR^{192}$ der Flatten-Funktion in den Zeilen 20 bis 26 berechnet. 
 
\subsection*{Vorwärtsgerichtete neuronale Netze}
Die letzte Schicht eines CNN besteht meistens aus einem FFN mit einer oder mehreren verdeckten Neuronenschichten. Die Theorie von FFN wird in Kapitel \ref{kap:NN} vorgestellt. Dabei wird die Ausgabe eines neuronalen Natzes mittels Vorwärtsrechnung, vgl. Algorithmus \ref{alg:ff}, berechnet. Diese lässt sich ausschließlich mit bereits vorgestellten Basisoperationen wie der Matrixvektormultiplikation bzw. Vektoraddition und einfachen Funktionsauswertung darstellen. Wird das Modell \ref{modell} zur Klassifikation von Ziffern genutzt, so ist die Vorwärtsrechnung lediglich für eine Neuronenschicht datenbankgestützt umzusetzen. Die Verallgemeinerung für beliebig viele Schichten wurde bereits in einer Projektarbeit der Universität Rostock untersucht \cite{myprojekt}. 

Sei die Gewichtsmatrix $W \in \RR^{192 \times 10}$ sowie der Biasvektor $b \in \RR^{10}$ wie in Abschnitt \ref{abs:model_mnist} gegeben. Die entsprechenden Relationen \textbf{W} und \textbf{B} werden gemäß dem Coordinate-Schema erstellt. Weiter sei $T$ eine beliebige Aktivierungsfunktion. Beim Modell \ref{modell} ist $T$ die logistische Funktion. Schließlich sei für eine Eingabekarte $X$ in der Relation \textbf{FLATTEN} der Vektor $f=T_f(X)$ der Flatten-Schicht gespeichert, welcher als Eingabe des FFN dient. Die Ausgabe des FFN und damit des gesamten CNN lässt sich durch die SQL-Anfrage \ref{sql:ffn} berechnen. In den Zeilen 1-9 wird das Matrixvektorprodukt $Wf$ berechnet, welches anschließend mit dem Biasvektor $b$ manipuliert wird. Schließlich werden die Aktivierungen der Ausgabeschicht mit der Funktion $T$ ermittelt. In den Zeilen 10-15 wird der Index mit der maximalen Aktivierung selektiert, welcher, vermindert um Eins, die Klassifikation der Ziffer darstellt.

\lstinputlisting[label=sql:ffn, caption=SQL-Code zur Umsetzung der Vorwärtsrechnung des einschichtigen FFN aus Modell \ref{modell}, language=SQL]{sql_code/ffn_in_sql.txt}

Die vollständige datenbankgestützte Vorwärtsrechnung für das tranierte Modell \ref{modell} ist im Anhang \ref{app:app_2} dargestellt. LAUFZEIT eval TODO, TODO full pass im Anhang
\chapter{Zusammenfassung und Ausblick}
\label{kap:sum}
%\chapter{Die Faltung}
In diesem Kapitel wird erläutert, wie die Faltung (engl. \textit{convolution}) bei CNNs zu verstehen ist und wie diese Operation bei diesen neuronalen Netzen motiviert wird. In diesem Zusammenhang werden Begriffe wie Merkmalskarten (engl. \textit{feature maps}) und Filter (engl. \textit{kernels}) eingeführt. Des Weiteren wird die Arithmetik der Faltungsoperation für zweidimensionale Eingaben, repräsentiert durch Matrizen, erklärt und das Verfahren \textit{padding} beziehungsweise das Nutzen von \textit{strides} erläutert. Das gesamte Kapitel wird mit konkreten Beispielen begleitet, um die verschiedenen Effekte der Faltungsoperation zu beleuchten.

\section{Die Faltungsoperation}
In der Analysis ist die Faltung ein mathematischer Operator und liefert für zwei Funktionen $f$ und $g$ die Funktion $ f \ast g$, wobei mit dem Sternchen die Faltungsoperation gemeint ist.

\begin{defi}[Faltung]\label{allg_faltung}
    Für zwei Funktionen $f,g: \Rnv \rightarrow \mathbb{C}$ ist die Faltung als
    \begin{equation*}
        (f \ast g) (x) := \int_{\Rnv} f(\tau) g(x-\tau) \mathrm{d} \tau
    \end{equation*}
    definiert, wobei gefordert wird, dass das Integral für fast alle $x$ wohldefiniert ist.
   \end{defi}

Bei klassischen neuronalen Netzen, siehe Kapitel \ref{classicNN} werden Eingabedaten durch eine Verkettung von affinen Transformationen verarbeitet. Typischerweise wird die Eingabe als Vektor dargestellt und  mit einer Matrix multipliziert, gegebenenfalls mit einem Biasvektor manipuliert und schließlich so die Ausgabe generiert. Bilder-, Audio- oder Videoaufnahmen besitzen jedoch mehrere Merkmale in unterschiedlichen Achsen. Oft sind solche Eingabedaten im Bereich des Machine-Learnings als mehrdimensionale Arrays abgelegt, welche eine oder mehrere Achsen repräsentieren, wobei die Ordnung dieser eine Rolle spielt. Bei digitalisierten Bildern sind das bespielsweise die Höhe und Breite des Bildes, bei Audioaufnahmen gibt es nur eine Achse, und zwar die Zeitachse. Hinzu kommen Kanalachsen als weitere Verfeinerung der Daten, zum Beispiel besitzen RGB-Farbbilder drei Kanäle der Farben rot, grün und blau. 

Diese speziellen Eigenschaften können bei affinen Transformationen nicht berücksichtigt werden. Alle Merkmale sowie Achsen werden gewissermaßen gleich behandelt und die wesentliche topologische Struktur kann so nicht zum Vorteil ausgenutzt werden. Hier soll nun die sogenannte diskrete Faltung Abhilfe schaffen.

\begin{defi}[Diskrete Faltung]\label{disk_faltung}
    Für zwei Funktionen $f,g: D \rightarrow \mathbb{C}$ mit einem diskreten Definitionsbereich $D \subseteq \mathbb{Z}^n$ ist die diskrete Faltung als
    \begin{equation*}
        (f \ast g) (n) := \sum_{k \in D} f(k) g(x-k)
    \end{equation*}
    definiert. Hier wird über dem gesamten Definitionsbereich $D$ summiert. Ist $D$ beschränkt, werden $f$ beziehungsweise $g$ durch Nullen fortgesetzt. 
   \end{defi}

Besonders bei der Bildverarbeitung wird oft die diskrete Faltung als lineare Operation verwendet. 

\begin{defi}[Matrixfaltung, vgl. \cite{gruening}] \label{matrix_faltung}
    Für gegebene Matrizen $X \in \RR^{h \times w}$ und $K \in \RR^{k_h \times k_w}$ seien 
    \begin{equation*}
        h_l=\begin{cases}
           \lfloor k_h/2  \rfloor &, k_h \, \text{ungerade} \\
           k_h/2-1 &, \text{sonst}
        \end{cases}, \; \; 
        w_l=\begin{cases}
            \lfloor k_w/2 \rfloor &, k_w \, \text{ungerade} \\
            k_w/2 -1 &, \text{sonst}    
        \end{cases}.
    \end{equation*} 
    Die Matrixfaltung $K \ast X \in \RR^{h \times w}$ ist als 
    \begin{equation}
        \label{matrix_faltung_op}
        (K \ast X)_{i,j}:=\sum_{l=-h_l}^{\lfloor k_h/2  \rfloor} \sum_{m=-w_l}^{\lfloor k_w/2 \rfloor} K_{l+h_l+1, m+w_l+1} X_{i+l,j+m} \; \; \forall i \in [h], j \in [w]
    \end{equation} mit $X_{i,j}=0$ für $i \notin [h]$ und $j \notin [w]$ definiert.
    \end{defi}

\begin{bem}\label{bem_strides}
    Die Matrixfaltung kann durch viele weitere Parameter genauer spezifiziert werden. Sogannte \textit{strides} bestimmen die Reduktion bei der Faltung von $X$ in der Höhe $h$ beziehungsweise Breite $w$. Für strides $s_h, s_w \in \mathbb{N}$ ist die Matrixfaltung als
    \begin{equation*}
        (K \ast X)_{i,j}:=\sum_{l=-h_l}^{\lfloor k_h/2  \rfloor} \sum_{m=-w_l}^{\lfloor k_w/2 \rfloor} K_{l+h_l+1, m+w_l+1} X_{i \cdot s_h +l,j \cdot s_w +m} \; \; \forall i \in [h], j \in [w].
    \end{equation*}
    Für $s_h=s_w=1$ ergibt sich die Standardvariante wie in \ref{matrix_faltung_op}.
    \end{bem}

Im Folgenden werden konkrete Beispiele für verschiedene zweidimensionale Faltungen, welche in dieser Arbeit im Fokus stehen, gegeben. Dabei sind die Eingabe $X \in \RR^{h \times w}$ und der Filter $K \in \RR^{k_h \times k_w}$ immer als Matrizen zu verstehen. Das Ergebnis der Faltung $S= X \ast K$ wird als Merkmalskarte bezeichnet. Es sei angemerkt, dass oft $k_h=k_w$ sowie $k_h$ ungerade gewählt wird, z.B. $k_h=3$ oder $k_h=5$. Die Größe der Merkmalskarte wird durch die Parameter
\begin{itemize}
    \item $h,w$: Die Höhe und Breite der Eingabe,
    \item $k_h, k_w$: Die Abmessungen des Filters,
    \item $s_h, s_w$: Die Wahl der strides, 
    \item $p_h, p_w$: Die Größe des zero paddings
\end{itemize}
beeinflusst. Mit zero padding ist gemeint, dass künstliche Nullen um Randpixel der Eingabe $X$ eingefügt werden, damit die Berechnung mit dem Filter um jene Pixel gelingt. Ein Beispiel für das Verwenden von zero padding wird in Abbildung \ref{abb_simplematrixconv_padding} gezeigt. In Abbildung \ref{abb_simplematrixconv} ist die Berechnung einer einfachen zweidimensionalen Matrixfaltung dargestellt. Ein vorher festgelegter Filter (grau) bewegt sich über die Eingabe (blau) und berechnet jeweils die Einträge der Ausgabe(grün). 

\begin{figure}[h]
    \includegraphics[width=0.8\textwidth]{pics/abb_simpleconv}
    \centering
    \caption{Es wird die Merkmalskarte $S \in \RR^{3 \times 3}$ mit den Parametern ${h,w=5}, k_h=k_w=3, s_h=s_w=1$ und $p_h=p_w=1.$}
    \label{abb_simplematrixconv}
\end{figure}

\begin{figure}[h]
    \includegraphics[width=0.8\textwidth]{pics/abb_simplecov_padding}
    \centering
    \caption{Es wird die Merkmalskarte $S \in \RR^{3 \times 3}$ mit den Parametern ${h,w=5}, k_h=k_w=3, s_h=s_w=2$ und $p_h=p_w=1$ berechnet.}
    \label{abb_simplematrixconv_padding}
\end{figure}



\section{Motivation der Faltung}
..
Sie nutzt wichtige Konzepte zur Optimierung von Machine-Learning-Verfahren wie spärliche Konnektivität (engl. \textit{sparse connectivity}), \textit{Parameter Sharing} und \textit{äquivariante Repräsentation}, vgl. \cite{goodfellow}. Spärlicher Konnektivität bedeutet, dass die Ausgabeeinheit auf einer bestimmten Schicht nur durch wenige Eingabeeinheiten beeinflusst wird. Dies ist bei CNNs typisch, da meist die verwendeten Filter viel kleiner als die Eingabe ist. Noch mehr erklären + Abbildung

Mit Parameter Sharing ist die Nutzung von gleichen Parametern für mehrere Funktionen im neuronalen Netz gemeint. In herkömmlichen Feed-Forward-Netzen wird jedes Element der Gewichtsmatrizen für die Berechnung der Aktivierungen der jeweiligen Schichten verwendet. Anschließend werden diese Gewichte dann nicht mehr gebraucht. Im Zusammenhang von CNNs bedeutet Parameter Sharing während der Faltungsoperation, dass nur eine bestimmte Menge von Parametern erlernt werden müssen
Noch mehr erklären + Abbildung



%\chapter{Weiteres Kapitel}
%Hier wird dies und das vorgestellt. Unter anderem Fußnoten.\footnote{Dies ist eine Fußnote.}
%\section{Umgebungen und Formeln}


\printbibliography %hier Bibliographie ausgeben lassen

\appendix
\chapter{Anhang}

% \tocless
\section{Weitere Basisoperation in SQL}
\label{app:app_1}
Die euklidische Norm 
\begin{equation*}
    ||x||_{2}:=\left(\sum_{i=1}^n x_i^2\right)^{\frac{1}{2}}
\end{equation*}
eines Vektors $x \in \RR^n$ und die Frobeniusnorm 
\begin{equation*}
    ||A||_F:=\left(\sum_{i=1}^m \sum_{j=1}^n a_{ij}^2\right)^{\frac{1}{2}} 
\end{equation*}
einer Matrix $A \in \RR^{m \times n}$ sind als SQL-Anfrage durch
\begin{align*}
    & \mathbf{SELECT} \; \mathbf{sqrt}(\mathbf{SUM}(v*v)) \; \mathbf{AS} \; 2Norm \\
    & \mathbf{FROM} \; x
\end{align*}
beziehungsweise
\begin{align*}
    & \mathbf{SELECT} \; \mathbf{sqrt}(\mathbf{SUM}(v*v)) \; \mathbf{AS} \; FNorm \\
    & \mathbf{FROM} \; A
\end{align*}
gegeben.
%skalaprodukt
Im euklidischen Vektorraum $\RR^n$ ist das Skalarprodukt $\langle x,y \rangle$ zweier Vektoren $x,y \in \RR^n$ definiert als 
\begin{equation*}
    \langle x,y \rangle :=\sum_{i=1}^n x_i y_i.
\end{equation*}
Eine entsprechende Transformation in SQL mit dem Aggregationsoperator \textbf{SUM} lautet
\begin{align*}
    & \mathbf{SELECT} \; \mathbf{SUM}(x.v*y.v) \; \mathbf{AS} \; v \\
    & \mathbf{FROM} \; x \; \mathbf{JOIN} \; y \; \mathbf{ON} \; x.i=y.i
\end{align*}
Die Transponierte Matrix $A^T \in \RR^{n \times m}$ einer Matrix $A \in \RR^{m \times n}$ kann durch die Anfrage 
\begin{align*}
    & \mathbf{SELECT} \; A.j \; \mathbf{AS} \; i, \; A.i \; \mathbf{AS} \; j, \; A.v\\
    & \mathbf{FROM} \; A
\end{align*}
berechnet werden.

Sei $A \in \mathbb{C}^{m \times n}$ eine Matrix im erweitertem Coordinate-Schema. Die adjungierte Matrix $A^* \in \mathbb{C}^{n \times m}$ kann durch die SQL-Anfrage
\begin{align*}
    & \mathbf{SELECT} \; A.j \; \mathbf{AS} \; i, \; A.i \; \mathbf{AS} \; j, \; A.re \;\mathbf{AS} \; re, \; -(A.im) \; \mathbf{AS} \; im  \\
    & \mathbf{FROM} \; A
\end{align*}
berechnet werden.
% \tocless
%\section{Überschrift A2}
%\label{sec:app_2}
%blablabla
%Wenn gewünscht Danksagung einfügen
%\addchap*{Danksagungen}

%%%%%%%%%%%%%%%%%%%%%%%%%%%%%%%%%%%%%%%%%%%%%%%%%
%%%%%%%%%%%%%%%%%% Formalia %%%%%%%%%%%%%%%%%%%%%
%%%%%%%%%%%%%%%%%%%%%%%%%%%%%%%%%%%%%%%%%%%%%%%%%
% Muss in jedem Fall in die Arbeit!
\eigenstaenigkeitserklaerung

\cleardoublepage

%%%%%%%%%%%%%%%%%%%%%%%%%%%%%%%%%%%%%%%%%%%%%%%%%

\end{document}
