\chapter{Einleitung}
Es mag euch wundern, dass die Einleitung in einem separaten File abgelegt ist. Dies muss natürlich nicht so sein. Es könnte aber bei einer langen Abschlussarbeit durchaus die Übersichtlichkeit erhöhen, wenn ihr für verschiedene Kapitel einzelne Dateien anlegt und diese mittels 
\begin{verbatim}\input{<DateiName>}\end{verbatim}

\bigskip
oder

\bigskip
\begin{verbatim}\include{<DateiName>}\end{verbatim}
einfügt.

\bigskip
\noindent\textcolor{red}{Verwendet keine Umlaute oder Leerzeichen in Dateinamen.}

\noindent\texttt{input} fügt den Text direkt an die Stelle des \texttt{input}-Befehls ein.

\noindent\texttt{include} fügt den Text auf einer neuen Seite ein. 
